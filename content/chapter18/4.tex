
C++20提供了一个新的辅助函数,允许开发者为编译时和运行时计算实现不同的代码:std::is\_constant\_evaluate()。其在头文件<type\_traits>中定义(尽管它实际上不是一个类型函数)。

其允许代码只能在运行时调用的辅助函数,再切换到可以在编译时使用的代码。例如:

\filename{comptime/isconsteval.hpp}

\begin{cpp}
#include <type_traits>
#include <cstring>

constexpr int len(const char* s)
{
	if (std::is_constant_evaluated()) {
		int idx = 0;
		while (s[idx] != '\0') { // compile-time friendly code
			++idx;
		}
		return idx;
	}
	else {
		return std::strlen(s); // function called at runtime
	}
}
\end{cpp}

函数len()中,计算原始字符串或字符串字面值的长度。若在运行时调用,则使用标准的C函数strlen()。但为了在编译时使用该函数,若在编译时上下文中,则提供不同的实现。

下面是这两个分支的调用方式:

\begin{cpp}
constexpr int l1 = len("hello"); // uses then branch
int l2 = len("hello"); // uses else branch (no required compile-time context)
\end{cpp}

len()的第一次调用发生在编译时上下文中,is\_constant\_evaluate()产生true,因此可使用then分支。len()的第二次调用发生在运行时上下文中,因此\_constant\_evaluate()产生false并调用strlen()。即使编译器决定在编译时对调用求值,后一种情况也会发生。重要的一点是这些调用是否需要在编译时发生。

下面是一个相反的函数:在编译时和运行时都将整型值转换为字符串:

\filename{comptime/asstring.hpp}

\begin{cpp}
#include <string>
#include <format>

// convert an integral value to a std::string
// - can be called at compile time or runtime
constexpr std::string asString(long long value)
{
	if (std::is_constant_evaluated()) {
		// compile-time version:
		if (value == 0) {
			return "0";
		}
		if (value < 0) {
			return "-" + asString(-value);
		}
		std::string s = asString(value / 10) + std::string(1, value % 10 + '0');
		if (s.size() > 1 && s[0] == '0') { // skip leading 0 if there is one
			s.erase(0, 1);
		}
		return s;
	}
	else {
		// runtime version:
		return std::format("{}", value);
	}
}
\end{cpp}

运行时,只需使用std::format()。编译时,可手动创建一个由可选的负号和所有数字组成的字符串(使用递归方法将它们按正确的顺序排列)。在关于将编译时字符串导出到运行时字符串的部分中,可以找到一个使用它的示例。


\mySubsubsection{18.4.1}{std::is\_constant\_evaluated()的详情}

根据C++20标准,std::is\_constant\_evaluate()在明显为常量求值的表达式或转换中调用时产生true。称其为:

\begin{itemize}
\item 
在常量表达式中

\item 
在常量上下文中(在if constexpr、consteval函数或常量初始化中)

\item 
用于可在编译时使用的变量的初始化器
\end{itemize}

例如:

\begin{cpp}
constexpr bool isConstEval() {
	return std::is_constant_evaluated();
}

bool g1 = isConstEval(); // true
const bool g2 = isConstEval(); // true
static bool g3 = isConstEval(); // true
static int g4 = g1 + isConstEval(); // false

int main()
{
	bool l1 = isConstEval(); // false
	const bool l2 = isConstEval(); // true
	static bool l3 = isConstEval(); // true
	int l4 = g1 + isConstEval(); // false
	const int l5 = g1 + isConstEval(); // false
	static int l6 = g1 + isConstEval(); // false
	int l7 = isConstEval() + isConstEval(); // false
	const auto l8 = isConstEval() + 42 + isConstEval(); // true
}
\end{cpp}

通过constexpr函数间接调用isConstEval(),也会得到相同的效果。

\mySamllsection{使用constexpr和consteval的std::is\_constant\_evaluated()}

例如,假设定义了以下函数:

\begin{cpp}
bool runtimeFunc() {
	return std::is_constant_evaluated(); // always false
}

constexpr bool constexprFunc() {
	return std::is_constant_evaluated(); // may be false or true
}

consteval bool constevalFunc() {
	return std::is_constant_evaluated(); // always true
}
\end{cpp}

然后有以下行为:

\begin{cpp}
void foo()
{
	bool b1 = runtimeFunc(); // false
	bool b2 = constexprFunc(); // false
	bool b3 = constevalFunc(); // true
	
	static bool sb1 = runtimeFunc(); // false
	static bool sb2 = constexprFunc(); // true
	static bool ab3 = constevalFunc(); // true
	const bool cb1 = runtimeFunc(); // ERROR
	const bool cb2 = constexprFunc(); // true
	const bool cb3 = constevalFunc(); // true
	
	int y = 42;
	static bool sb4 = y + runtimeFunc(); // function yields false
	static bool sb5 = y + constexprFunc(); // function yields false
	static bool ab6 = y + constevalFunc(); // function yields true
	const bool cb4 = y + runtimeFunc(); // function yields false
	const bool cb5 = y + constexprFunc(); // function yields false
	const bool cb6 = y + constevalFunc(); // function yields true
}
\end{cpp}

通过使用constexpr或静态constinit(而非const),不带y的初始化将具有与cb1、cb2和cb3相同的效果。但若涉及到y,使用constexpr或静态constinit总是会导致错误,因为涉及到运行时的值。

通常,在以下情况下使用std::is\_constant\_evaluate()是没有意义的。

\begin{itemize}
\item 
作为编译时if的条件,总是产生true:

\begin{cpp}
if constexpr (std::is_constant_evaluated()) { // always true
	...
}
\end{cpp}

在if constexpr中,有一个编译时上下文,所以是否处于编译时上下文的问题的答案总是为true(不管整个函数是从哪个上下文调用的)。

\item 
在纯运行时函数中,这通常会产生false。

唯一的例外是if is\_constant\_evaluate()用于局部常量求值:

\begin{cpp}
void foo() {
	if (std::is_constant_evaluated()) { // always false
		...
	}
	const bool b = std::is_constant_evaluated(); // true
}
\end{cpp}

\item 
在consteval函数中,总是产生true:

\begin{cpp}
consteval void foo() {
	if (std::is_constant_evaluated()) { // always true
		...
	}
}
\end{cpp}

\end{itemize}

因此,使用std::is\_constant\_evaluate()通常只在constexpr函数中有意义。在constexpr函数中使用std::is\_constant\_evaluate()来调用consteval函数也是没有意义的,通常不允许从constexpr函数调用consteval函数:[C++23可能会使用if consteval启用这样的代码。]

\begin{cpp}
consteval int funcConstEval(int i) {
	return i;
}

constexpr int foo(int i) {
	if (std::is_constant_evaluated()) {
		return funcConstEval(i); // ERROR
	}
	else {
		return funcRuntime(i);
	}
}
\end{cpp}

\mySamllsection{std::is\_constant\_evaluated()和三元操作符?:}

C++20标准中,有一个有趣的例子来说明如何使用std::is\_constant\_evaluate()。稍作修改,如下图所示:

\begin{cpp}
int sz = 10;
constexpr bool sz1 = std::is_constant_evaluated() ? 20 : sz; // true, so 20
constexpr bool sz2 = std::is_constant_evaluated() ? sz : 20; // false, so ERROR
\end{cpp}

这种行为的原因如下所示:

\begin{itemize}
\item 
sz1和sz2的初始化要么是静态初始化,要么是动态初始化。

\item 
对于静态初始化,初始化项必须是常量,所以编译器尝试用std::is\_constant\_evaluate()作为值为true的常量来计算初始化式。

\begin{itemize}
\item 
对于sz1,这是成功的。结果是1,这是一个常数,所以sz1是一个用20初始化的常数。

\item
对于sz2,结果是sz,不是常数。因此,sz2(理论上)是动态初始化的,先前的结果将丢弃,并使用std::is\_constant\_evaluate()对初始化器进行计算,结果反而产生false,则初始化sz2的表达式也是20。

但sz2不一定是常量,因为std::is\_constant\_evaluate()在此求值期间不一定是常量表达式,所以用这个20初始化sz2不能编译。
\end{itemize}
\end{itemize}

使用const(而不是constexpr)会使情况更加棘手:

\begin{cpp}
int sz = 10;
const bool sz1 = std::is_constant_evaluated() ? 20 : sz; // true, so 20
const bool sz2 = std::is_constant_evaluated() ? sz : 20; // false, so also 20

double arr1[sz1]; // OK
double arr2[sz2]; // may or may not compile
\end{cpp}

只有sz1是编译时常数,并且总是可以用来初始化数组。由于上述原因,sz2也可初始化为20。但由于初始值不一定是常量,因此arr2的初始化可能会编译,也可能不会编译(取决于所使用的编译器和优化)。


















