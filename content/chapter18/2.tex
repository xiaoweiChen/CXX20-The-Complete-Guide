
C++11起,有了关键字constexpr,支持在编译时对函数求值。若函数的所有方面在编译时已知,可以在编译时上下文中使用其结果。然而,constexpr函数也可以作为“普通”运行时函数使用。

C++20引入了一个类似的关键字consteval,要求进行编译时计算。与用constexpr标记的函数相反,用consteval标记的函数不能在运行时调用;相反,其需要在编译时调用。若这不可能,则程序处于病态。因为这些函数是立即调用的,所以当编译器看到对它们的调用时,这些函数也称为立即函数。

\mySubsubsection{18.2.1}{第一个consteval的示例}

考虑下面的例子:

\filename{comptime/consteval1.cpp}

\begin{cpp}
#include <iostream>
#include <array>

constexpr
bool isPrime(int value)
{
	for (int i = 2; i <= value/2; ++i) {
		if (value % i == 0) {
			return false;
		}
	}
	return value > 1; // 0 and 1 are not prime numbers
}
template<
int Num>
consteval
std::array<int, Num> primeNumbers()
{
	std::array<int, Num> primes;
	int idx = 0;
	for (int val = 1; idx < Num; ++val) {
		if (isPrime(val)) {
			primes[idx++] = val;
		}
	}
	return primes;
}

int main()
{
	// init with prime numbers:
	auto primes = primeNumbers<100>();
	for (auto v : primes) {
		std::cout << v << '\n';
	}
}
\end{cpp}

这里,使用consteval定义了函数primeNumbers<N>(),在编译时返回一个包含前N个素数的数组:

\begin{cpp}
template<int Num>
consteval
std::array<int, Num> primeNumbers()
{
	std::array<int, Num> primes;
	...
	return primes;
}
\end{cpp}

为了计算素数,primeNumbers()使用一个辅助函数isPrime(),该函数是用constexpr声明的,在运行时和编译时都可用(也可以用consteval声明,但这样它在运行时就不可用了)。

然后使用primeNumbers<>()初始化一个包含100个素数的数组:

\begin{cpp}
auto primes = primeNumbers<100>();
\end{cpp}

因为primeNumbers<>()是consteval,所以这个初始化必须在编译时进行。若使用constexpr声明primeNumbers<>(),并在编译时上下文中调用该函数(例如,初始化constexpr或constinit数组prime),也会有同样的效果:

\begin{cpp}
template<int Num>
constexpr
std::array<int, Num> primeNumbers()
{
	std::array<int, Num> primes;
	...
	return primes;
}
...
constinit static auto primes = primeNumbers<100>();
\end{cpp}

这两种情况下,当要为初始化计算的素数数量显著增加时,编译时间明显变慢。

但请注意编译器在计算常量表达式时通常有限制,C++标准只保证在一个核心常量exp内对1,048,576个表达式求值


\mySamllsection{consteval的Lambda}

现在也可以将Lambda声明为consteval,从而要求在编译时对Lambda求值。

考虑下面的例子:

\begin{cpp}
int main(int argc, char* argv[])
{
	// compile-time function to compute hash value for string literals:
	// (for the algorithm, see http://www.cse.yorku.ca/~oz/hash.html )
	auto hashed = [] (const char* str) consteval {
					std::size_t hash = 5381;
					while (*str != '\0') {
						hash = hash * 33 ^ *str++;
					}
					return hash;
				};

	// OK (requires hashed() in compile-time context):
	enum class drinkHashes : long { beer = hashed("beer"), wine = hashed("wine"),
									water = hashed("water"), ... };
	// OK (hashed() guaranteed to be called at compile time):
	std::array arr{hashed("beer"), hashed("wine"), hashed("water")};

	if (argc > 1) {
		switch (hashed(argv[1])) { // ERROR: argv is not known at compile time
			...
		}
	}
}
\end{cpp}

这里,用一个只能在编译时使用的Lambda初始化hash。因此,Lambda的使用必须在具有编译时值的编译时上下文中进行。这些调用只有在接受字符串字面值或constexpr const char*类型的形参时才有效。

若使用constexpr声明Lambda,则switch语句将生效。(没有必要使用constexpr声明,从C++17开始,所有的Lambda都是隐式的constexpr)。然而,这样就不能保证在编译时对arr的初始值进行计算。

有关更多细节,请参阅有关consteval Lambda的讨论的章节中的另一个例子。

\mySubsubsection{18.2.2}{constexpr和consteval}

使用constexpr和consteval,现在有以下选项来影响何时可以调用/被调用函数:

\begin{itemize}
\item
既不是constexpr也不是consteval:

这些函数只能在运行时上下文中使用,但编译器仍然可以在编译时进行优化。

\item
constexpr:

这些函数可以在编译时和运行时上下文中使用。即使在运行时上下文中,编译器仍然可以在编译时执行对函数求值的优化,编译器还可以在运行时对编译时上下文中的函数求值。

\item
consteval:

这些函数只能在编译时使用,但结果可以在运行时上下文中使用。
\end{itemize}

例如,考虑在头文件中声明的以下三个函数(因此,squareR()是用inline声明的):

\filename{comptime/consteval2.hpp}

\begin{cpp}
// square() for runtime only:
inline int squareR(int x) {
	return x * x;
}

// square() for compile time and runtime:
constexpr int squareCR(int x) {
	return x * x;
}

// square() for compile time only:
consteval int squareC(int x) {
	return x * x;
}
\end{cpp}

可以这样使用这些函数:

\filename{comptime/consteval2.cpp}

\begin{cpp}
#include "consteval2.hpp"
#include <iostream>
#include <array>

int main()
{
	int i = 42;

	// using the square functions at runtime with runtime value:
	std::cout << squareR(i) << '\n'; // OK
	std::cout << squareCR(i) << '\n'; // OK
	// std::cout << squareC(i) << ’’; // ERROR
	// using the square functions at runtime with c
	ompile-time value:
	std::cout << squareR(42) << '\n'; // OK
	std::cout << squareCR(42) << '\n'; // OK
	std::cout << squareC(42) << '\n'; // OK: square computed at compile time

	// using the square functions at compile time:
	// std::array<int, squareR(42)> arr1; // ERROR
	std::array<int, squareCR(42)> arr2; // OK: square computed at compile time
	std::array<int, squareC(42)> arr3; // OK: square computed at compile time
	// std::array<int, squareC(i)> arr4; // ERROR
}
\end{cpp}

constexpr和consteval的区别如下:

\begin{itemize}
\item
不允许consteval函数处理编译时未知的参数:

\begin{cpp}
std::cout << squareCR(i) << '\n'; // OK
std::cout << squareC(i) << '\n'; // ERROR
std::array<int, squareC(i)> arr4; // ERROR
\end{cpp}

\item
consteval函数必须在编译时执行计算:

\begin{cpp}
std::cout << squareCR(42) << '\n'; // may be computed at compile time or runtime
std::cout << squareC(42) << '\n'; // computed at compile time
\end{cpp}

\end{itemize}

在两种情况下使用consteval是有意义的:

\begin{itemize}
\item
希望强制执行编译时计算。

\item
希望禁用在运行时使用的函数。
\end{itemize}

例如,函数square()或hashed()是constexpr还是consteval在编译时上下文中没有区别:

\begin{cpp}
enum class Drink = { water = hashed("water"), wine = hashed("wine") };

switch (value) {
	case square(42):
	...
}

if constexpr(hashed("wine") > hashed("water")) {
	...
}
\end{cpp}

运行时上下文中,consteval可能会有所不同,因为那时不需要编译时计算:

\begin{cpp}
std::array drinks = { hashed("water"), hashed("wine") };

std::cout << hashed("water");

if (hashed("wine") > hashed("water")) {
	...
}
\end{cpp}

\mySubsubsection{18.2.3}{实践consteval}

对于consteval函数,实际使用时有一些限制。

\mySamllsection{consteval的限制}

consteval函数与constexpr函数几乎相同(注意,C++20放宽了这些限制):

\begin{itemize}
\item
参数和返回类型(若不是void)必须是文字类型。

\item
函数体只能包含既不是静态,也不是thread\_local的字面量类型的变量。

\item
不允许使用goto和标签。

\item
函数只能是构造函数或析构函数,类没有虚基类。

\item
consteval函数隐式内联。

\item
consteval函数不能用作协程。
\end{itemize}


\mySamllsection{使用consteval函数的调用链}

带有consteval标记的函数可以调用其他带有constexpr或consteval标记的函数:

\begin{cpp}
constexpr int funcConstExpr(int i) {
	return i;
}

consteval int funcConstEval(int i) {
	return i;
}

consteval int foo(int i) {
	return funcConstExpr(i) + funcConstEval(i); // OK
}
\end{cpp}

然而,constexpr函数不能为变量调用consteval函数:

\begin{cpp}
consteval int funcConstEval(int i) {
	return i;
}

constexpr int foo(int i) {
	return funcConstEval(i); // ERROR
}
\end{cpp}

函数foo()无法编译,原因是i仍然不是编译时值,可以在运行时调用。foo()只能用编译时变量调用funcConstEval():

\begin{cpp}
constexpr int foo(int i) {
	return funcConstEval(42); // OK
}
\end{cpp}

即使if(std::is\_constant\_evaluate())在这里也没有太大的作用。

带有consteval标记的函数也不允许调用纯运行时函数(既不带有constexpr,也不带有consteval标记的函数),但只有在真正执行调用时才会检查这一点。对于consteval函数,只要没有到达运行时函数,包含调用运行时函数的语句就不是错误(这条规则已经适用于编译时调用的constexpr函数)。

考虑下面的例子:

\begin{cpp}
void compileTimeError()
{
}

consteval int nextTwoDigitValue(int val)
{
	if (val < 0 || val >= 99) {
		compileTimeError(); // call something not valid to call at compile time
	}
	return ++val;
}
\end{cpp}

这个编译时函数有一个有趣的效果,只能用于一些值在0到98之间的参数:

\begin{cpp}
constexpr int i1 = nextTwoDigitValue(0); // OK (initializes i1 with 1)
constexpr int i2 = nextTwoDigitValue(98); // OK (initializes i2 with 99)
constexpr int i3 = nextTwoDigitValue(99); // compile-time ERROR
constexpr int i4 = nextTwoDigitValue(-1); // compile-time ERROR
\end{cpp}

使用static\_assert()在这里不起作用,因为只能为编译时已知的值调用,并且consteval不会使val成为函数内部的编译时值:

\begin{cpp}
consteval int nextTwoDigitValue(int val)
{
	static_assert(val >= 0 && val < 99); // always ERROR: val is not a compile-time value
	...
}
\end{cpp}

通过使用这个技巧,可以将编译时函数约束为某些值,就可以在编译时对解析的字符串发出无效格式的信号。

\mySubsubsection{18.2.4}{编译时值与编译时上下文}

可以假设编译时函数中的每个值都是编译时值,但事实并非如此。在编译时函数中,代码的静态类型和值的动态计算之间也存在差异。

考虑下面的例子:

\begin{cpp}
consteval void process()
{
	constexpr std::array a1{0, 8, 15};
	constexpr auto n1 = std::ranges::count(a1, 0); // OK
	std::array<int, n1> a1b; // OK

	std::array a2{0, 8, 15};
	constexpr auto n2 = std::ranges::count(a2, 0); // ERROR

	std::array a3{0, 8, 15};
	auto n3 = std::ranges::count(a3, 0); // OK
	std::array<int, n3> a3b; // ERROR
}
\end{cpp}

虽然这是一个只能在编译时使用的函数,但a2不是编译时值,所以不能用于通常需要编译时值的地方,例如:初始化constexpr变量n2。

出于同样的原因,只能使用n1来声明std::array的大小,使用非constexpr的n3会失败(即使将n3声明为const)。

若process()声明为constexpr函数,同样适用。但是否在编译时上下文中调用,则并不重要。





