

对于算法,C++20提供了一些扩展(其中一些已经在本书的其他章节中描述过了)。

\mySubsubsection{23.7.1}{支持范围}

对于许多算法,现在都支持:

\begin{itemize}
\item
用于将整个范围(容器,视图)作为单个参数传递

\item
用于将投影参数作为单个参数传递
\end{itemize}

这种支持要求在命名空间std::ranges中调用算法。

对于以下算法,目前还不支持范围:

\begin{itemize}
\item
数值算法(如accumulate())

\item
并行执行算法

\item
算法lexicographical\_compare\_three\_way()
\end{itemize}

有关所有标准算法的范围支持的更多详细信息,请参阅算法概述。

\mySubsubsection{23.7.2}{新算法}

“新标准算法"表列出了C++20中引入的标准算法。

% Please add the following required packages to your document preamble:
% \usepackage{longtable}
% Note: It may be necessary to compile the document several times to get a multi-page table to line up properly
\begin{longtable}[c]{|l|l|}
\hline
\textbf{名称}  & \textbf{效果}                            \\ \hline
\endfirsthead
%
\endhead
%
min()          & 生成传递的范围的最小值 \\ \hline
max()          & 生成传递的范围的最大值 \\ \hline
minmax()       & 生成传递范围的最小值和最大值       \\ \hline
shift\_left()  & 将所有元素移到前面            \\ \hline
shift\_right() & 将所有元素移到后面             \\ \hline
lexicographical\_compare\_three\_way() & 使用运算符对两个范围排序 \textless{}=\textgreater{} \\ \hline
\end{longtable}

\begin{center}
表23.5 新标准算法
\end{center}

\mySamllsection{支持范围的min(), max()和minmax()}

范围库中引入了std::ranges::min()、std::ranges::max()和std::ranges::minmax()算法,以产生传递的范围的最小值和/或最大值。可以选择传递比较条件和投影。

例如:

\filename{lib/minmax.cpp}

\begin{cpp}
#include <iostream>
#include <vector>
#include <algorithm>

int main()
{
	std::vector coll{0, 8, 15, 47, 11};

	std::cout << std::ranges::min(coll) << '\n';
	std::cout << std::ranges::max(coll) << '\n';
	auto [min, max] = std::ranges::minmax(coll);
	std::cout << min << ' ' << max << '\n';
}
\end{cpp}

该程序有以下输出:

\begin{shell}
0
47
0 47
\end{shell}

std中没有接受两个迭代器的相应算法,只有函数接受两个值或一个std::initializer\_list<>(和一个比较条件)作为参数。与往常一样,范围库还为其提供了传递投影的选项.

\mySamllsection{shift\_left() 和shift\_right()}

使用新的标准算法shift\_left()和shift\_right,可以分别将元素移动到前面或后面。这些算法分别返回新的结束或开始。

例如:

\filename{lib/shift.cpp}

\begin{cpp}
#include <iostream>
#include <vector>
#include <algorithm>
#include <ranges>

void print (const auto& coll)
{
	for (const auto& elem : coll) {
		std::cout << elem << ' ';
	}
	std::cout << '\n';
}

int main()
{
	std::vector coll{1, 2, 3, 4, 5, 6, 7, 8};

	print(coll);

	// shift one element to the front (returns new end):
	std::shift_left(coll.begin(), coll.end(), 1);
	print(coll);

	// shift three elements to the back (returns new begin):
	auto newbeg = std::shift_right(coll.begin(), coll.end(), 3);
	print(coll);
	print(std::ranges::subrange{newbeg, coll.end()});
}
\end{cpp}

该程序有以下输出:

\begin{shell}
1 2 3 4 5 6 7 8
2 3 4 5 6 7 8 8
2 3 4 2 3 4 5 6
2 3 4 5 6
\end{shell}

可以通过执行策略来允许使用多个线程,但支持传递单参数范围和/或项目的范围(可能是疏忽)。

\mySamllsection{lexicographical\_compare\_three\_way()}

为了以返回一种新的比较类别类型的方式比较两个不同容器中的元素,C++20引入了std::lexicographical\_compare\_three\_way()算法,在<=>操作符的章节中进行了描述。

对于这个算法,还不支持范围。既不能将范围作为单个参数传递,也不能传递投影参数(这可能是一个疏忽)。

\mySubsubsection{23.7.2}{unseq算法执行策略}

C++17为新引入的并行算法引入了各种执行策略,可以启用并行计算并允许线程并行地操作多个数据项(称为向量化或SIMD处理)。

但不能允许算法在多个数据项上操作,而是限制算法仅使用一个线程,因此C++20现在提供了执行策略std::execution::unseq。与往常一样,新执行策略是命名空间std::execution中对应的惟一类unsequenced\_policy的constexpr对象。“执行策略”表列出了现在支持的所有标准化执行策略。

% Please add the following required packages to your document preamble:
% \usepackage{longtable}
% Note: It may be necessary to compile the document several times to get a multi-page table to line up properly
\begin{longtable}[c]{|l|l|}
\hline
\textbf{策略}     & \textbf{意义}                                           \\ \hline
\endfirsthead
%
\endhead
%
std::execution::seq & 用一个线程顺序计算单个值     \\ \hline
std::execution::par & 用多个线程并行计算单个值 \\ \hline
std::execution::unseq      & 用一个线程并行计算多个值(C++20起) \\ \hline
std::execution::par\_unseq & 用多个线程并行计算多个值        \\ \hline
\end{longtable}

\begin{center}
表23.6 执行策略
\end{center}

非顺序执行策略允许在单个执行线程中交错执行传递给访问元素的操作,不应该将此策略用于阻塞同步(例如使用互斥锁),这可能会导致死锁。

下面是一个使用的例子:

\filename{lib/unseq.cpp}

\begin{cpp}
#include <iostream>
#include <vector>
#include <cmath>
#include <algorithm>
#include <execution>

int main (int argc, char** argv)
{
	// init number of argument from command line (default: 1000):
	int numElems = 1000;
	if (argc > 1) {
		numElems = std::atoi(argv[1]);
	}

	// init vector of different double values:
	std::vector<double> coll;
	coll.reserve(numElems);
	for (int i=0; i<numElems; ++i) {
		coll.push_back(i * 4.37);
	}

	// process square roots:
	// - allow SIMD processing but only one thread
	std::for_each(std::execution::unseq, // since C++20
				  coll.begin(), coll.end(),
				  [](auto& val) {
					val = std::sqrt(val);
				  });

	for (double value : coll) {
		std::cout << value << '\n';
	}
}
\end{cpp}

与通常的执行策略一样,无法影响何时以及如何使用策略。使用此策略,可以启用向量化或SIMD计算,但不能强制这样做。在不支持此策略的硬件上,或若实现在运行时决定不使用(例如,因为CPU负载太高),则unseq策略将会串行执行。




















