
C++20为处理位提供了更好、更清晰的支持:

\begin{itemize}
\item
缺少底层位操作

\item
位的强制转换

\item
检查平台的端序
\end{itemize}

所有这些工具都在头文件<bit>中定义。

\mySubsubsection{23.5.1}{位操作}

硬件通常对位操作有特殊的支持,比如“左移”或“右移”。C++20之前,C++开发者不能直接访问这些指令,新引入的位操作为底层CPU的位指令提供了一个直接的API。

“位操作”表列出了C++20引入的所有标准化的位操作,其在头文件<bit>中作为命名空间std中的独立函数提供。

% Please add the following required packages to your document preamble:
% \usepackage{longtable}
% Note: It may be necessary to compile the document several times to get a multi-page table to line up properly
\begin{longtable}[c]{|l|l|}
\hline
\textbf{操作}    & \textbf{意义}                                    \\ \hline
\endfirsthead
%
\endhead
%
rotl(val, n)          & 生成向左移n位的val          \\ \hline
rotr(val, n)          & 生成向右移n位的val         \\ \hline
countl\_zero(val)     & 生成前置位中(从最高有效位起)0的个数   \\ \hline
countl\_one(val)      & 产生前置位中(从最高有效位起)1的个数   \\ \hline
countr\_zero(val)     & 产生后置位中(从最低有效位起)0的个数 \\ \hline
countl\_one(val)      & 产生后置位中(从最低有效位起)1的个数 \\ \hline
popcount(val)         & 输出值中1位的个数                \\ \hline
has\_single\_bit(val) & 返回val是否为2的幂(二进制集合)    \\ \hline
bit\_floor(val)       & 得到之前的2次幂值                  \\ \hline
bit\_ceil(val)        & 得到下一个2次幂值                      \\ \hline
bit\_width(val)       & 生成存储该值所需的位数  \\ \hline
\end{longtable}

\begin{center}
表23.3 位操作
\end{center}

考虑下面的程序:

\filename{lib/bitops8.cpp}

\begin{cpp}
#include <iostream>
#include <format>
#include <bitset>
#include <bit>

int main()
{
	std::uint8_t i8 = 0b0000'1101;
	std::cout
		<< std::format("{0:08b} {0:3}\n", i8) // 00001101
		<< std::format("{0:08b} {0:3}\n", std::rotl(i8, 2)) // 00110100
		<< std::format("{0:08b} {0:3}\n", std::rotr(i8, 1)) // 10000110
		<< std::format("{0:08b} {0:3}\n", std::rotr(i8, -1)) // 00011010
		<< std::format("{}\n", std::countl_zero(i8)) // four leading zeros
		<< std::format("{}\n", std::countr_one(i8)) // one trailing one
		<< std::format("{}\n", std::popcount(i8)) // three ones
		<< std::format("{}\n", std::has_single_bit(i8)) // false
		<< std::format("{0:08b} {0:3}\n", std::bit_floor(i8)) // 00001000
		<< std::format("{0:08b} {0:3}\n", std::bit_ceil(i8)) // 00010000
		<< std::format("{}\n", std::bit_width(i8)); // 4
}
\end{cpp}

该程序有以下输出:

\begin{shell}
00001101  13
00110100  52
10000110 134
00011010  26
4
1
3
false
00001000  8
00010000 16
4
\end{shell}

注意事项如下:

\begin{itemize}
\item
只有当传递的类型是无符号整型时,才提供所有这些函数。

\item
位移函数也取负n,则改变了方向。

\item
所有返回计数的函数的返回类型都是int。唯一的例外是bit\_width(),会返回传递类型的值,这是标准中的不一致(我认为是错误)。当将其作为int类型使用或直接打印时,可能必须使用静态强制转换。
\end{itemize}

若为std::uint16\_t运行相应的程序(参见lib/bitops16.cpp),会得到以下输出:

\begin{shell}
0000000000001101    13
0000000000110100    52
1000000000000110 32774
0000000000011010    26
12
1
3
false
0000000000001000     8
0000000000010000    16
4
\end{shell}

这些函数仅为无符号整型定义。所以:

\begin{itemize}
\item
不能对有符号整型使用位操作:

\begin{cpp}
int b1 = ... ;
auto b2 = std::rotl(b1, 2); // ERROR
\end{cpp}

\item
不能对char类型使用位操作:

\begin{cpp}
char b1 = ... ;
auto b2 = std::rotl(b1, 2); // ERROR
\end{cpp}

unsigned char类型可以。

\item
不能对std::byte类型使用位操作:

\begin{cpp}
std::byte b1{ ... };
auto b2 = std::rotl(b1, 2); // ERROR
\end{cpp}

\end{itemize}

“位操作的硬件支持”表,摘自\url{http://wg21.link/p0553r4},列出了一些新的位操作到现有硬件的可能映射。

% Please add the following required packages to your document preamble:
% \usepackage{longtable}
% Note: It may be necessary to compile the document several times to get a multi-page table to line up properly
\begin{longtable}[c]{|l|l|l|l|}
\hline
\textbf{操作} & \textbf{Intel/AMD} & \textbf{ARM} & \textbf{PowerPC} \\ \hline
\endfirsthead
%
\endhead
%
rotl()             & ROL                & -            & rldicl           \\ \hline
rotr()             & ROR                & ROR, EXTR    & -                \\ \hline
popcount()         & POPCNT             & -            & popcntb          \\ \hline
countl\_zero()     & BSR, LZCNT         & CLZ          & cntlzd           \\ \hline
countl\_one()      & -                  & CLS          & -                \\ \hline
countr\_zero()     & BSF, TZCNT         & -            & -                \\ \hline
countr\_ont()      & -                  & -            & -                \\ \hline
\end{longtable}

\begin{center}
表23.4 位操作的硬件支持
\end{center}

\mySubsubsection{23.5.2}{std::bit\_cast<>()}

C++20提供了一个新的强制转换操作,用于更改位序列的类型。与使用reinterpret\_cast<>或联合相比,操作符std::bit\_cast<>确保位数合适,使用标准布局,并且不使用指针类型。

例如:

\begin{cpp}
std::uint8_t b8 = 0b0000'1101;

auto bc = std::bit_cast<char>(b8); // OK
auto by = std::bit_cast<std::byte>(b8); // OK
auto bi = std::bit_cast<int>(b8); // ERROR: wrong number of bits
\end{cpp}

\mySubsubsection{23.5.2}{std::endian}

C++20引入了一个新的实用程序枚举类型std::endian,可用于检查执行环境的端序。其引入了三个枚举值:

\begin{itemize}
\item
std::endian::big,一个代表“大端”的值(标量类型存储时,最重要的字节放在第一位,其余字节按降序排列)。

\item
std::endian::little,一个代表“l小端”的值(标量类型将最低有效位字节放在首位,其余字节按升序存储)。

\item
std::endian::native,用于指定执行环境的端序
\end{itemize}

若所有标量类型都是大端字节数,std::endian::native等于std::endian::big。若所有标量类型都是little-endian, std::endian::native等于std::endian::little。std::endian::native的值既不是std::endian::big,也不是std::endian::little。

若所有标量类型的大小都为1,则std::endian::little、std::endian::big和std::endian::native都具有相同的值。

该类型在头文件<bit>中定义。

作为枚举值,这些值可以在编译时使用。例如:

\begin{cpp}
#include <bit>
...

if constexpr (std::endian::native == std::endian::big) {
	... // handle big-endian
}
else if constexpr (std::endian::native == std::endian::little) {
	... // handle little-endian
}
else {
	... // handle mixed endian
}
\end{cpp}









