
本节讨论C++20中创建引用现有外部值的视图的所有特性(通常作为单个范围参数传递,作为开始迭代器和哨兵,或作为开始迭代器和计数)。

\mySubsubsection{8.3.1}{子范围}

\begin{longtable}[c]{|l|l|}
\hline
\textbf{类型:}                 & std::ranges::subrange\textless{}\textgreater{} \\ \hline
\endfirsthead
%
\endhead
%
\textbf{内容:}              & 从传递开始到结束的所有元素   \\ \hline
\textbf{工厂:}      & \begin{tabular}[c]{@{}l@{}}std::views::counted()\\ std::views::reverse()在逆子范围上\end{tabular} \\ \hline
\textbf{元素类型:}         & 传递迭代器中值的类型               \\ \hline
\textbf{要求:}             & 至少是输入迭代器                       \\ \hline
\textbf{类别:}             & 和传入的一样                                 \\ \hline
\textbf{是否为有限范围:} & 若传递的是普通随机访问迭代器或是长度提示                                                \\ \hline
\textbf{是否为常规范围:}      & 若在通用迭代器上                         \\ \hline
\textbf{是否为租借范围:}    & 总是                                         \\ \hline
\textbf{缓存:}               & 无                                        \\ \hline
\textbf{常量可迭代:}       & 若传递的迭代器可复制               \\ \hline
\textbf{传播常量性:} & 从不                                          \\ \hline
\end{longtable}

类模板std::ranges::subrange<>定义了一个范围元素的视图,通常以开始迭代器和哨兵(结束迭代器)对的形式传递,也可以传递单个范围对象,并间接传递开始迭代器和计数。在内部,视图本身通过存储开始(迭代器)和结束(哨兵)来展示元素。

子范围视图的主要用例是将一对开始迭代器和哨兵(结束迭代器)转换为一个对象。例如:

\begin{cpp}
std::vector<int> coll{0, 8, 15, 47, 11, -1, 13};

std::ranges::subrange s1{std::ranges::find(coll, 15),
						std::ranges::find(coll, -1)};
print(coll); // 15 47 11
\end{cpp}

你可以用结束值的特殊哨兵来初始化子范围:

\begin{cpp}
std::ranges::subrange s2{coll.begin() + 1, EndValue<-1>{}};
print(s2); // 8 15 47 11
\end{cpp}

这两种方法在将一对迭代器转换为范围/视图时都特别有用,例如,这样范围适配器就可以处理元素:

\begin{cpp}
void foo(auto beg, auto end)
{
	// init view for all but the first five elements (if there are any):
	auto v = std::ranges::subrange{beg, end} | std::views::drop(5);
	...
}
\end{cpp}

下面是演示子范围用法的完整示例:

\filename{ranges/subrange.cpp}

\begin{cpp}
#include <iostream>
#include <string>
#include <unordered_map>
#include <ranges>

void printPairs(auto&& rg)
{
	for (const auto& [key, val] : rg) {
		std::cout << key << ':' << val << ' ';
	}
	std::cout << '\n';
}

int main()
{
	// English/German dictionary:
	std::unordered_multimap<std::string,std::string> dict = {
		{"strange", "fremd"},
		{"smart", "klug"},
		{"car", "Auto"},
		{"smart","raffiniert"},
		{"trait", "Merkmal"},
		{"smart", "elegant"},
	};
	
	// get begin and end of all elements with translation of "smart":
	auto [beg, end] = dict.equal_range("smart");
	
	// create subrange view to print all translations:
	printPairs(std::ranges::subrange(beg, end));
}
\end{cpp}

该程序有以下输出:

\begin{shell}
smart:klug smart:elegant smart:raffiniert
\end{shell}

equal\_range()返回了键为"smart"的dict中所有元素的开头和结尾,这里使用一个子范围将这两个迭代器转换为一个视图(这里是字符串对元素的视图):

\begin{cpp}
std::ranges::subrange(beg, end)
\end{cpp}

然后,可以将视图传递给printPairs(),遍历元素并打印出来。

注意,子范围可以改变它的大小,以便在begin和end都有效的情况下在begin和end之间插入或删除元素:

\begin{cpp}
std::list coll{1, 2, 3, 4, 5, 6, 7, 8};

auto v1 = std::ranges::subrange(coll.begin(), coll.end());
print(v1); // 1 2 3 4 5 6 7 8

coll.insert(++coll.begin(), 0);
coll.push_back(9);
print(v2); // 1 0 2 3 4 5 6 7 8 9
\end{cpp}

\mySamllsection{子范围的工厂}

没有范围工厂用于从begin(迭代器)和end(哨兵)初始化子范围。然而,有一个范围工厂可以从begin和count创建子范围:[C++20最初声明all()也可以产生子范围。当后来引入了所属视图时,这个选项被删除了(参见\url{http://wg21.link/p2415})]

\begin{cpp}
std::views::counted(beg, sz)
\end{cpp}

std::views::counted()创建一个包含非连续范围的前sz个元素的子范围,从迭代器beg引用的元素开始(对于连续范围,counts()创建一个span视图)。

当std::views::counted()创建子范围时,子范围以std::counted\_iterator作为开始,以std::default\_sentinel\_t类型的伪哨兵作为结束:

\begin{cpp}
std::views::counted(rg.begin(), 5);
\end{cpp}

相当于:

\begin{cpp}
std::ranges::subrange{std::counted_iterator{rg.begin(), 5},
	std::default_sentinel};
\end{cpp}

这样做的效果是,即使插入或删除元素,该子范围内的计数仍然保持稳定:

\begin{cpp}
std::list coll{1, 2, 3, 4, 5, 6, 7, 8};

auto v2 = std::views::counted(coll.begin(), coll.size());
print(v2); // 1 2 3 4 5 6 7 8

coll.insert(++coll.begin(), 0);
coll.push_back(9);
print(v2); // 1 0 2 3 4 5 6 7
\end{cpp}

有关详细信息,请参阅std::views::counted()和std::counted\_iterator类型的描述。

下面是一个完整的示例,演示了counted()如何使用子范围:

\filename{ranges/subrangecounted.cpp}

\begin{cpp}
#include <iostream>
#include <string>
#include <unordered_map>
#include <ranges>

void printPairs(auto&& rg)
{
	for (const auto& [key, val] : rg) {
		std::cout << key << ':' << val << ' ';
	}
	std::cout << '\n';
}

int main()
{
	// English/German dictionary:
	std::unordered_multimap<std::string,std::string> dict = {
		{"strange", "fremd"},
		{"smart", "klug"},
		{"car", "Auto"},
		{"smart","raffiniert"},
		{"trait", "Merkmal"},
		{"smart", "elegant"},
	};
	
	printPairs(dict);
	printPairs(std::views::counted(dict.begin(), 3));
}
\end{cpp}

该程序有以下输出:

\begin{shell}
trait:Merkmal car:Auto smart:klug smart:elegant smart:raffiniert strange:fremd
trait:Merkmal car:Auto smart:klug
\end{shell}

注意,在极少数情况下,std::views::reverse()也可能产生子范围。当反转一个反转的子范围时,就会发生这种情况,将会得到原始的子范围。


\mySamllsection{子范围的特殊特征}

子范围可以是也可以不是一个长度范围。实现的方式是,子范围有第三个模板形参,枚举类型std::ranges::subrange\_kind,它的值可以是std::ranges::unsized或std::ranges::sized。该模板形参的值可以指定,也可以从为迭代器和哨兵类型调用的std::sized\_sentinel\_for概念派生。

会有以下后果:

\begin{itemize}
\item
若传递相同类型的连续迭代器或随机访问迭代器,则子范围的大小为:

\begin{cpp}
std::vector vec{1, 2, 3, 4, 5}; // has random access
...
std::ranges::subrange sv{vec.begin()+1, vec.end()-1}; // sized
std::cout << std::ranges::sized_range<decltype(sv)>; // true
\end{cpp}

\item
若传递的迭代器不支持随机访问或不常见,则子范围不确定大小:

\begin{cpp}
std::list lst{1, 2, 3, 4, 5}; // no random access
...
std::ranges::subrange sl{++lst.begin(), --lst.end()}; // unsized
std::cout << std::ranges::sized_range<decltype(sl)>; // false
\end{cpp}

\item
后一种情况下,可以传递一个长度来使子范围变成一个长度范围:

\begin{cpp}
std::list lst{1, 2, 3, 4, 5}; // no random access
...
std::ranges::subrange sl2{++lst.begin(), --lst.end(), lst.size()-2}; // sized
std::cout << std::ranges::sized_range<decltype(sl2)>; // true
\end{cpp}

若用错误的大小初始化视图,或者在开始和结束之间插入或删除元素之后使用视图,则为未定义行为。
\end{itemize}

\mySamllsection{子范围的接口}

“类std::ranges::subrange<>的操作表”列出了子范围的API。

只有当迭代器是默认可初始化时,才提供默认构造函数。

使用szHint参数的构造函数,允许开发者将未设置长度的范围转换为设置长度的子范围,如上所示。

子范围的迭代器指向底层(或临时创建的)范围,所以子范围是租借范围。但请注意,当底层范围不再存在时,迭代器仍然为悬空。

\mySamllsection{类似元组的子范围接口}

std::ranges::subange也有一个类似于元组的接口来支持结构化绑定(C++17中引入),所以可以很容易地初始化一个开始迭代器和一个哨兵(结束迭代器),方法如下:

\begin{cpp}
auto [beg, end] = std::ranges::subrange{coll};
\end{cpp}

为此,提供类std::ranges::subange;

\begin{itemize}
\item
std::tuple\_size<>的特化

\item
std::tuple\_element<>的特化

\item
获取索引0和1的get<>()函数
\end{itemize}

\begin{longtable}[c]{|l|l|}
\hline
\textbf{操作}      & \textbf{效果}                                                           \\ \hline
\endfirsthead
%
\endhead
%
subrange r\{\}          & 创建一个空子范围                                                 \\ \hline
subrange r\{rg\}        & 使用范围rg的元素创建子范围                          \\ \hline
subrange r\{rg, szHint\}       & 使用范围rg的元素创建子范围,指定该范围有szHint个元素           \\ \hline
subrange r\{beg, end\}  & 使用range {[}beg, end)的元素创建子范围                \\ \hline
subrange r\{beg, end, szHint\} & 使用range {[}beg, end)的元素创建子范围,指定该范围有szHint个元素 \\ \hline
r.begin()               & 生成begin迭代器                                                 \\ \hline
r.end()                 & 生成哨兵(end迭代器)                                        \\ \hline
r.empty()               & 生成r是否为空                                                 \\ \hline
if (r)                  & 若r不为空,则为True                                                    \\ \hline
r.size()                & 生成元素的数量(若有长度则可用)                         \\ \hline
r.front()               & 生成第一个元素(若转发可用)                         \\ \hline
r.back()                & 生成最后一个元素(若为双向且通用则可用)           \\ \hline
r{[}idx{]}              & 生成第n个元素(如果随机访问可用)                      \\ \hline
r.data()                & 生成一个指向元素内存的原始指针(若元素在连续内存中可用)。       \\ \hline
r.next()                & 生成从第二个元素开始的子范围                        \\ \hline
r.next(n)               & 生成从第n个元素开始的子范围                          \\ \hline
r.prev()                & 生成从第一个元素之前的元素开始的子范围      \\ \hline
r.prev(n)               & 生成一个子范围,从第n个元素开始,在第一个元素之前 \\ \hline
r.advance(dist)         & 修改r,使其稍后开始num元素(以负num开始较早)                       \\ \hline
auto {[}beg, end{]} = r & 用begin和end/哨兵(r)初始化beg和end                  \\ \hline
\end{longtable}

\begin{center}
表8.4 类std::ranges::subrange<>的操作
\end{center}

\mySamllsection{使用子范围协调原始数组类型}

子范围的类型仅取决于迭代器的类型(以及是否提供了size()),这可以用来协调原始数组的类型。

考虑下面的例子:

\begin{cpp}
int a1[5] = { ... };
int a2[10] = { ... };

std::same_as<decltype(a1), decltype(a2)> // false

std::same_as<decltype(std::ranges::subrange{a1}),
			 decltype(std::ranges::subrange{a2})> // true
\end{cpp}

注意,这只适用于原始数组。对于std::array<>和其他容器,迭代器(可能)具有不同的类型,所以协调的类型与子范围是不可移植的。

\mySubsubsection{8.3.2}{参考视图}

\begin{longtable}[c]{|l|l|}
\hline
\textbf{类型:}                 & std::ranges::ref\_view\textless{}\textgreater{} \\ \hline
\endfirsthead
%
\endhead
%
\textbf{内容:}              & 一个范围的所有元素   \\ \hline
\textbf{工厂:}      & \begin{tabular}[c]{@{}l@{}}std::views::all() and all other adaptors on lvalues\end{tabular} \\ \hline
\textbf{元素类型:}         & 与传入的范围类型相同                 \\ \hline
\textbf{要求:}             & 至少为输入范围                     \\ \hline
\textbf{类别:}             & 与传入相同                                 \\ \hline
\textbf{是否是长度范围:} & 若是长度范围                                                \\ \hline
\textbf{是否是通用范围:}      & 若是常规范围                         \\ \hline
\textbf{是否是租借范围:}    & 总是                                         \\ \hline
\textbf{缓存:}               & 无                                        \\ \hline
\textbf{常量可迭代:}       & 若在可迭代范围内               \\ \hline
\textbf{传播常量性:} & 从不                                          \\ \hline
\end{longtable}

类模板std::ranges::ref\_view<>定义了一个简单引用范围的视图,通过值传递视图的效果就像通过引用传递范围一样。

其效果类似于std::reference\_wrapper<>类型(C++11引入),对使用std::ref()和std::cref()创建的第一类对象进行引用,但这个包装器的好处仍然可以直接用作范围。

ref\_view的用例是将容器转换为复制成本较低的轻量级对象。例如:

\begin{cpp}
void foo(std::ranges::input_range auto coll) // NOTE: takes range by value
{
	for (const auto& elem : coll) {
		...
	}
}

std::vector<std::string> coll{ ... };

foo(coll); // copies coll
foo(std::ranges::ref_view{coll}); // pass coll by reference
\end{cpp}

将容器传递给协程(通常必须按值接受参数)可能是这种技术的一种应用。

请注意,只能创建一个左值(一个有名字的范围)的参考视图:

\begin{cpp}
std::vector coll{0, 8, 15};
...
std::ranges::ref_view v1{coll}; // OK, refers to coll
std::ranges::ref_view v2{std::move(coll)}; // ERROR
std::ranges::ref_view v3{std::vector{0, 8, 15}}; // ERROR
\end{cpp}

对于右值,必须使用归属视图。

下面是一个使用参考视图的完整示例:

\filename{ranges/refview.cpp}

\begin{cpp}
#include <iostream>
#include <string>
#include <vector>
#include <ranges>

void printByVal(std::ranges::input_range auto coll) // NOTE: takes range by value
{
	for (const auto& elem : coll) {
		std::cout << elem << ' ';
	}
	std::cout << '\n';
}

int main()
{
	std::vector<std::string> coll{"love", "of", "my", "life"};
	
	printByVal(coll); // copies coll
	printByVal(std::ranges::ref_view{coll}); // pass coll by reference
	printByVal(std::views::all(coll)); // ditto (using a range adaptor)
}
\end{cpp}

\mySamllsection{参考视图的范围适配器}

参考视图也可以(通常应该)用range工厂创建:

\begin{cpp}
std::views::all(rg)
\end{cpp}

std::views::all()创建一个ref\_view到传入的范围rg,前提是rg是左值,而非视图(若rg已经是视图,则返回rg;若传递右值,则返回归属视图)。

此外,若传递的左值还不是视图,几乎所有其他范围适配器也会创建一个参考视图。

上面的示例程序使用这种间接的方式,通过最后一次调用printByVal()创建了一个参考视图。它传递一个std::ranges::ref\_view<std::vector<std::string>{}>{}>:

\begin{cpp}
std::vector<std::string> coll{"love", "of", "my", "life"};
...
printByVal(std::views::all(coll));
\end{cpp}

请注意,所有其他接受范围的视图都间接地为传递的范围调用all()(通过使用std::views::all\_t<>)。出于这个原因,若传递一个不是视图的左值给其中一个视图,一个参考视图总是自动创建的。例如,调用:

\begin{cpp}
std::views::take(coll, 3)
\end{cpp}

基本上与调用以下代码的效果相同:

\begin{cpp}
std::ranges::take_view{std::ranges::ref_view{coll}, 3};
\end{cpp}

但使用适配器时,可能会有一些优化。

详细信息请参见std::views::all()的描述。

\mySamllsection{参考视图的特点}

参考视图存储对基础范围的引用,所以参考视图是一个租借范围。只要引用的视图有效,就可以使用它。例如,若底层视图是一个vector,重新分配并不会使该视图失效。但请注意,当底层范围不再存在时,迭代器仍然可以悬空。

\mySamllsection{参考视图的接口}

“类std::ranges::ref\_view<>的操作表”列出了ref\_view的API。

% Please add the following required packages to your document preamble:
% \usepackage{longtable}
% Note: It may be necessary to compile the document several times to get a multi-page table to line up properly
\begin{longtable}[c]{|l|l|}
\hline
\textbf{操作} & \textbf{效果}                                                \\ \hline
\endfirsthead
%
\endhead
%
ref\_view r\{rg\}  & 创建一个引用范围rg的ref\_view                 \\ \hline
r.begin()          & 生成begin迭代器                                  \\ \hline
r.end()            & 生成哨兵(end迭代器)                             \\ \hline
r.empty()          & 生成r是否为空(若范围支持则可用) \\ \hline
if(r)              & 若r不为空则为true(若定义了empty()则可用)        \\ \hline
r.size() & 生成元素的数量(若引用了一个长度范围,则可用)                            \\ \hline
r.front()          & 生成第一个元素(若转发可用)              \\ \hline
r.back()           & 生成最后一个元素(若双向且通用则可用) \\ \hline
r{[}idx{]}         & 生成第n个元素(若随机访问可用)            \\ \hline
r.data() & 生成一个指向元素内存的原始指针(若元素在连续内存中可用) \\ \hline
r.base()           & 生成对r所引用的范围的引用               \\ \hline
\end{longtable}

\begin{center}
表8.5 类std::ranges::ref\_view<>的操作表
\end{center}

注意,没有为这个视图提供默认构造函数。

\mySubsubsection{8.3.3}{归属视图}

% Please add the following required packages to your document preamble:
% \usepackage{longtable}
% Note: It may be necessary to compile the document several times to get a multi-page table to line up properly
\begin{longtable}[c]{|l|l|}
\hline
\textbf{类型:}                 & std::ranges::owning\_view\textless{}\textgreater{}  \\ \hline
\endfirsthead
%
\endhead
%
\textbf{内容:}              & 移动范围内的所有元素                       \\ \hline
\textbf{适配器:}             & std::views::all()和所有其他对右值的适配器 \\ \hline
\textbf{元素类型:}         & 与传递的范围类型相同                           \\ \hline
\textbf{要求:}             & 至是少输入范围                                \\ \hline
\textbf{类别:}             & 与传入的类别相同                                      \\ \hline
\textbf{是否是长度范围:}       & 若为长度范围                                  \\ \hline
\textbf{是否是通用范围:}      & 若为常规范围                                   \\ \hline
\textbf{是否是租借范围:}    & 若为租借范围                                \\ \hline
\textbf{缓存:}               & 无                                             \\ \hline
\textbf{常量可迭代:}       & 若在可迭代范围内                          \\ \hline
\textbf{传播常量性:} & 总是                                              \\ \hline
\end{longtable}

类模板std::ranges::owning\_view<>定义了一个获取另一个范围元素所有权的视图。[最初,C++20没有拥有视图,该视图是在后来的修复中引入的(参见\url{http://wg21.link/p2415})。]

这是(到目前为止)唯一可能拥有多个元素的视图,但构造成本仍然很低,因为初始范围必须是右值(临时对象或用std::move()标记的对象)。然后,构造函数将范围移动到视图的内部成员。

例如:

\begin{cpp}
std::vector vec{0, 8, 15};
std::ranges::owning_view v0{vec}; // ERROR
std::ranges::owning_view v1{std::move(vec)}; // OK
print(v1); // 0 8 15
print(vec); // unspecified value (was moved away)

std::array<std::string, 3> arr{"tic", "tac", "toe"};
std::ranges::owning_view v2{arr}; // ERROR
std::ranges::owning_view v2{std::move(arr)}; // OK
print(v2); // "tic" "tac" "toe"
print(arr); // "" "" ""
\end{cpp}

这是C++20中唯一一个完全不支持复制的标准视图,只能移动它。例如:

\begin{cpp}
std::vector coll{0, 8, 15};
std::ranges::owning_view v0{std::move(coll)};

auto v3 = v0; // ERROR
auto v4 = std::move(v0); // OK (range in v0 moved to v4)
\end{cpp}

owning\_view的主要用例是从一个范围创建视图,而不再依赖于范围的生命周期。例如:

\begin{cpp}
void foo(std::ranges::view auto coll) // NOTE: takes range by value
{
	for (const auto& elem : coll) {
		...
	}
}

std::vector<std::string> coll{ ... };

foo(coll); // ERROR: no view
foo(std::move(coll)); // ERROR: no view
foo(std::ranges::owning_view{coll}); // ERROR: must pass rvalue
foo(std::ranges::owning_view{std::move(coll)}); // OK: move coll as view
\end{cpp}

将范围转换为归属视图,通常是通过将临时容器传递给范围适配器来隐式地完成的(见下文)。

\mySamllsection{所属视图的范围适配器}

归属视图也可以(通常应该)用范围工厂创建:

\begin{cpp}
std::views::all(rg)
\end{cpp}

std::views::all()为传入的范围rg创建一个owning\_view,前提是rg是一个右值而不是一个视图(若已经是一个视图则返回rg,否则传递左值则返回ref视图)。

此外,若传递的右值还不是视图,几乎所有其他范围适配器也会创建一个所属视图。

因此,上面的例子可以像下面这样对foo()的调用:

\begin{cpp}
foo(std::views::all(std::move(coll))); // move coll as view
\end{cpp}

要创建一个归属视图,适配器all()需要一个右值(对于一个左值,all()创建一个参考视图),所以必须传递一个临时范围或一个用std::move()标记的命名范围(若传递左值,all()将产生一个参考视图)。

下面是一个完整的例子:

\filename{ranges/owningview.cpp}

\begin{cpp}
#include <iostream>
#include <string>
#include <vector>
#include <ranges>

auto getColl()
{
	return std::vector<std::string>{"you", "don't", "fool", "me"};
}

int main()
{
	std::ranges::owning_view v1 = getColl(); // view owning a vector of strings
	auto v2 = std::views::all(getColl()); // ditto
	static_assert(std::same_as<decltype(v1), decltype(v2)>);
	
	// iterate over drop view of owning view of vector<string>:
	for (const auto& elem : getColl() | std::views::drop(1)) {
		std::cout << elem << ' ';
	}
	std::cout << '\n';
}
\end{cpp}

这里,v1和v2都有类型std::ranges::owning\_view<std::vector<std::string>{}>。然而,创建拥有视图的常用方法是在程序末尾基于范围的for循环中进行演示:将一个临时容器传递给范围适配器时,会间接地创建了一个归属视图。

表达式

\begin{cpp}
getColl() | std::views::drop(1)
\end{cpp}

创建了

\begin{cpp}
std::ranges::drop_view<std::ranges::owning_view<std::vector<std::string>>>
\end{cpp}

这是通过使用std::views::all\_t<>对传递的临时范围间接调用all()实现的。

详细信息请参见std::views::all()的描述。


\mySamllsection{所属视图的特殊特性}

拥有视图将传递的范围移动到自身,若拥有的范围是租借范围,则拥有视图是租借范围。

\mySamllsection{所属视图的接口}

类std::ranges::owning\_view<>的表操作列出了owning\_view的API。

只有当迭代器是默认可初始化时,才提供默认构造函数。

% Please add the following required packages to your document preamble:
% \usepackage{longtable}
% Note: It may be necessary to compile the document several times to get a multi-page table to line up properly
\begin{longtable}[c]{|l|l|}
\hline
\textbf{操作}   & \textbf{效果}                                                \\ \hline
\endfirsthead
%
\endhead
%
owning\_view r\{\}   & 创建一个空的owning\_view                          \\ \hline
owning\_view r\{rg\} & 创建一个拥有rg元素的owning\_view           \\ \hline
r.begin()            & 生成begin迭代器                                      \\ \hline
r.end()              & 生成哨兵(end迭代器)                            \\ \hline
r.empty()            & 生成r是否为空(若范围支持则可用)  \\ \hline
if(r)                & 若r不为空则为True(若定义了empty()则可用)       \\ \hline
r.size() & 生成元素的数量(若引用了一个长度范围,则可用)                             \\ \hline
r.front()            & 生成第一个元素(若转发可用)              \\ \hline
r.back()             & 生成最后一个元素(若是双向且通用的,则可用) \\ \hline
r{[}idx{]}           & 生成第n个元素(若随机访问可用)            \\ \hline
r.data() & 生成一个指向元素内存的原始指针(若元素在连续内存中可用)。 \\ \hline
r.base()             & 生成对生成区域的引用                  \\ \hline
\end{longtable}

\begin{center}
表8.6 类std::ranges::owning\_view<>的操作
\end{center}

\mySubsubsection{8.3.4}{通用视图}

% Please add the following required packages to your document preamble:
% \usepackage{longtable}
% Note: It may be necessary to compile the document several times to get a multi-page table to line up properly
\begin{longtable}[c]{|l|l|}
\hline
\textbf{类型:}     & std::ranges::common\_view\textless{}\textgreater{}                \\ \hline
\endfirsthead
%
\endhead
%
\textbf{内容:}  & 具有统一迭代器类型的范围的所有元素            \\ \hline
\textbf{适配器:}              & std::views::common()        \\ \hline
\textbf{元素类型:}         & 与传入的range类型相同 \\ \hline
\textbf{要求:} & 非通用至少前向范围(适配器接受通用范围) \\ \hline
\textbf{类别:} & 通常是向前的(若在一个连续的范围内连续)        \\ \hline
\textbf{是否是长度范围:}       & 若为长度范围           \\ \hline
\textbf{是否是通用范围:}      & 总是                      \\ \hline
\textbf{是否是租借范围:}   & 若为租借范围      \\ \hline
\textbf{缓存:}               & 无                     \\ \hline
\textbf{常量可迭代:}       & 若在可迭代范围内  \\ \hline
\textbf{传播常量性:} & 只有在右值范围内     \\ \hline
\end{longtable}

类模板std::ranges::common\_view<>是一个视图,其协调范围的开始迭代器和结束迭代器类型,以便能够将它们传递给需要相同迭代器类型的代码(例如容器的构造函数或传统算法)。

注意,视图的构造函数要求传递的范围不是通用范围(迭代器有不同的类型)。例如:

\begin{cpp}
std::list<int> lst{1, 2, 3, 4, 5, 6, 7, 8, 9};

auto v1 = std::ranges::common_view{lst}; // ERROR: is already common

auto take5 = std::ranges::take_view{lst, 5}; // yields no common view
auto v2 = std::ranges::common_view{take5}; // OK
\end{cpp}

相应的范围适配器还可以处理已经很常见的范围,所以最好使用范围适配器std::views::common()来创建通用视图。

\mySamllsection{通用视图的范围适配器}

通用视图也可以(通常应该)使用范围适配器创建:

\begin{cpp}
std::views::common(rg)
\end{cpp}

std::views::common()创建一个将rg引用为通用范围的视图。若rg已经是公共的,返回std::views::all(rg)。

因此,可以使上面的示例始终按照以下方式编译:

\begin{cpp}
std::list<int> lst{1, 2, 3, 4, 5, 6, 7, 8, 9};

auto v1 = std::views::common(lst); // OK

auto take5 = std::ranges::take_view{lst, 5}; // yields no common view
auto v2 = std::views::common(take5); // v2 is common view
std::vector<int> coll{v2.begin(), v2.end()}; // OK
\end{cpp}

请注意,通过使用适配器,从已经很常见的范围中创建通用视图并不是编译时错误。这是选择适配器,而非直接初始化视图的关键原因。

下面是使用通用视图的完整示例程序:

\filename{ranges/commonview.cpp}

\begin{cpp}
#include <iostream>
#include <string>
#include <list>
#include <vector>
#include <ranges>

void print(std::ranges::input_range auto&& coll)
{
	for (const auto& elem : coll) {
		std::cout << elem << ' ';
	}
	std::cout << '\n';
}

int main()
{
	std::list<std::string> coll{"You're", "my", "best", "friend"};
	
	auto tv = coll | std::views::take(3);
	static_assert(!std::ranges::common_range<decltype(tv)>);
	// could not initialize a container by passing begin and end of the view
	
	std::ranges::common_view vCommon1{tv};
	static_assert(std::ranges::common_range<decltype(vCommon1)>);
	
	auto tvCommon = coll | std::views::take(3) | std::views::common;
	static_assert(std::ranges::common_range<decltype(tvCommon)>);
	
	std::vector<std::string> coll2{tvCommon.begin(), tvCommon.end()}; // OK
	print(coll);
	print(coll2);
}
\end{cpp}

这里,首先创建一个视图,若不是一个随机访问范围,这是不常见的,所以不能用它来初始化容器的元素:

\begin{cpp}
std::vector<std::string> coll2{tv.begin(), tv.end()}; // ERROR: begin/end types differ
\end{cpp}

使用通用视图,初始化工作得很好。若不知道要协调的范围是否已经是公共的,则使用std::views::common()。

通用视图的主要用例是将begin和end迭代器,具有不同类型的范围或视图传递给需要begin和end具有相同类型的泛型代码。一个典型的例子是将,范围的开始和结束传递给容器构造函数或传统算法。例如:

\begin{cpp}
std::list<int> lst {1, 2, 3, 4, 5, 6, 7, 8, 9};

auto v1 = std::views::take(lst, 5); // Note: types of begin() and end() differ
std::vector<int> coll{v1.begin(), v1.end()}; // ERROR: containers require the same type

auto v2 = std::views::common(std::views::take(lst, 5)); // same type now
std::vector<int> coll{v2.begin(), v2.end()}; // OK
\end{cpp}

\mySamllsection{通用视图的接口}

“类std::ranges::common\_view<>的操作表”列出了通用视图的API。


% Please add the following required packages to your document preamble:
% \usepackage{longtable}
% Note: It may be necessary to compile the document several times to get a multi-page table to line up properly
\begin{longtable}[c]{|l|l|}
\hline
\textbf{操作}   & \textbf{效果}                                                \\ \hline
\endfirsthead
%
\endhead
%
common\_view r\{\} & 创建引用默认构造范围的common\_view                                   \\ \hline
common\_view r\{rg\} & 创建引用范围rg的common\_view                 \\ \hline
r.begin()            & 生成begin迭代器                                      \\ \hline
r.end()              & 生成哨兵(end迭代器)                              \\ \hline
r.empty()            & 生成r是否为空(若范围支持则可用) \\ \hline
if (r)               & 若r不为空则为true(若定义了empty()则可用)        \\ \hline
r.size()           & 生成元素的数量(若引用了一个长度范围,则可用)                            \\ \hline
r.front()            & 生成第一个元素(若转发可用)              \\ \hline
r.back()             & 生成最后一个元素(若双向且通用的,则可用)  \\ \hline
r{[}idx{]}           & 生成第n个元素(如果随机访问可用)            \\ \hline
r.data()           & 产生一个指向元素内存的原始指针(若元素在连续内存中可用)。 \\ \hline
r.base()             & 产生对r所引用范围的引用               \\ \hline
\end{longtable}

\begin{center}
表8.7 类std::ranges::common\_view<>的操作表
\end{center}

只有当迭代器是默认可初始化时,才提供默认构造函数。在内部,common\_view使用std::common\_iterator类型的迭代器。

