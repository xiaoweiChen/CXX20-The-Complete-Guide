

本节讨论过滤给定范围或视图元素的所有视图。


\mySubsubsection{8.5.1}{获取视图}

% Please add the following required packages to your document preamble:
% \usepackage{longtable}
% Note: It may be necessary to compile the document several times to get a multi-page table to line up properly
\begin{longtable}[c]{|l|l|}
\hline
\textbf{类型:}                 & std::ranges::take\_view\textless{}\textgreater{} \\ \hline
\endfirsthead
%
\endhead
%
\textbf{内容:}              & 范围的第一个(最多)num个元素        \\ \hline
\textbf{适配器:}              & std::views::take()                               \\ \hline
\textbf{元素类型:}         & 与传入的类型相同 range                        \\ \hline
\textbf{要求:}             & 至少为输入范围                             \\ \hline
\textbf{类别:}             & 和传入一样                                   \\ \hline
\textbf{是否是长度范围:}       & 若为长度范围                              \\ \hline
\textbf{是否是通用范围:}      &若为长度随机访问范围                  \\ \hline
\textbf{是否是租借范围:}    & 若是租借视图或左值非视图        \\ \hline
\textbf{缓存:}               & 无                                          \\ \hline
\textbf{常量可迭代:}       & 若在可迭代范围内                       \\ \hline
\textbf{传播常量性:} & 只有在右值范围内                          \\ \hline
\end{longtable}

类模板std::ranges::take\_view<>定义了一个引用,传入范围的前num个元素的视图。若传递的范围没有足够的元素,则视图引用所有元素。

例如:

\begin{cpp}
std::vector<int> rg{1, 2, 3, 4, 5, 6, 7, 8, 9, 10, 11, 12, 13};
...
for (const auto& elem : std::ranges::take_view{rg, 5}) {
	std::cout << elem << ' ';
}
\end{cpp}

循环会输出:

\begin{shell}
1 2 3 4 5
\end{shell}

\mySamllsection{获取视图的范围适配器}

获取视图也可以(通常应该)使用范围适配器创建:

\begin{cpp}
std::views::take(rg, n)
\end{cpp}

例如:

\begin{cpp}
for (const auto& elem : std::views::take(rg, 5)) {
	std::cout << elem << ' ';
}
\end{cpp}

或:

\begin{cpp}
for (const auto& elem : rg | std::views::take(5)) {
	std::cout << elem << ' ';
}
\end{cpp}

适配器可能并不总是产生take\_view:

\begin{itemize}
\item
若传递一个empty\_view,则只返回该视图。

\item
若传递了一定大小的随机访问范围,可以初始化相同类型的范围,其中begin() + num,则返回这样的范围(这适用于子范围,iota视图,字符串视图和span)。
\end{itemize}

例如:

\begin{cpp}
std::vector<int> vec;

// using constructors:
std::ranges::take_view tv1{vec, 5}; // take view of ref view of vector
std::ranges::take_view tv2{std::vector<int>{}, 5}; // take view of owing view of vector
std::ranges::take_view tv3{std::views::iota(1,9), 5}; // take view of iota view

// using adaptors:
auto tv4 = std::views::take(vec, 5); // take view of ref view of vector
auto tv5 = std::views::take(std::vector<int>{}, 5); // take view of owning view of vector
auto tv6 = std::views::take(std::views::iota(1,9), 5); // pure iota view
\end{cpp}

获取视图在内部存储传递的范围(可选择以与all()相同的方式转换为视图),所以其只有在传递的范围有效的情况下才有效(除非传递了右值,则在内部使用了归属视图)。

迭代器是传递的范围的迭代器,若是一个有长度的随机访问范围,或者是一个对它进行计数的迭代器和一个默认的哨兵,所以只有在传递了一定长度的随机访问范围时,该范围才是通用的。

下面是使用获取视图的完整示例:

\filename{ranges/takeview.cpp}

\begin{cpp}
#include <iostream>
#include <string>
#include <vector>
#include <ranges>

void print(std::ranges::input_range auto&& coll)
{
	for (const auto& elem : coll) {
		std::cout << elem << ' ';
	}
	std::cout << '\n';
}

int main()
{
	std::vector coll{1, 2, 3, 4, 1, 2, 3, 4, 1};
	
	print(coll); // 1 2 3 4 1 2 3 4 1
	print(std::ranges::take_view{coll, 5}); // 1 2 3 4 1
	print(coll | std::views::take(5)); // 1 2 3 4 1
}
\end{cpp}

该程序有以下输出:

\begin{shell}
1 2 3 4 1 2 3 4 1
1 2 3 4 1
1 2 3 4 1
\end{shell}

\mySamllsection{获取视图的特点}

只有当底层范围是长度范围和通用范围时,获取视图才通用(迭代器和哨兵具有相同的类型)。为了协调类型,可能必须使用通用视图。

\mySamllsection{获取视图的接口}

“类std::ranges::take\_view<>的操作”表列出了获取视图的API。


% Please add the following required packages to your document preamble:
% \usepackage{longtable}
% Note: It may be necessary to compile the document several times to get a multi-page table to line up properly
\begin{longtable}[c]{|l|l|}
\hline
\textbf{操作} & \textbf{效果}                                                        \\ \hline
\endfirsthead
%
\endhead
%
take\_view r\{\}   & 创建一个引用默认构造范围的take\_view        \\ \hline
take\_view r\{rg, num\} & 创建一个引用范围rg前num个元素的take\_view                             \\ \hline
r.begin()          & 生成begin迭代器                                              \\ \hline
r.end()            & 生成哨兵(end迭代器)                                     \\ \hline
r.empty()          & 生成的r是否为空(若范围支持则可用)          \\ \hline
if (r)             & 若r不为空则为true(若定义了empty()则可用)                \\ \hline
r.size()           & 生成元素的数量(若引用了一个长度范围,则可用) \\ \hline
r.front()          & 生成的第一个元素(若转发可用)                      \\ \hline
r.back()           & 生成的最后一个元素(若双向且通用,则可用)         \\ \hline
r{[}idx{]}         & 生成的第n个元素(若随机访问可用)                    \\ \hline
r.data()           & 生成一个指向元素内存的原始指针(若元素在连续内存中可用) \\ \hline
r.base()           & 生成对r所引用或拥有的范围的引用               \\ \hline
\end{longtable}

\begin{center}
表8.14 类std::ranges::take\_view<>的操
\end{center}

\mySubsubsection{8.5.2}{获取段视图}

% Please add the following required packages to your document preamble:
% \usepackage{longtable}
% Note: It may be necessary to compile the document several times to get a multi-page table to line up properly
\begin{longtable}[c]{|l|l|}
\hline
\textbf{类型:}                 & std::ranges::take\_while\_view\textless{}\textgreater{}        \\ \hline
\endfirsthead
%
\endhead
%
\textbf{内容:}              & 范围中与谓词匹配的所有前导元素         \\ \hline
\textbf{适配器:}           & std::views::take\_while() \\ \hline
\textbf{元素类型:}      & 与传递的范围类型相同 \\ \hline
\textbf{要求:}          & 至少为输入范围      \\ \hline
\textbf{类别:}          & 和传递的一样            \\ \hline
\textbf{是否是长度范围:}    & 从不                     \\ \hline
\textbf{是否是通用范围:}   & 从不                     \\ \hline
\textbf{是否是租借范围:} & 从不                     \\ \hline
\textbf{缓存:}            & 无                   \\ \hline
\textbf{常量可迭代:}       & 若在常量可迭代的范围和谓词适用于常量的值 \\ \hline
\textbf{传播常量性:} & 只有在右值范围内                                        \\ \hline
\end{longtable}

类模板std::ranges::take\_while\_view<>定义了一个视图,该视图引用传递的范围中匹配某个谓词的所有前置元素:

\begin{cpp}
std::vector<int> rg{1, 2, 3, 4, 5, 6, 7, 8, 9, 10, 11, 12, 13};
...
for (const auto& elem : std::ranges::take_while_view{rg, [](auto x) {
									return x % 3 != 0;
							}}) {
	std::cout << elem << ' ';
}
\end{cpp}

循环的输出为:

\begin{shell}
1 2
\end{shell}

\mySamllsection{获取段视图的范围适配器}

获取段视图也可以(通常应该)使用范围适配器创建,适配器只需将其参数传递给std::ranges::take\_while\_view构造函数:

\begin{cpp}
std::views::take_while(rg, pred)
\end{cpp}

例如:

\begin{cpp}
for (const auto& elem : std::views::take_while(rg, [](auto x) {
						return x % 3 != 0;
					})) {
	std::cout << elem << ' ';
}
\end{cpp}

或:

\begin{cpp}
for (const auto& elem : rg | std::views::take_while([](auto x) {
						return x % 3 != 0;
					})) {
	std::cout << elem << ' ';
}
\end{cpp}

传递的谓词必须是满足std::predicate概念的可调用对象。这暗示了std::regular\_invocable的概念,所以谓词永远不应该修改底层范围的传递值,但不修改值是语义约束,所以在编译时不能总是检查这一点,所以至少应该将谓词声明为按值或按const引用接受实参。

获取段视图在内部存储传递的范围(可选择以与all()相同的方式转换为视图),所以只有在传递的范围有效的情况下才有效(除非传递了右值,则在内部使用了归属视图)。

迭代器只是传递的范围和一个特殊的内部哨兵类型的迭代器。

下面是使用获取段视图的完整示例:

\filename{ranges/takewhileview.cpp}

\begin{cpp}
#include <iostream>
#include <string>
#include <vector>
#include <ranges>

void print(std::ranges::input_range auto&& coll)
{
	for (const auto& elem : coll) {
		std::cout << elem << ' ';
	}
	std::cout << '\n';
}

int main()
{
	std::vector coll{1, 2, 3, 4, 1, 2, 3, 4, 1};
	print(coll); // 1 2 3 4 1 2 3 4 1
	auto less4 = [] (auto v) { return v < 4; };
	print(std::ranges::take_while_view{coll, less4}); // 1 2 3
	print(coll | std::views::take_while(less4)); // 1 2 3
}
\end{cpp}

该程序有以下输出:

\begin{shell}
1 2 3 4 1 2 3 4 1
1 2 3
1 2 3
\end{shell}

\mySamllsection{获取段视图的接口}

“类std::ranges::take\_while\_view<>的操作”表列出了获取段视图的API。

% Please add the following required packages to your document preamble:
% \usepackage{longtable}
% Note: It may be necessary to compile the document several times to get a multi-page table to line up properly
\begin{longtable}[c]{|l|l|}
\hline
\textbf{操作} & \textbf{效果}                                                        \\ \hline
\endfirsthead
%
\endhead
%
take\_while\_view r\{\}         & 创建一个引用默认构造范围的take\_while\_view                             \\ \hline
take\_while\_view r\{rg. pred\} & 创建一个take\_while\_view,引用pred为真的rg范围的前导元素 \\ \hline
r.begin()          & 生成begin迭代器                                              \\ \hline
r.end()            & 生成哨兵(end迭代器)                                      \\ \hline
r.empty()          & 生成的r是否为空(若范围支持,则可用)         \\ \hline
if (r)             & 若r不为空,则为true(若定义了empty(),则可用)                \\ \hline
r.size()           & 生成元素的数量(若引用了一个长度范围,则可用) \\ \hline
r.front()          & 生成第一个元素(若转发可用)                     \\ \hline
r{[}idx{]}         & 生成第n个元素(若为随机访问,则可用)                    \\ \hline
r.data()           & 生成指向元素内存的原始指针(若元素在连续的内存中,则可用)  \\ \hline
r.base()           & 生成对r所引用或拥有的范围的引用               \\ \hline
r.pred()           & 生成对谓词的引用                                    \\ \hline
\end{longtable}

\begin{center}
表8.15 类std::ranges::take\_while\_view<>的操作
\end{center}

\mySubsubsection{8.5.3}{丢弃视图}

% Please add the following required packages to your document preamble:
% \usepackage{longtable}
% Note: It may be necessary to compile the document several times to get a multi-page table to line up properly
\begin{longtable}[c]{|l|l|}
\hline
\textbf{类型:}                 & std::ranges::drop\_view\textless{}\textgreater{} \\ \hline
\endfirsthead
%
\endhead
%
\textbf{内容:}              & 除第num个元素外的所有元素      \\ \hline
\textbf{适配器:}              & std::views::drop()                               \\ \hline
\textbf{元素类型:}         & 与传递的范围类型相同                       \\ \hline
\textbf{要求:}             & 至少为输入范围                             \\ \hline
\textbf{类别:}             & 和传递的一样                                   \\ \hline
\textbf{是否是长度范围:}       & 若在长度范围内                              \\ \hline
\textbf{是否是通用范围:}      & 若在通用范围内                               \\ \hline
\textbf{是否是租借范围:} & 若为租借视图或左值非视图                           \\ \hline
\textbf{缓存:}            & 若没有随机访问范围或没有长度范围,则缓存begin()          \\ \hline
\textbf{常量可迭代:}    & 若是一个可常量迭代范围,该范围提供随机访问并确定长度 \\ \hline
\textbf{传播常量性:} & 只有在右值范围内                          \\ \hline
\end{longtable}

类模板std::ranges::drop\_view<>定义了一个视图,该视图引用传递的范围中除第num个元素外的所有元素,产生与获取视图相反的元素。

例如:

\begin{cpp}
std::vector<int> rg{1, 2, 3, 4, 5, 6, 7, 8, 9, 10, 11, 12, 13};
...
for (const auto& elem : std::ranges::drop_view{rg, 5}) {
	std::cout << elem << ' ';
}
\end{cpp}

循环输出:

\begin{shell}
6 7 8 9 10 11 12 13
\end{shell}

\mySamllsection{丢弃视图的范围适配器}

丢弃视图也可以(通常应该)使用范围适配器创建:

\begin{cpp}
std::views::drop(rg, n)
\end{cpp}

例如:

\begin{cpp}
for (const auto& elem : std::views::drop(rg, 5)) {
	std::cout << elem << ' ';
}
\end{cpp}

或:

\begin{cpp}
for (const auto& elem : rg | std::views::drop(5)) {
	std::cout << elem << ' ';
}
\end{cpp}

注意,适配器可能并不总是产生drop\_view:

\begin{itemize}
\item
若传递了一个空视图,则直接返回空视图。

\item
若传递了一定长度的随机访问范围,可以初始化相同类型的范围,其中begin() + num,则返回这样的范围(这适用于子范围,iota视图,字符串视图和span)。
\end{itemize}

例如:

\begin{cpp}
std::vector<int> vec;

// using constructors:
std::ranges::drop_view dv1{vec, 5}; // drop view of ref view of vector
std::ranges::drop_view dv2{std::vector<int>{}, 5}; // drop view of owing view of vector
std::ranges::drop_view dv3{std::views::iota(1,10), 5}; // drop view of iota view

// using adaptors:
auto dv4 = std::views::drop(vec, 5); // drop view of ref view of vector
auto dv5 = std::views::drop(std::vector<int>{}, 5); // drop view of owing view of vector
auto dv6 = std::views::drop(std::views::iota(1,10), 5); // pure iota view
\end{cpp}

丢弃视图在内部存储传递的范围(可选择以与all()相同的方式转换为视图),所以只有在传递的范围有效的情况下才有效(除非传递了右值,则在内部使用了归属视图)。

begin迭代器在第一次调用begin()时初始化并缓存。在没有随机访问的范围上,这需要线性时间,所以重用丢弃视图比重新创建要好。

下面是一个使用丢弃视图的完整示例:

\filename{ranges/dropview.cpp}

\begin{cpp}
#include <iostream>
#include <string>
#include <vector>
#include <ranges>

void print(std::ranges::input_range auto&& coll)
{
	for (const auto& elem : coll) {
		std::cout << elem << ' ';
	}
	std::cout << '\n';
}

int main()
{
	std::vector coll{1, 2, 3, 4, 1, 2, 3, 4, 1};
	
	print(coll); // 1 2 3 4 1 2 3 4 1
	print(std::ranges::drop_view{coll, 5}); // 2 3 4 1
	print(coll | std::views::drop(5)); // 2 3 4 1
}
\end{cpp}

该程序有以下输出:

\begin{shell}
1 2 3 4 1 2 3 4 1
2 3 4 1
2 3 4 1
\end{shell}

\mySamllsection{丢弃视图和缓存}

为了获得更好的性能(常数复杂度),丢弃视图会在视图中缓存begin()的结果(除非范围只是一个输入范围),所以对丢弃视图元素的第一次迭代要比以后的迭代更慢。

出于这个原因,最好初始化一次丢弃视图,并使用它两次:

\begin{cpp}
// better:
auto v1 = coll | std::views::drop(5);
check(v1);
process(v1);
\end{cpp}

而非初始化并使用两次:

\begin{cpp}
// worse:
check(coll | std::views::drop(5));
process(coll | std::views::drop(5));
\end{cpp}

注意,对于具有随机访问的范围(例如,数组,vector和队列),缓存的开始偏移量与视图一起复制。否则,不会复制缓存的起始位置。

当过滤器视图引用的范围被修改时,此缓存可能会产生功能性后果。

例如:

\filename{ranges/dropcache.cpp}

\begin{cpp}
#include <iostream>
#include <vector>
#include <list>
#include <ranges>

void print(auto&& coll)
{
	for (const auto& elem : coll) {
		std::cout << elem << ' ';
	}
	std::cout << '\n';
}

int main()
{
	std::vector vec{1, 2, 3, 4};
	std::list lst{1, 2, 3, 4};
	
	auto vVec = vec | std::views::drop(2);
	auto vLst = lst | std::views::drop(2);
	
	// insert a new element at the front (=> 0 1 2 3 4)
	vec.insert(vec.begin(), 0);
	lst.insert(lst.begin(), 0);
	
	print(vVec); // OK: 2 3 4
	print(vLst); // OK: 2 3 4
	
	// insert more elements at the front (=> 98 99 0 -1 0 1 2 3 4)
	vec.insert(vec.begin(), {98, 99, 0, -1});
	lst.insert(lst.begin(), {98, 99, 0, -1});
	
	print(vVec); // OK: 0 -1 0 1 2 3 4
	print(vLst); // OOPS: 2 3 4
	
	// creating a copy heals:
	auto vVec2 = vVec;
	auto vLst2 = vLst;
	
	print(vVec2); // OK: 0 -1 0 1 2 3 4
	print(vLst2); // OK: 0 -1 0 1 2 3 4
}
\end{cpp}

从形式上讲,将无效视图复制到vector会创建未定义的行为,因为C++标准没有指定如何进行缓存。由于vector的重新分配会使所有迭代器失效,因此缓存的迭代器将失效。对于随机访问范围,视图通常缓存偏移量,而不是迭代器。所以视图仍然有效,因为它仍然包含不包含前两个元素的范围。

根据经验,\textbf{不要使用在底层范围被修改后调用begin()的丢弃视图。}

\mySamllsection{丢弃视图和const}

注意,不能总是迭代const丢弃视图。实际上,所引用的范围必须是一个可随机访问范围和一定长度的范围。

例如:

\begin{cpp}
void printElems(const auto& coll) {
	for (const auto elem& e : coll) {
		std::cout << elem << '\n';
	}
}

std::vector vec{1, 2, 3, 4, 5};
std::list lst{1, 2, 3, 4, 5};

printElems(vec | std::views::drop(3)); // OK
printElems(lst | std::views::drop(3)); // ERROR
\end{cpp}

要在泛型代码中支持这个视图,必须使用通用/转发引用:

\begin{cpp}
void printElems(auto&& coll) {
	...
}

std::list lst{1, 2, 3, 4, 5};

printElems(lst | std::views::drop(3)); // OK
\end{cpp}

\mySamllsection{丢弃视图的接口}

“类std::ranges::drop\_view<>的操作”表列出了丢弃视图的API。

% Please add the following required packages to your document preamble:
% \usepackage{longtable}
% Note: It may be necessary to compile the document several times to get a multi-page table to line up properly
\begin{longtable}[c]{|l|l|}
\hline
\textbf{操作} & \textbf{效果}                                                        \\ \hline
\endfirsthead
%
\endhead
%
drop\_view r\{\}   & 创建一个drop\_view,引用一个默认构造的范围        \\ \hline
drop\_view r\{rg, num\} & 创建一个drop\_view,引用rg范围内除前num个元素外的所有元素                      \\ \hline
r.begin()          & 生成begin迭代器                                              \\ \hline
r.end()            & 生成哨兵(end迭代器)                                     \\ \hline
r.empty()          & 生成的r是否为空(若范围支持,则可用)         \\ \hline
if (r)             & 若r不为空则为true(若定义了empty(),则可用)                \\ \hline
r.size()           & 生成元素的数量(若引用了一个大小长度范围,则可用) \\ \hline
r.front()          & 生成的第一个元素(若转发可用)                      \\ \hline
r.back()           & 生成的最后一个元素(若双向且通用,则可用)         \\ \hline
r{[}idx{]}         & 生成的第n个元素(若随机访问,则可用)                    \\ \hline
r.data()           & 生成一个指向元素内存的原始指针(若元素在连续内存中,则可用)。 \\ \hline
r.base()           & 生成对r所引用或拥有的范围的引用               \\ \hline
\end{longtable}

\begin{center}
表8.16 类std::ranges::drop\_view<>的操作
\end{center}

\mySubsubsection{8.5.4}{丢弃段视图}

% Please add the following required packages to your document preamble:
% \usepackage{longtable}
% Note: It may be necessary to compile the document several times to get a multi-page table to line up properly
\begin{longtable}[c]{|l|l|}
\hline
\textbf{类型:}                 & std::ranges::drop\_while\_view\textless{}\textgreater{}        \\ \hline
\endfirsthead
%
\endhead
%
\textbf{内容:}              & 范围中除前导元素外与谓词匹配的所有元素 \\ \hline
\textbf{适配器:}              & std::views::drop\_while()                                      \\ \hline
\textbf{元素类型:}         & 与传递的范围类型相同                                      \\ \hline
\textbf{要求:}             & 至少为输入范围                                           \\ \hline
\textbf{类别:}             & 和传递的一样                                                 \\ \hline
\textbf{是否是长度范围:}       & 若传递了普通的随机访问范围                           \\ \hline
\textbf{是否是通用范围:}      & 若在常规范围内                                             \\ \hline
\textbf{是否是租借范围:}    & 若是租借视图或左值非视图                      \\ \hline
\textbf{缓存:}               & 总是缓存begin()                                          \\ \hline
\textbf{常量可迭代:}       & Never                                                          \\ \hline
\textbf{传播常量性:} & --(若为const,不能调用begin())                               \\ \hline
\end{longtable}

类模板std::ranges::drop\_while\_view<>定义了一个视图,该视图跳过与某个谓词匹配的传递范围的所有前导元素,生成了与获取段视图相反的元素:

\begin{cpp}
std::vector<int> rg{1, 2, 3, 4, 5, 6, 7, 8, 9, 10, 11, 12, 13};
...
for (const auto& elem : std::ranges::drop_while_view{rg, [](auto x) {
						return x % 3 != 0;
				}}) {
	std::cout << elem << ' ';
}
\end{cpp}

循环输出为:

\begin{shell}
3 4 5 6 7 8 9 10 11 12 13
\end{shell}

\mySamllsection{丢弃段视图的范围适配器}

丢弃段视图也可以(通常应该)使用范围适配器创建,适配器只需将其参数传递给std::ranges::drop\_while\_view构造函数:

\begin{cpp}
std::views::drop_while(rg, pred)
\end{cpp}

例如:

\begin{cpp}
for (const auto& elem : std::views::drop_while(rg, [](auto x) {
						return x % 3 != 0;
					})) {
	std::cout << elem << ' ';
}
\end{cpp}

或:

\begin{cpp}
for (const auto& elem : rg | std::views::drop_while([](auto x) {
						return x % 3 != 0;
					})) {
	std::cout << elem << ' ';
}
\end{cpp}

传递的谓词必须是满足std::predicate概念的可调用对象。这暗示了std::regular\_invocable的概念,谓词永远不应该修改底层范围的传递值,但不修改值是语义约束,所以在编译时不能总是检查这一点,所以至少应该将谓词声明为按值或按const引用接受实参。

丢弃段视图在内部存储传递的范围(可选择以与all()相同的方式转换为视图),所以只有在传递的范围有效的情况下才有效(除非传递了右值,则在内部使用了归属视图)。

begin迭代器在第一次调用begin()时初始化并缓存,这总是需要线性时间,所以重用丢弃段视图比重新创建要好。

下面是一个使用丢弃段视图的完整示例:

\filename{ranges/dropwhileview.cpp}

\begin{cpp}
#include <iostream>
#include <string>
#include <vector>
#include <ranges>

void print(std::ranges::input_range auto&& coll)
{
	for (const auto& elem : coll) {
		std::cout << elem << ' ';
	}
	std::cout << '\n';
}

int main()
{
	std::vector coll{1, 2, 3, 4, 1, 2, 3, 4, 1};
	
	print(coll); // 1 2 3 4 1 2 3 4 1
	auto less4 = [] (auto v) { return v < 4; };
	print(std::ranges::drop_while_view{coll, less4}); // 4 1 2 3 4 1
	print(coll | std::views::drop_while(less4)); // 4 1 2 3 4 1
}
\end{cpp}

该程序有以下输出:

\begin{shell}
1 2 3 4 1 2 3 4 1
4 1 2 3 4 1
4 1 2 3 4 1
\end{shell}

\mySamllsection{丢弃段视图和缓存}

为了获得更好的性能(常数复杂度),丢弃段视图会在视图中缓存begin()的结果(除非该范围只是一个输入范围)。所以对丢弃段视图元素的第一次迭代,比以后的迭代代价更慢。

出于这个原因,最好初始化一个过滤器视图,然后使用它两次:

\begin{cpp}
// better:
auto v1 = coll | std::views::drop_while(myPredicate);
check(v1);
process(v1);
\end{cpp}

而非初始化并使用它两次:

\begin{cpp}
// worse:
check(coll | std::views::drop_while(myPredicate));
process(coll | std::views::drop_while(myPredicate));
\end{cpp}

当修改过滤视图引用的范围时,缓存可能会产生功能性后果。例如:

\filename{ranges/dropwhilecache.cpp}

\begin{cpp}
#include <iostream>
#include <vector>
#include <list>
#include <ranges>

void print(auto&& coll)
{
	for (const auto& elem : coll) {
		std::cout << elem << ' ';
	}
	std::cout << '\n';
}

int main()
{
	std::vector vec{1, 2, 3, 4};
	std::list lst{1, 2, 3, 4};
	
	auto lessThan2 = [](auto v){
		return v < 2;
	};
	
	auto vVec = vec | std::views::drop_while(lessThan2);
	auto vLst = lst | std::views::drop_while(lessThan2);
	
	// insert a new element at the front (=> 0 1 2 3 4)
	vec.insert(vec.begin(), 0);
	lst.insert(lst.begin(), 0);
	
	print(vVec); // OK: 2 3 4
	print(vLst); // OK: 2 3 4
	
	// insert more elements at the front (=> 0 98 99 -1 0 1 2 3 4)
	vec.insert(vec.begin(), {0, 98, 99, -1});
	lst.insert(lst.begin(), {0, 98, 99, -1});
	
	print(vVec); // OOPS: 99 -1 0 1 2 3 4
	print(vLst); // OOPS: 2 3 4
	
	// creating a copy heals (except with random access):
	auto vVec2 = vVec;
	auto vLst2 = vLst;
	
	print(vVec2); // OOPS: 99 -1 0 1 2 3 4
	print(vLst2); // OK: 98 99 -1 0 1 2 3 4
}
\end{cpp}

从形式上讲,将无效视图复制到vector会创建未定义的行为,因为C++标准没有指定如何进行缓存。由于vector的重新分配会使所有迭代器失效,因此缓存的迭代器将失效。对于随机访问范围,视图通常缓存偏移量,而不是迭代器。因为它仍然包含一个有效范围,所以视图仍然有效,尽管开始不再适合谓词。

根据经验,\textbf{不要使用在底层范围被修改后调用begin()的丢弃段视图}。

\mySamllsection{丢弃段视图和const}

不能迭代const丢弃段视图。

例如:

\begin{cpp}
void printElems(const auto& coll) {
	for (const auto elem& e : coll) {
		std::cout << elem << '\n';
	}
}

std::vector vec{1, 2, 3, 4, 5};

printElems(vec | std::views::drop_while( ... )); // ERROR
\end{cpp}

问题在于,begin()仅为非const丢弃段视图提供,因为缓存迭代器会修改视图。

要在泛型代码中支持这个视图,必须使用通用/转发引用:

\begin{cpp}
void printElems(auto&& coll) {
	...
}

std::list lst{1, 2, 3, 4, 5};

printElems(vec | std::views::drop_while( ... )); // OK
\end{cpp}

\mySamllsection{丢弃段视图的接口}

“类std::ranges::drop\_while\_view<>的操作”表列出了丢弃段视图的API。

% Please add the following required packages to your document preamble:
% \usepackage{longtable}
% Note: It may be necessary to compile the document several times to get a multi-page table to line up properly
\begin{longtable}[c]{|l|l|}
\hline
\textbf{操作} & \textbf{效果}                                                         \\ \hline
\endfirsthead
%
\endhead
%
drop\_while\_view r\{\}         & 创建一个drop\_while\_view,引用一个默认构造的范围                                           \\ \hline
drop\_while\_view r\{rg, pred\} & 创建一个drop\_while\_view,引用rg的所有元素,除了pred为真的前导元素 \\ \hline
r.begin()          & 生成begin迭代器                                               \\ \hline
r.end()            & 生成哨兵(end迭代器)                                       \\ \hline
r.empty()          & 生成的r是否为空(若范围支持,则可用)          \\ \hline
if (r)             & 若r不为空,则为true(若定义了empty(),则可用)                 \\ \hline
r.size()           & 生成元素的数量(若引用了一个长度范围,则可用) \\ \hline
r.front()          & 生成的第一个元素(若转发可用)                       \\ \hline
r.back()           & 生成的最后一个元素(若果双向且通用,则可用)          \\ \hline
r{[}idx{]}         & 生成的第n个元素(若随机访问可用)                     \\ \hline
r.data()           & 生成生一个指向元素内存的原始指针(若元素在连续内存中可用)。              \\ \hline
r.base()           & 生成对r所引用或归属范围的引用                \\ \hline
r.pred()           & 生成对谓词的引用                                     \\ \hline
\end{longtable}

\begin{center}
表8.17 类std::ranges::drop\_while\_view<>的操作
\end{center}

\mySubsubsection{8.5.5}{过滤视图}

% Please add the following required packages to your document preamble:
% \usepackage{longtable}
% Note: It may be necessary to compile the document several times to get a multi-page table to line up properly
\begin{longtable}[c]{|l|l|}
\hline
\textbf{类型:}                 & std::ranges::filter\_view\textless{}\textgreater{} \\ \hline
\endfirsthead
%
\endhead
%
\textbf{内容:}              & 范围中与谓词匹配的所有元素     \\ \hline
\textbf{适配器:}              & std::views::filter()                               \\ \hline
\textbf{元素类型:}         & 与传递的范围类型相同                          \\ \hline
\textbf{要求:}             & 至少为输入范围                               \\ \hline
\textbf{类别:}             & 和传入的一样,但至多是双向的           \\ \hline
\textbf{是否是长度范围:}       & 从不                                              \\ \hline
\textbf{是否是通用范围:}      & 若在常规范围内                                 \\ \hline
\textbf{是否是租借范围:}    & 从不                                              \\ \hline
\textbf{缓存:}               & 总是缓存begin()                              \\ \hline
\textbf{常量可迭代:}       & 从不                                              \\ \hline
\textbf{传播常量性:} & -- (若是const,则不能使用begin())                  \\ \hline
\end{longtable}

类模板std::ranges::filter\_view<>定义了一个视图,该视图只迭代底层范围中与某个谓词匹配的元素。也就是说,过滤掉所有与谓词不匹配的元素。例如:

\begin{cpp}
std::vector<int> rg{1, 2, 3, 4, 5, 6, 7, 8, 9, 10, 11, 12, 13};
...
for (const auto& elem : std::ranges::filter_view{rg, [](auto x) {
							return x % 3 != 0;
					}}) {
	std::cout << elem << ' ';
}
\end{cpp}

循环输出:

\begin{shell}
1 2 4 5 7 8 10 11 13
\end{shell}

\mySamllsection{过滤视图的范围适配器}

过滤视图也可以(通常应该)使用范围适配器创建,适配器只需将其参数传递给std::ranges::filter\_view构造函数:

\begin{cpp}
std::views::filter(rg, pred)
\end{cpp}

例如:

\begin{cpp}
for (const auto& elem : std::views::filter(rg, [](auto x) {
						return x % 3 != 0;
					})) {
	std::cout << elem << ' ';
}
\end{cpp}

或:

\begin{cpp}
for (const auto& elem : rg | std::views::filter([](auto x) {
						return x % 3 != 0;
					})) {
	std::cout << elem << ' ';
}
\end{cpp}

传递的谓词必须是满足std::predicate概念的可调用对象。这表示了std::regular\_invocable的概念,这意味着谓词永远不应该修改底层范围的传递值。但不修改值是语义约束,在编译时不能总是检查这一点,所以至少应该将谓词声明为按值或按const引用接受实参。

过滤视图是特殊的,开发者应该知道何时、何地以及如何使用它,以及使用它所产生的副作用。事实上,对管道的性能有很大的影响,并且有时会以令人惊讶的方式限制对元素的写访问。

因此,应该谨慎地使用过滤视图:

\begin{itemize}
\item
在管道中,应该尽可能早地获得它。

\item
在使用过滤器之前,要警惕昂贵的转换。

\item
当写访问破坏谓词时,不要将其用于对元素的写访问。
\end{itemize}

过滤视图在内部存储传递的范围(可选择以与all()相同的方式转换为视图),所以只有在传递的范围有效的情况下才有效(除非传递了右值,则在内部使用了归属视图)。

begin迭代器被初始化,通常在第一次调用begin()时进行缓存,这总是需要线性时间,所以重用过滤器视图比从头创建过滤器视图要好。

下面是使用过滤器视图的完整示例:

\filename{ranges/filterview.cpp}

\begin{cpp}
#include <iostream>
#include <string>
#include <vector>
#include <ranges>

void print(std::ranges::input_range auto&& coll)
{
	for (const auto& elem : coll) {
		std::cout << elem << ' ';
	}
	std::cout << '\n';
}

int main()
{
	std::vector coll{1, 2, 3, 4, 1, 2, 3, 4, 1};
	
	print(coll); // 1 2 3 4 1 2 3 4 1
	auto less4 = [] (auto v) { return v < 4; };
	print(std::ranges::filter_view{coll, less4}); // 1 2 3 1 2 3 1
	print(coll | std::views::filter(less4)); // 1 2 3 1 2 3 1
}
\end{cpp}

该程序有以下输出:

\begin{shell}
1 2 3 4 1 2 3 4 1
1 2 3 1 2 3 1
1 2 3 1 2 3 1
\end{shell}

\mySamllsection{过滤视图和缓存}

为了获得更好的性能(常数复杂度),过滤器视图会在视图中缓存begin()的结果(除非该范围只是一个输入范围),所以过滤视图元素上的第一次迭代比以后的迭代更慢。

最好初始化一个过滤器视图,然后使用它两次:

\begin{cpp}
// better:
auto v1 = coll | std::views::drop_while(myPredicate);
check(v1);
process(v1);
\end{cpp}

而非初始化并使用它两次:

\begin{cpp}
// worse:
check(coll | std::views::drop_while(myPredicate));
process(coll | std::views::drop_while(myPredicate));
\end{cpp}

对于具有随机访问的范围(例如,数组,vector和队列),缓存的开始偏移量与视图一起复制。否则,不复制缓存的开头。

当修改过滤视图引用的范围时,缓存可能会产生功能性后果。

例如:

\filename{ranges/filtercache.cpp}

\begin{cpp}
#include <iostream>
#include <vector>
#include <list>
#include <ranges>

void print(auto&& coll)
{
	for (const auto& elem : coll) {
		std::cout << elem << ' ';
	}
	std::cout << '\n';
}

int main()
{
	std::vector vec{1, 2, 3, 4};
	std::list lst{1, 2, 3, 4};
	
	auto lessThan2 = [](auto v){
		return v < 2;
	};
	
	auto vVec = vec | std::views::filter(biggerThan2);
	auto vLst = lst | std::views::filter(biggerThan2);
	
	// insert a new element at the front (=> 0 1 2 3 4)
	vec.insert(vec.begin(), 0);
	lst.insert(lst.begin(), 0);
	
	print(vVec); // OK: 3 4
	print(vLst); // OK: 3 4
	
	// insert more elements at the front (=> 98 99 0 -1 0 1 2 3 4)
	vec.insert(vec.begin(), {98, 99, 0, -1});
	lst.insert(lst.begin(), {98, 99, 0, -1});
	
	print(vVec); // OOPS: -1 3 4
	print(vLst); // OOPS: 3 4
	
	// creating a copy heals (except with random access):
	auto vVec2 = vVec;
	auto vLst2 = vLst;
	
	print(vVec2); // OOPS: -1 3 4
	print(vLst2); // OK: 98 99 3 4
}
\end{cpp}

从形式上讲,将无效视图复制到vector会创建未定义的行为,因为C++标准没有指定如何进行缓存。由于vector的重新分配会使所有迭代器失效,因此缓存的迭代器将失效。对于随机访问范围,视图通常缓存偏移量,而不是迭代器。因为它仍然包含一个有效的范围,所以视图仍然有效,尽管begin现在是第三个元素,而不管它是否适合过滤。

根据经验,\textbf{不要使用在底层范围被修改后调用begin()的过滤视图}。

\mySamllsection{修改元素时的过滤视图}

使用过滤器视图时,对写访问有一个重要的附加限制:必须确保修改后的值仍然满足传递给过滤器的谓词。

要理解为什么会出现这种情况,请考虑以下代码:

\begin{cpp}
oid printElems(const auto& coll) {
	for (const auto elem& e : coll) {
		std::cout << elem << '\n';
	}
}

std::vector vec{1, 2, 3, 4, 5};

printElems(vec | std::views::drop_while( ... )); // ERROR
\end{cpp}

问题在于,begin()仅为非const丢弃段视图提供,因为缓存迭代器会修改视图。

要在泛型代码中支持这个视图,必须使用通用/转发引用:

\filename{ranges/viewswrite.cpp}

\begin{cpp}
#include <iostream>
#include <vector>
#include <ranges>
namespace vws = std::views;

void print(const auto& coll)
{
	std::cout << "coll: ";
	for (int i : coll) {
		std::cout << i << ' ';
	}
	std::cout << '\n';
}

int main()
{
	std::vector<int> coll{1, 4, 7, 10, 13, 16, 19, 22, 25};
	
	// view for all even elements of coll:
	auto isEven = [] (auto&& i) { return i % 2 == 0; };
	auto collEven = coll | vws::filter(isEven);
	
	print(coll);
	
	// modify even elements in coll:
	for (int& i : collEven) {
		std::cout << " increment " << i << '\n';
		i += 1; // ERROR: undefined behavior because filter predicate is broken
	}
	print(coll);
	
	// modify even elements in coll:
	for (int& i : collEven) {
		std::cout << " increment " << i << '\n';
		i += 1; // ERROR: undefined behavior because filter predicate is broken
	}
	print(coll);
}
\end{cpp}

对集合中的偶数元素迭代两次并自增,菜鸟开发者会假设得到以下输出:

\begin{shell}
coll: 1 4 7 10 13 16 19 22 25
coll: 1 5 7 11 13 17 19 23 25
coll: 1 5 7 11 13 17 19 23 25
\end{shell}

但程序有以下输出:

\begin{shell}
coll: 1 4 7 10 13 16 19 22 25
coll: 1 5 7 11 13 17 19 23 25
coll: 1 6 7 11 13 17 19 23 25
\end{shell}

第二次迭代中,再次增加前面的第一个偶数元素。为什么?

第一次使用确保只处理偶数元素的视图时,工作得很好。但视图缓存与谓词匹配的第一个元素的位置,以便begin()不必重新计算它。当我们访问那里的值时,过滤器不会再次应用谓词,因为它已经知道这是第一个匹配元素。当第二次迭代时,过滤器返回前一个元素。但对于所有其他元素,必须再次执行检查,因为它们现在都是奇数,所以过滤器没有找到更多的元素。

关于使用过滤器的一次写访问是否应该是格式良好的,即使它破坏了过滤器的谓词,也有一些讨论。因为修改不应违反过滤器谓词的要求,使一些非常合理的示例无效:下面是其中之一:

\begin{cpp}
for (auto& m : collOfMonsters | filter(isDead)) {
	m.resurrect(); // a shaman’s doing, of course
}
\end{cpp}

这段代码通常可以编译并运行。但形式上,又有了未定义的行为,因为过滤器的谓词(“怪物必须是死的”)被破坏了。对(死亡)怪物的其他“修改”都是可能的(例如,“烧”它们)。

要中断谓词,必须使用普通循环:

\begin{cpp}
for (auto& m : collOfMonsters) {
	if (m.isDead()) {
		m.resurrect(); // a shaman’s doing, of course
	}
}
\end{cpp}

\mySamllsection{过滤视图和const}

注意,不能在const过滤视图上迭代。

例如:

\begin{cpp}
void printElems(const auto& coll) {
	for (const auto elem& e : coll) {
		std::cout << elem << '\n';
	}
}

std::vector vec{1, 2, 3, 4, 5};

printElems(vec | std::views::filter( ... )); // ERROR
\end{cpp}

问题在于,begin()仅为非const过滤视图提供,因为缓存迭代器会修改视图。

要在泛型代码中支持这个视图,必须使用通用/转发引用:

\begin{cpp}
void printElems(auto&& coll) {
	...
}

std::list lst{1, 2, 3, 4, 5};

printElems(vec | std::views::filter( ... )); // OK
\end{cpp}

\mySamllsection{管道中的过滤视图}

当在管道中使用过滤视图时,有几个问题需要考虑:

\begin{itemize}
\item
在管道中,应该尽可能早地获得它。

\item
过滤器之前,要警惕昂贵的转换。

\item
当在修改元素的同时使用过滤器时,确保在这些修改之后,过滤器的谓词仍然是满足的。详情见下文。
\end{itemize}

这样做的原因是,过滤器视图前面的视图和适配器可能必须多次评估元素,一次是决定它们是否通过过滤,一次是决定最终使用的值。

因此,像下面这样的管道:

\begin{cpp}
rg | std::views::filter(pred) | std::views::transform(func)
\end{cpp}

要比下面的方式有更好的性能比

\begin{cpp}
rg | std::views::transform(func) | std::views::filter(pred)
\end{cpp}

\mySamllsection{过滤视图的接口}

“类std::ranges::filter\_view<>的操作”表列出了过滤视图的API。

过滤视图从不提供size()、data()或[]操作符,因其既不确定大小,也不提供随机访问。

% Please add the following required packages to your document preamble:
% \usepackage{longtable}
% Note: It may be necessary to compile the document several times to get a multi-page table to line up properly
\begin{longtable}[c]{|l|l|}
\hline
\textbf{操作} & \textbf{效果}                                                \\ \hline
\endfirsthead
%
\endhead
%
filter\_view r\{\}         & 创建一个引用默认构造范围的filter\_view               \\ \hline
filter\_view r\{rg. pred\} & 创建一个filter\_view,它引用pred为true的所有元素或rg \\ \hline
r.begin()          & 生成begin迭代器                                      \\ \hline
r.end()            & 生成哨兵(end迭代器)                              \\ \hline
r.empty()          & 生成的r是否为空(若范围支持,则可用)  \\ \hline
if (r)             & 若r不为空,则为true(若定义了empty(),则可用)       \\ \hline
r.front()          & 生成的第一个元素(若转发可用)              \\ \hline
r.back()           & 生成的最后一个元素(若为双向且通用,则可用) \\ \hline
r.base()           & 生成指向r的引用范围       \\ \hline
r.pred()           & 生成对谓词的引用                            \\ \hline
\end{longtable}

\begin{center}
表8.18 类std::ranges::filter\_view<>的操作
\end{center}
