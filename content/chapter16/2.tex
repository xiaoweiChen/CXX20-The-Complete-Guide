

模块的目的是处理分布在多个文件上的大量代码。模块可用于包装由2个、10个甚至100个文件组成的小型、中型和超大型组件的代码。这些文件甚至可能由多个开发者和团队提供和维护。

为了演示这种方法的可扩展性及其好处,现在来看一下如何使用多个文件来定义可以被其他代码使用/导入的模块。示例的代码大小仍然很小,通常不会将其分散到多个文件中。我们的目标是用非常简单的示例来演示这些特性。

\mySubsubsection{16.2.1}{模块单元}

通常,模块由多个模块单元组成。模块单元是属于一个模块的翻译单元。

所有模块单元都必须以某种方式编译,只包含声明(传统代码中的头文件),也需要进行某种预编译。因此,这些文件总可转换成某种特定于平台的内部格式,以避免不得不一次又一次地(预)编译相同的代码。

除了主要的模块接口单元,C++还提供了另外三种单元类型来将模块的代码拆分为多个文件:

\begin{itemize}
\item 
模块实现单元允许开发者在自己的文件中实现定义,这样就可以单独编译(类似于传统的C++源代码在.cpp文件中)。

\item 
内部分区允许开发者在单独的文件中提供仅在模块内可见的声明和定义。

\item 
接口分区甚至允许开发者将导出的模块API拆分为多个文件。
\end{itemize}

下一节将通过示例介绍这些额外的模块单元。

\mySubsubsection{16.2.2}{使用已实现的单元}

第一个在多个文件中实现的模块示例演示了,如何分割定义(例如函数实现),以避免将它们放在一个文件中。这样做是为了能够分别编译定义。

可以通过使用模块实现(正式名称是模块实现单元)来完成,其处理方式与单独编译源文件类似。

让我们来看一个例子。

\mySamllsection{带有全局模块的主接口}

和往常一样,首先需要定义导出内容的主接口:

\filename{modules/mod1/mod1.cppm}

\begin{cpp}
module; // start module unit with global module fragment

#include <string>
#include <vector>

export module Mod1; // module declaration

struct Order {
	int count;
	std::string name;
	double price;
	
	Order(int c, const std::string& n, double p)
	: count{c}, name{n}, price{p} {
	}
};

export class Customer {
private:
	std::string name;
	std::vector<Order> orders;
public:
	Customer(const std::string& n)
	: name{n} {
	}
	void buy(const std::string& ordername, double price) {
		orders.push_back(Order{1, ordername, price});
	}
	void buy(int num, const std::string& ordername, double price) {
		orders.push_back(Order{num, ordername, price});
	}
	double sumPrice() const;
	double averagePrice() const;
	void print() const;
};
\end{cpp}

这一次,模块从模块开始;来表示我们有一个模块,并且可以在模块中使用一些预处理命令:

\begin{cpp}
module; // start module unit with global module fragment

#include <iostream>
#include <string>
#include <vector>

export module Mod1; // module declaration
...
\end{cpp}

模块之间的区域;模块可声明称为全局模块。可以使用它来放置预处理器命令,如\#define和\#include,该区域中的内容都不会导出(没有宏、没有声明、没有定义)。

在用声明正式启动模块单元之前,不能做其他事情(当然,注释除外):

\begin{cpp}
export module mod1; // module declaration
\end{cpp}

这个模块中定义的是:

\begin{itemize}
\item 
内部数据结构

\begin{cpp}
struct Order {
	...
};
\end{cpp}

此数据结构用于订单信息。每个信息保存有关订购了多少项、名称和价格的信息。构造函数确保初始化所有成员。

\item 
一个customer类,我们导出它:

\begin{cpp}
export class Customer {
	...
};
\end{cpp}
\end{itemize}

我们需要头文件和内部数据结构Order来定义Customer类,但由于不导出它们,导入该模块的代码就不能直接使用。

对于Customer类,只声明了成员函数averagePrice()、sumPrice()和print()。这里,使用特性在模块实现单元中定义它们。

\mySamllsection{模块实现单元}

一个模块可以有任意数量的实现单元。示例中,我们提供了其中的两个:一个用于实现数值操作,另一个用于实现I/O操作。

数值运算的模块实现单元如下所示:

\filename{modules/mod1/mod1price.cpp}

\begin{cpp}
module Mod1; // implementation unit of module Mod1

double Customer::sumPrice() const
{
	double sum = 0.0;
	for (const Order& od : orders) {
		sum += od.count * od.price;
	}
	return sum;
}

double Customer::averagePrice() const
{
	if (orders.empty()) {
		return 0.0;
	}
	return sumPrice() / orders.size();
}
\end{cpp}

该文件是一个模块实现单元,以声明这是模块Mod1的文件开始:

\begin{cpp}
module Mod1;
\end{cpp}

该声明导入了模块的主接口单元(但没有其他内容),所以Order和Customer类型的声明是已知的,可以直接提供它们的成员函数的实现。

注意,模块实现单元不导出任何东西。导出只允许在模块(主接口或接口分区)的接口文件中进行,这些文件是用Export module声明的(记住,每个模块只允许一个主接口)。

同样,模块实现单元可以从全局模块开始,可以在I/O模块实现单元中看到:

\filename{modules/mod1/mod1io.cpp}

\begin{cpp}
module; // start module unit with global module fragment

#include <iostream>
#include <format>

module Mod1; // implementation unit of module Mod1

void Customer::print() const
{
	// print name:
	std::cout << name << ":\n";
	// print order entries:
	for (const auto& od : orders) {
		std::cout << std::format("{:3} {:14} {:6.2f} {:6.2f}\n",
								  od.count, od.name, od.price, od.count * od.price);
	}
	// print sum:
	std::cout << std::format("{:25} ------\n", ' ');
	std::cout << std::format("{:25} {:6.2f}\n", " Sum:", sumPrice());
}
\end{cpp}

这里,用模块来介绍模块;为我们在实现单元中使用的头文件提供一个全局模块,<format>是新格式库的头文件。

模块实现单元使用传统C++翻译单元的文件扩展名(大多数情况下是.cpp),编译器就像处理其他非模块的C++代码一样处理它们。

\mySamllsection{使用模块}

使用该模块的代码如下所示:

\filename{modules/mod1/testmod1.cpp}

\begin{cpp}
#include <iostream>

import Mod1;

int main()
{
	Customer c1{"Kim"};
	
	c1.buy("table", 59.90);
	c1.buy(4, "chair", 9.20);
	
	c1.print();
	std::cout << " Average: " << c1.averagePrice() << '\n';
}
\end{cpp}

这里,使用从主界面导出的Customer类来创建客户、下订单、输出带有所有订单的客户,以及打印订单的平均值。

该程序有以下输出:

\begin{shell}
Kim:
  1 table        59.90   59.90
  4 chair         9.20   36.80
                        ------
    Sum:                 96.70
Average: 48.35
\end{shell}

注意,导入模块的代码中使用Order类型的尝试都会导致编译时错误。

还要注意,模块的使用并不取决于我们有多少实现单元。实现单元的数量之所以重要,只是因为链接器必须使用为其生成的目标文件。

\mySubsubsection{16.2.3}{内部分区}

在前面的示例中,我们介绍了仅在模块内使用的数据结构Order。看起来必须在主接口中声明它,以使其对所有实现单元可用。当然,这在大型项目中是无法扩展的。使用内部分区,可以在单独的文件中声明和定义模块的内部类型和函数。请注意,分区还可以用于在单独的文件中定义导出接口的各个部分,我们将在后面讨论。

请注意,内部分区有时被称为分区实现单元,这是基于这样一个事实:C++20标准中,其正式的名称为“模块分区的模块实现单元”,这听起来像是提供接口分区的实现,但并非如此。其和模块的内部头文件一样,可以提供声明和定义。

\mySamllsection{定义内部分区}

使用内部分区,可以在自己的模块单元中定义本地类型Order,如下所示:

\filename{modules/mod2/mod2order.cppp}

\begin{cpp}
module; // start module unit with global module fragment

#include <string>

module Mod2:Order; // internal partition declaration

struct Order {
	int count;
	std::string name;
	double price;
	
	Order(int c, const std::string& n, double p)
		: count{c}, name{n}, price{p} {
	}
};
\end{cpp}

分区有模块名,然后是冒号,然后是分区名:

\begin{cpp}
module Mod2:Order;
\end{cpp}

不支持Mod2:Order:Main的子分区。

还需要了解的是,该文件使用了另一个新的文件扩展名:.cppp,我们将在稍后查看其内容后讨论它。

主接口只能使用名称来导入这个分区:Order:

\filename{modules/mod2/mod2.cpp}

\begin{cpp}
module; // start module unit with global module fragment

#include <string>
#include <vector>

export module Mod2; // module declaration

import :Order; // import internal partition Order

export class Customer {
private:
	std::string name;
	std::vector<Order> orders;
public:
	Customer(const std::string& n)
	: name{n} {
	}
	void buy(const std::string& ordername, double price) {
		orders.push_back(Order{1, ordername, price});
	}
	void buy(int num, const std::string& ordername, double price) {
		orders.push_back(Order{num, ordername, price});
	}
	double sumPrice() const;
	double averagePrice() const;
};
\end{cpp}

主接口必须导入内部分区,因为它使用Order类型。通过导入,分区在模块的所有单元中都可用。若主接口不需要Order类型,也不导入内部分区,则所有需要Order类型的模块单元都必须直接导入内部分区。

再次注意,分区只是模块的内部实现方面。对于代码的用户来说,代码是在主模块中、实现中还是在内部分区中都无关紧要,但不能导出内部分区中的代码。

\mySubsubsection{16.2.4}{分区的接口}

还可以将模块的接口拆分为多个文件,可以声明接口分区,这些分区本身可以导出相应的内容。

若模块提供由不同开发者和/或团队维护的多个接口,接口分区特别有用。为简单起见,炸裂只使用当前的示例来演示如何通过在单独的文件中只定义Customer接口来使用该特性。

为了只定义Customer接口,可以提供以下文件:

\filename{modules/mod3/mod3customer.cppm}

\begin{cpp}
module; // start module unit with global module fragment

#include <string>
#include <vector>

export module Mod3:Customer; // interface partition declaration

import :Order; // import internal partition to use Order

export class Customer {
private:
	std::string name;
	std::vector<Order> orders;
public:
	Customer(const std::string& n)
	: name{n} {
	}
	void buy(const std::string& ordername, double price) {
		orders.push_back(Order{1, ordername, price});
	}
	void buy(int num, const std::string& ordername, double price) {
		orders.push_back(Order{num, ordername, price});
	}
	double sumPrice() const;
	double averagePrice() const;
	void print() const;
};
\end{cpp}

分区或多或少是前主接口,但有一点不同:

\begin{itemize}
\item 
作为一个分区,在模块名和冒号后面声明其名称:Mod3:Customer
\end{itemize}

像主接口一样:

\begin{itemize}
\item 
导出这个模块分区:

\begin{cpp}
export module Mod3:Customer;
\end{cpp}

\item 
使用新的文件扩展名.cppm,稍后将再次讨论
\end{itemize}

主接口仍然是指定模块导出内容的唯一地方,但主模块可以将导出委托给接口分区。这样做的方法是将导入的接口分区作为一个整体直接导出:

\filename{modules/mod3/mod3.cppm}

\begin{cpp}
export module Mod3; // module declaration

export import :Customer; // import and export interface partition Customer
... // import and export other interface partitions
\end{cpp}

通过同时导入接口分区和导出接口分区(两个关键字都要写),主接口导出分区Customer的接口作为自己的接口:

\begin{cpp}
export import :Customer; // import and export partition Customer
\end{cpp}

不允许导入接口分区,而不导出接口分区。

分区只是模块的内部实现方面,接口和实现是否在分区中提供并不重要,并且分区不会创建新的作用域。

因此,对于Customer的成员函数的实现,将类的声明移动到分区中并不重要。作为模块Mod3的一部分,实现了类Customer的成员函数:

\filename{modules/mod3/mod3io.cppm}

\begin{cpp}
module; // start module unit with global module fragment

#include <iostream>
#include <vector>
#include <format>

module Mod3; // implementation unit of module Mod3

import :Order; // import internal partition to use Order

void Customer::print() const
{
	// print name:
	std::cout << name << ":\n";
	// print order entries:
	for (const Order& od : orders) {
		std::cout << std::format("{:3} {:14} {:6.2f} {:6.2f}\n",
								   od.count, od.name, od.price, od.count * od.price);
	}
	// print sum:
	std::cout << std::format("{:25} ------\n", ' ');
	std::cout << std::format("{:25} {:6.2f}\n", " Sum:", sumPrice());
}
\end{cpp}

然而,这个实现单元中有一个不同之处:因为主接口不再导入内部分区:Order。这个模块必须这样做,因为它使用了Order类型。

对于导入模块的代码,内部分发代码的方式也无关紧要,仍然可在全局作用域中导出Customer类:

\filename{modules/mod3/testmod3.cpp}

\begin{cpp}
#include <iostream>

import Mod3;

int main()
{
	Customer c1{"Kim"};
	
	c1.buy("table", 59.90);
	c1.buy(4, "chair", 9.20);
	
	c1.print();
	std::cout << " Average: " << c1.averagePrice() << '\n';
}
\end{cpp}


\mySubsubsection{16.2.5}{将模块拆分到不同文件的总结}

本节中的示例演示了如何处理不断增加的代码大小的模块,以便拆分代码对“驯服野兽”有帮助,甚至是必要的:

\begin{itemize}
\item 
模块实现单元允许项目将定义拆分为多个文件,这样源代码可以由不同的程序员维护,若局部情况发生变化,不必重新编译所有代码。

\item 
内部分区允许项目将模块局部声明和定义移到主接口之外。可以由主接口导入,也可以只由需要它们的模块单元导入。

\item 
接口分区允许项目在不同的文件中维护导出的接口。若导出的API变得如此之大,以至于有不同的文件(以及团队)来处理其中的一部分,这就是有意义的。
\end{itemize}

主接口将所有内容集合在一起,并指定导出给模块用户的内容(通过直接导出符号或导出导入的接口分区)。

我们拥有的模块单元类型取决于C++源文件中的模块声明(可以在注释和预处理器命令的全局模块之后):

\begin{itemize}
\item 
export module name; 

主接口。对于每个模块,只能在C++程序中存在一次。

\item 
module name;

仅提供定义(可能使用局部声明)的实现单元。想要多少有多少。

\item 
module name:partname;

一个内部分区,声明和定义仅在模块内使用。可以有多个分区,但是对于每个partname,只能有一个内部分区文件。

\item 
export module name:partname; 

一个接口分区。可以有多个接口分区,但是对于每个partname,只能有一个接口分区文件。
\end{itemize}

因为没有针对不同模块单元的标准后缀,所以工具必须解析C++源文件的头部,以检测它们是否是模块单元,以及是哪种类型的模块单元。注意,模块声明可能发生在注释和全局模块之后。请参阅\url{http://github.com/josuttis/cppmodules}中的clmod.py,以获得一个Python脚本,该脚本演示处理模块时可能出现的情况。






