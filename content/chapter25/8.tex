


\mySamllsection{resource acquisition is initialization (RAII)}

A programming pattern to delegate clean-ups necessary for the end of using a resource to a destructor so that the clean-ups happen automatically when the object that represents the resource leaves its scope or ends its lifetime.


\mySamllsection{regular type}

A type that matches the semantics of built-in value types (such as int). Based on the definition in \url{http://stepanovpapers.com/DeSt98.pdf}, a regular type provides the following basic operations:

\begin{itemize}
\item [-]
Default construction (T x;)

\item [-]
Copying (T y = x;) and in C++, moving

\item [-]
Assignment (x = y;) and in C++, move assignment

\item [-]
Equality and inequality (x == y and x != y)

\item [-]
Ordering (x < y etc.)
\end{itemize}

These operations follow the “usual” naive rules:

\begin{itemize}
\item [-]
If one object is a copy of the other, the objects are equal.

\item [-]
A copy of an object is equal to an object created with the default constructor to which the source value was assigned.

\item [-]
Objects have value semantics. If two objects are equal and we modify one of them, they are no longer equal.
\end{itemize}

\mySamllsection{rvalue}

A value category of expressions that are not lvalues. An rvalue can be a prvalue (such as a temporary object without a name) or an xvalue (e.g., an lvalue marked with std::move()). What was called an rvalue before C++11 is called a prvalue since C++11.

















