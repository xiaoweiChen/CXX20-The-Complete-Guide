当在协程内部抛出异常并且该异常没有得到本地处理时,unhandled\_exception()将在协程主体后面的catch子句中调用(参见图15.1)。在出现initial\_suspend()、yield\_value()、return\_void()、return\_value()或协程中使用的任何可等待对象的异常情况时,也会调用该函数。

以下几种方式来处理这些异常:

\begin{itemize}
\item
忽略异常

unhandled\_exception()只有一个空函数体:

\begin{cpp}
void unhandled_exception() {
}
\end{cpp}

\item
本地处理异常

unhandled\_exception()只处理异常。在catch子句中,必须重新抛出异常并在本地处理:

\begin{cpp}
void unhandled_exception() {
	try {
		throw; // rethrow caught exception
	}
	catch (const std::exception& e) {
		std::cerr << "EXCEPTION: " << e.what() << std::endl;
	}
	catch (...) {
		std::cerr << "UNKNOWN EXCEPTION" << std::endl;
	}
}
\end{cpp}

\item
结束或终止程序(例如,通过调用std::terminate()):

\begin{cpp}
void unhandled_exception() {
	...
	std::terminate();
}
\end{cpp}

\item
使用std::current\_exception()存储异常以供以后使用

需要在promise类型中设置一个异常指针:

\begin{cpp}
struct promise_type {
	std::exception_ptr ePtr;
	...
	void unhandled_exception() {
		ePtr = std::current_exception();
	}
};
\end{cpp}

必须提供在协程接口中处理异常的方法。例如:

\begin{cpp}
class [[nodiscard]] CoroTask {
	...
	bool resume() const {
		if (!hdl || hdl.done()) {
			return false;
		}
		hdl.promise().ePtr = nullptr; // no exception yet
		hdl.resume(); // RESUME
		if (hdl.promise().ePtr) { // RETHROW any exception from the coroutine
			std::rethrow_exception(hdl.promise().ePtr);
		}
		return !hdl.done();
	}
};
\end{cpp}
\end{itemize}

当然,这些方法也可以进行组合。

调用unhandled\_exception()而不结束程序之后,协程就结束了。直接调用final\_suspend(),协程会挂起。

若在unhandled\_exception()中抛出或重新抛出异常,协程也会挂起。unhandled\_exception()抛出的任何异常都会忽略。









































