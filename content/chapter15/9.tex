

原则上,将操作数传递给co\_await可以执行任何代码。可以跳转到一个非常不同的上下文,或者执行一些操作,并在得到结果后继续执行。即使其他操作在不同的线程中运行,也不需要同步机制。

本节提供一个基本示例来进行演示,其一部分是处理协程的简单线程池,使用std::jthread和C++20的一些新的并发特性。

\mySubsubsection{15.9.1}{co\_await协程}

假设有以下相互调用的协程:

\filename{coro/coroasync.hpp}

\begin{cpp}
#include "coropool.hpp"
#include <iostream>
#include <syncstream> // for std::osyncstream

inline auto syncOut(std::ostream& strm = std::cout) {
	return std::osyncstream{strm};
}

CoroPoolTask print(std::string id, std::string msg)
{
	syncOut() << "       > " << id << " print: " << msg
			  << "       on thread: " << std::this_thread::get_id() << std::endl;
	co_return; // make it a coroutine
}

CoroPoolTask runAsync(std::string id)
{
	syncOut() << "===== " << id << " start "
			  << "   on thread: " << std::this_thread::get_id() << std::endl;

	co_await print(id + "a", "start");
	syncOut() << "===== " << id << " resume "
			  << "   on thread " << std::this_thread::get_id() << std::endl;

	co_await print(id + "b", "end ");
	syncOut() << "===== " << id << " resume "
			  << "   on thread " << std::this_thread::get_id() << std::endl;

	syncOut() << "===== " << id << " done" << std::endl;
}
\end{cpp}

这两个协程都使用CoroPoolTask,一个在线程池中运行任务的协程接口。接口和池的实现将在后面讨论。

重要的部分是协程runAsync()使用co\_await调用另一个协程print():

\begin{cpp}
CoroPoolTask runAsync(std::string id)
{
	...
	co_await print( ... );
	...
}
\end{cpp}

这样做的效果是,协程print()将被安排在线程池中,在不同的线程中运行。此外,co\_await阻塞直到print()完成。

print()需要一个co\_return来确保它被编译器视为协程。若没有这个,就不会有任何co\_关键字,编译器(假设这是一个普通的函数)会抱怨有返回类型,但没有返回语句。

使用辅助函数syncOut(),会产生一个std::osyncstream,以确保来自不同线程的并发输出逐行同步。

假设按如下方式调用协程runAsync():

\filename{coro/coroasync1.cpp}

\begin{cpp}
#include "coroasync.hpp"
#include <iostream>

int main()
{
	// init pool of coroutine threads:
	syncOut() << "**** main() on thread " << std::this_thread::get_id()
			  << std::endl;
	CoroPool pool{4};

	// start main coroutine and run it in coroutine pool:
	syncOut() << "runTask(runAsync(1))" << std::endl;
	CoroPoolTask t1 = runAsync("1");
	pool.runTask(std::move(t1));

	// wait until all coroutines are done:
	syncOut() << "\n**** waitUntilNoCoros()" << std::endl;
	pool.waitUntilNoCoros();

	syncOut() << "\n**** main() done" << std::endl;
}
\end{cpp}

首先使用CoroPoolTask接口为所有协程创建一个CoroPool类型的线程池:

\begin{cpp}
CoroPool pool{4};
\end{cpp}

然后,使用协程,惰性地启动协程并返回接口,并将接口交给池来控制协程:

\begin{cpp}
CoroPoolTask t1 = runAsync("1");
pool.runTask(std::move(t1));
\end{cpp}

为协程接口使用(并且必须使用)移动语义,因为runTask()需要传递一个右值,以便能够接管协程的所有权(t1不再拥有它)。

也可以用一条语句来实现:

\begin{cpp}
pool.runTask(runAsync("1"));
\end{cpp}

结束这个项目时,协程池会阻塞,直到所有调度的协程都处理完:

\begin{cpp}
pool.waitUntilNoCoros();
\end{cpp}

当运行这个程序时,会得到如下输出(线程id不同):

\begin{shell}
**** main() on thread 0x80000008
runTask(runAsync(1))

**** waitUntilNoCoros()
===== 1 start on thread: 0x8002cd90
    > 1a print: start on thread: 0x8002ce68
===== 1 resume on thread 0x8002ce68
    > 1b print: end on thread: 0x8004d090
===== 1 resume on thread 0x8004d090
===== 1 done

**** main() done
\end{shell}

这里使用不同的线程:

\begin{itemize}
\item
协程runAsync()在与main()不同的线程上启动。

\item
调用的协程print()从第三个线程开始

\item
第二次调用协程print()是从第四个线程开始的
\end{itemize}

然而,每次调用print()时,runAsync()都会改变线程。通过使用co\_await,这些调用挂起runAsync()并在另一个线程上调用print()(挂起调度池中调用的协程)。print()结束时,调用协同程序runAsync()在print()运行的同一线程上恢复。

作为一种变体,还可以启动并调度四次协程runAsync():

\filename{coro/coroasync2.cpp}

\begin{cpp}
#include "coroasync.hpp"
#include <iostream>

int main()
{
	// init pool of coroutine threads:
	syncOut() << "**** main() on thread " << std::this_thread::get_id()
			  << std::endl;
	CoroPool pool{4};

	// start multiple coroutines and run them in coroutine pool:
	for (int i = 1; i <= 4; ++i) {
		syncOut() << "runTask(runAsync(" << i << "))" << std::endl;
		pool.runTask(runAsync(std::to_string(i)));
	}

	// wait until all coroutines are done:
	syncOut() << "\n**** waitUntilNoCoros()" << std::endl;
	pool.waitUntilNoCoros();

	syncOut() << "\n**** main() done" << std::endl;
}
\end{cpp}

这个程序可能有以下输出(由于不同的平台使用不同的线程id):

\begin{shell}
**** main() on thread 17308
runTask(runAsync(1))
runTask(runAsync(2))
runTask(runAsync(3))
runTask(runAsync(4))

**** waitUntilNoCoros()
===== 1 start on thread: 18016
===== 2 start on thread: 9004
===== 3 start on thread: 17008
===== 4 start on thread: 2816
    > 2a print: start on thread: 2816
    > 1a print: start on thread: 17008
===== 1 resume on thread 17008
    > 4a print: start on thread: 18016
===== 4 resume on thread 18016
    > 3a print: start on thread: 9004
===== 3 resume on thread 9004
===== 2 resume on thread 2816
    > 4b print: end on thread: 9004
    > 1b print: end on thread: 2816
===== 1 resume on thread 2816
===== 1 done
===== 4 resume on thread 9004
===== 4 done
    > 2b print: end on thread: 17008
===== 2 resume on thread 17008
===== 2 done
    > 3b print: end on thread: 18016
===== 3 resume on thread 18016
===== 3 done

**** main() done
\end{shell}

输出现在变化更大,因为有使用下一个可用线程的并发协程。总的来说,其有相同的行为:

\begin{itemize}
\item
每当协程调用print()时,都会调度并可能在不同的线程上启动(若没有其他线程可用,则可能是同一个线程)。

\item
每当print()完成时,调用协同程序runAsync()直接在print()运行的同一个线程上恢复,所以每次调用print()时,runAsync()都改变了运行线程。
\end{itemize}

\mySubsubsection{15.9.2}{协程任务的线程池}

下面是协程接口CoroPoolTask和对应的线程池类CoroPool的实现:

\filename{coro/coropool.hpp}

\begin{cpp}
#include <iostream>
#include <list>
#include <utility> // for std::exchange()
#include <functional> // for std::function
#include <coroutine>
#include <thread>
#include <mutex>
#include <condition_variable>
#include <functional>

class CoroPool;

class [[nodiscard]] CoroPoolTask
{
	friend class CoroPool;
public:
	struct promise_type;
	using CoroHdl = std::coroutine_handle<promise_type>;
private:
	CoroHdl hdl;
public:
	struct promise_type {
		CoroPool* poolPtr = nullptr; // if not null, lifetime is controlled by pool
		CoroHdl contHdl = nullptr; // coro that awaits this coro

		CoroPoolTask get_return_object() noexcept {
			return CoroPoolTask{CoroHdl::from_promise(*this)};
		}
		auto initial_suspend() const noexcept { return std::suspend_always{}; }
		void unhandled_exception() noexcept { std::exit(1); }
		void return_void() noexcept { }

		auto final_suspend() const noexcept {
			struct FinalAwaiter {
				bool await_ready() const noexcept { return false; }
				std::coroutine_handle<> await_suspend(CoroHdl h) noexcept {
					if (h.promise().contHdl) {
						return h.promise().contHdl; // resume continuation
					}
					else {
						return std::noop_coroutine(); // no continuation
					}
				}
				void await_resume() noexcept { }
			};
			return FinalAwaiter{}; // AFTER suspended, resume continuation if there is one
		}
	};

	explicit CoroPoolTask(CoroHdl handle)
	: hdl{handle} {
	}
	~CoroPoolTask() {
		if (hdl && !hdl.promise().poolPtr) {
			// task was not passed to pool:
			hdl.destroy();
		}
	}
	CoroPoolTask(const CoroPoolTask&) = delete;
	CoroPoolTask& operator= (const CoroPoolTask&) = delete;
	CoroPoolTask(CoroPoolTask&& t)
		: hdl{t.hdl} {
			t.hdl = nullptr;
	}
	CoroPoolTask& operator= (CoroPoolTask&&) = delete;

	// Awaiter for: co_await task()
	// - queues the new coro in the pool
	// - sets the calling coro as continuation
	struct CoAwaitAwaiter {
		CoroHdl newHdl;
		bool await_ready() const noexcept { return false; }
		void await_suspend(CoroHdl awaitingHdl) noexcept; // see below
		void await_resume() noexcept {}
	};
	auto operator co_await() noexcept {
		return CoAwaitAwaiter{std::exchange(hdl, nullptr)}; // pool takes ownership of hdl
	}
};

class CoroPool
{
private:
	std::list<std::jthread> threads; // list of threads
	std::list<CoroPoolTask::CoroHdl> coros; // queue of scheduled coros
	std::mutex corosMx;
	std::condition_variable_any corosCV;
	std::atomic<int> numCoros = 0; // counter for all coros owned by the pool

public:
	explicit CoroPool(int num) {
		// start pool with num threads:
		for (int i = 0; i < num; ++i) {
			std::jthread worker_thread{[this](std::stop_token st) {
					threadLoop(st);
			}};
			threads.push_back(std::move(worker_thread));
		}
	}

	~CoroPool() {
		for (auto& t : threads) { // request stop for all threads
			t.request_stop();
		}
		for (auto& t : threads) { // wait for end of all threads
			t.join();
		}
		for (auto& c : coros) { // destroy remaining coros
			c.destroy();
		}
	}

	CoroPool(CoroPool&) = delete;
	CoroPool& operator=(CoroPool&) = delete;

	void runTask(CoroPoolTask&& coroTask) noexcept {
		auto hdl = std::exchange(coroTask.hdl, nullptr); // pool takes ownership of hdl
		if (coroTask.hdl.done()) {
			coroTask.hdl.destroy(); // OOPS, a done() coroutine was passed
		}
		else {
			schedule coroutine in the pool
		}
	}

	// runCoro(): let pool run (and control lifetime of) coroutine
	// called from:
	// - pool.runTask(CoroPoolTask)
	// - co_await task()
	void runCoro(CoroPoolTask::CoroHdl coro) noexcept {
		++numCoros;
		coro.promise().poolPtr = this; // disables destroy in CoroPoolTask
		{
			std::scoped_lock lock(corosMx);
			coros.push_front(coro); // queue coro
			corosCV.notify_one(); // and let one thread resume it
		}
	}

	void threadLoop(std::stop_token st) {
		while (!st.stop_requested()) {
			// get next coro task from the queue:
			CoroPoolTask::CoroHdl coro;
			{
				std::unique_lock lock(corosMx);
				if (!corosCV.wait(lock, st, [&] {
												return !coros.empty();
											})) {
					return; // stop requested
				}
				coro = coros.back();
				coros.pop_back();
			}

			// resume it:
			coro.resume(); // RESUME

			// NOTE: The coro initially resumed on this thread might NOT be the coro finally called.
			// If a main coro awaits a sub coro, then the thread that finally resumed the sub coro
			// resumes the main coro as its continuation.
			// => After this resumption, this coro and SOME continuations MIGHT be done
			std::function<void(CoroPoolTask::CoroHdl)> destroyDone;
			destroyDone = [&destroyDone, this](auto hdl) {
							if (hdl && hdl.done()) {
								auto nextHdl = hdl.promise().contHdl;
								hdl.destroy(); // destroy handle done
								--numCoros; // adjust total number of coros
								destroyDone(nextHdl); // do it for all continuations done
							}
						};
		destroyDone(coro); // recursively destroy coroutines done
		numCoros.notify_all(); // wake up any waiting waitUntilNoCoros()
		// sleep a little to force another thread to be used next:
		std::this_thread::sleep_for(std::chrono::milliseconds{100});
		}
	}

	void waitUntilNoCoros() {
		int num = numCoros.load();
		while (num > 0) {
		numCoros.wait(num); // wait for notification that numCoros changed the value
		num = numCoros.load();
		}
	}
};

// CoroPoolTask awaiter for: co_await task()
// - queues the new coro in the pool
// - sets the calling coro as continuation
void CoroPoolTask::CoAwaitAwaiter::await_suspend(CoroHdl awaitingHdl) noexcept
{
	newHdl.promise().contHdl = awaitingHdl;
	awaitingHdl.promise().poolPtr->runCoro(newHdl);
}
\end{cpp}

CoroPoolTask和CoroPool两个类需要谨慎地使用:

\begin{itemize}
\item
CoroPoolTask用于将协程初始化为要调用的任务。任务应该在池中调度,然后由池控制它,直到销毁。

\item
CoroPool实现了一个最小的线程池,可以运行调度的协程,还提供了一个非常小的API
\begin{itemize}
\item
阻塞,直到所有计划的协程完成

\item
关闭池(由其析构函数自动完成)
\end{itemize}
\end{itemize}

详细地看一下代码。

\mySamllsection{类CoroPoolTask}

类CoroPoolTask提供了一个协程接口来运行具有以下特性的任务:

\begin{itemize}
\item
每个协程都知道有一个指向线程池的指针,这个指针在线程池控制协程时初始化:

\begin{cpp}
class [[nodiscard]] CoroPoolTask
{
	...
	struct promise_type {
		CoroPool* poolPtr = nullptr; // if not null, lifetime is controlled by pool
		...
	};
	...
};
\end{cpp}

\item
每个协程都有一个可选持续的成员:

\begin{cpp}
class [[nodiscard]] CoroPoolTask
{
	...
	struct promise_type {
		...
		CoroHdl contHdl = nullptr; // coro that awaits this coro
		...
		auto final_suspend() const noexcept {
			struct FinalAwaiter {
				bool await_ready() const noexcept { return false; }
				std::coroutine_handle<> await_suspend(CoroHdl h) noexcept {
					if (h.promise().contHdl) {
						return h.promise().contHdl; // resume continuation
					}
					else {
						return std::noop_coroutine(); // no continuation
					}
				}
				void await_resume() noexcept {}
			};
			return FinalAwaiter{}; // AFTER suspended, resume continuation if there is one
		}
	};
	...
};
\end{cpp}

\item
每个协程都有一个co\_await()运算符,则co\_await任务()使用一个特殊的awaiter,因此co\_await在将传递的协程调度到池中,并以当前协程作为可持续成员后挂起当前协程。
\end{itemize}

下面是实现的操作符co\_await()的工作原理:

\begin{cpp}
class [[nodiscard]] CoroPoolTask
{
	...
	struct CoAwaitAwaiter {
		CoroHdl newHdl;
		bool await_ready() const noexcept { return false; }
		void await_suspend(CoroHdl awaitingHdl) noexcept; // see below
		void await_resume() noexcept {}
	};
	auto operator co_await() noexcept {
		return CoAwaitAwaiter{std::exchange(hdl, nullptr)}; // pool takes ownership of hdl
	}
};

void CoroPoolTask::CoAwaitAwaiter::await_suspend(CoroHdl awaitingHdl) noexcept
{
	newHdl.promise().contHdl = awaitingHdl;
	awaitingHdl.promise().poolPtr->runCoro(newHdl);
}
\end{cpp}

当用co\_await调用print()时:

\begin{cpp}
CoroPoolTask runAsync(std::string id)
{
	...
	co_await print( ... );
	...
}
\end{cpp}

发生以下情况:

\begin{itemize}
\item
print()作为协程调用并初始化。

\item
对于返回的CoroPoolTask类型的协程接口,调用操作符co\_await()。

\item
操作符co\_await()接受用于初始化CoAwaitAwaiter类型的awaiter的句柄,所以新的协程print()将成为该await而的成员newwhdl。

\item
std::exchange()可确保在初始化的协程接口中,句柄HDL变为nullptr,这样析构函数就不会调用destroy()。

\item
runAsync()使用由操作符co\_await()初始化的awaiter挂起,该操作符以协同程序runAsync()的句柄作为参数调用await\_suspend()。

\item
await\_suspend()中,将传递的挂起的runAsync()作为持续存储,并在池中调度newHdl(print()),以便由其中一个线程恢复。
\end{itemize}

\mySamllsection{类CoroPool}

CoroPool通过结合本文和本书其他部分介绍的一些特性来实现其最小线程池。

首先让看一下其成员:

\begin{itemize}
\item
与其他线程池一样,这个类有一个使用新线程类std::jthread的线程成员,这样我们就不必处理异常,可以发出停止信号:

\begin{cpp}
std::list<std::jthread> threads; // list of threads
\end{cpp}

因为在初始化之后使用线程的唯一其他时刻是在shutdown()期间,所以不需要同步,其首先会连接所有线程(信号停止之后)。不支持复制和移动池(简单起见,可以很容易地提供移动支持)。

为了获得更好的性能,析构函数在开始join()之前首先请求所有线程停止。

\item
然后,有一个具有相应同步成员的协程队列要处理:

\begin{cpp}
std::list<CoroPoolTask::CoroHdl> coros; // queue of scheduled coros
std::mutex corosMx;
std::condition_variable_any corosCV;
\end{cpp}

\item
为了避免过早停止池,池还跟踪池拥有的协程总数:

\begin{cpp}
std::atomic<int> numCoros = 0; // counter for all coros owned by the pool
\end{cpp}

并提供成员函数waitUntilNoCoros()。
\end{itemize}

成员函数runCoro()是调度协程恢复的关键函数,接受接口CoroPoolTask的协程句柄作为参数。有两种方法可以调度协程接口本身:

\begin{itemize}
\item
调用runTask()

\item
使用co\_await task()
\end{itemize}

这两种方法都将协程句柄移动到池中,这样池就有责任在不再使用该句柄时destroy()该句柄。

实现正确的时刻调用destroy()(并调整协程的总数)并不容易正确和安全地完成。协程应该最终挂起,这就排除了在任务的final\_suspend()或final awaiter的await\_suspend()中执行此操作,所以跟踪和销毁工作如下所示:

\begin{itemize}
\item
每次将协程句柄传递给池时,将增加协程的数量:

\begin{cpp}
void runCoro(CoroPoolTask::CoroHdl coro) noexcept {
	++numCoros;
	...
}
\end{cpp}

\item
线程的恢复完成后,将检查协程和可能的延续是否已经完成。根据任务的最后一个等待器的await\_suspend(),协程帧会自动调用持续。

因此,在每次恢复完成后,会递归地迭代所有的可持续对象以销毁所有已完成的协程:

\begin{cpp}
std::function<void(CoroPoolTask::CoroHdl)> destroyDone;
destroyDone = [&destroyDone, this](auto hdl) {
					if (hdl && hdl.done()) {
						auto nextHdl = hdl.promise().contHdl;
						hdl.destroy(); // destroy handle done
						--numCoros; // adjust total number of coros
						destroyDone(nextHdl); // do it for all continuations done
					}
				};
destroyDone(coro); // recursively destroy coroutines done
\end{cpp}

因为Lambda是递归使用的,所以必须将其转发声明为std::function<>。

\item
最后,用原子类型的新线程同步特性唤醒所有等待的waitUntilNoCoros():

\begin{cpp}
numCoros.notify_all(); // wake up any waiting waitUntilNoCoros()
...
void waitUntilNoCoros() {
	int num = numCoros.load();
	while (num > 0) {
		numCoros.wait(num); // wait for notification that numCoros changed the value
		num = numCoros.load();
	}
}
\end{cpp}

\item
若池销毁了,也可在所有线程完成后销毁所有剩余的协程句柄。
\end{itemize}

\mySamllsection{同步等待异步协程}

实际的协程池会更复杂,需要使用额外的技巧来使代码更健壮和安全。

例如,池可以提供一种调度任务并等待其结束的方法,对于CoroPool来说,看起来如下所示:

\begin{cpp}
class CoroPool
{
	...
	void syncWait(CoroPoolTask&& task) {
		std::binary_semaphore taskDone{0};
		auto makeWaitingTask = [&]() -> CoroPoolTask {
			co_await task;
			struct SignalDone {
				std::binary_semaphore& taskDoneRef;
				bool await_ready() { return false; }
				bool await_suspend(std::coroutine_handle<>) {
					taskDoneRef.release(); // signal task is done
					return false; // do not suspend at all
				}
				void await_resume() { }
			};
			co_await SignalDone{taskDone};
		};
		runTask(makeWaitingTask());
		taskDone.acquire();
	}
};
\end{cpp}

这里我们使用一个二值信号量,就可以发出传递任务结束的信号,并在另一个线程中等待它。

必须注意信号的确切内容,信号表明已经落后于执行了co\_await调用的任务,所以也可以在await\_ready()中发出信号:

\begin{cpp}
bool await_ready() {
	taskDoneRef.release(); // signal task is done
	return true; // do not suspend at all
}
\end{cpp}

通常,必须考虑在await\_ready()中,协程尚未挂起。因此,检查是否为done(),甚至destroy()的信号都会导致致命的运行时错误。因为在这里可以使用并发代码,所以必须确保信号不会间接引起相应的调用。

\mySubsubsection{15.9.3}{C++20之后的库将提供什么特性}

在协程章节的这一节中,所看到的只是一个非常简单的代码示例,不够健壮、线程安全、完整或灵活,无法为协程的各种典型用例提供通用解决方案。

C++标准委员会正致力于在未来的库中提供更好的解决方案。

对于一个能够以安全和灵活的方式并发运行协程的程序,至少需要以下组件:

\begin{itemize}
\item
一种任务类型,允许将协程链接在一起

\item
允许启动多个独立运行的协程,并在稍后汇入

\item
比如syncWait(),允许同步函数阻塞等待异步函数

\item
允许多个协程在较少数量的线程上复用
\end{itemize}

类CoroPool将后三个方面或多或少地结合在一起,但更灵活的方法是使用可以以各种方式组合的单个构建块。此外,良好的线程安全性需要更好的设计和实现技术。

我们仍在学习和讨论如何做到最好。因此,即使在C++23中,也可能只有最低限度的支持(若有的话)。



