

\mySubsubsection{20.3.1}{remove\_cvref\_t<>}

\mySamllsection{std::remove\_cvref\_t<T>}

生成类型T,而不是引用、顶层const或volatile。

表达式为

\begin{cpp}
std::remove_cvref_t<T>
\end{cpp}

等价于:

\begin{cpp}
std::remove_cv_t<remove_reference_t<T>>
\end{cpp}

例如:

\begin{cpp}
std::remove_cvref_t<const std::string&> // std::string
std::remove_cvref_t<const char* const> // const char*
\end{cpp}

\mySubsubsection{20.3.2}{unwrap\_reference<> and unwrap\_ref\_decay\_t}

\mySamllsection{std::unwrap\_reference\_t<T>}

若T是std::reference\_wrapper<>(使用std::ref()或std::cref()创建),则返回T的包装类型,否则为T。

\mySamllsection{std::unwrap\_ref\_decay\_t<T>}

若T是std::reference\_wrapper<>(使用std::ref()或std::cref()创建),则返回已封装的T类型,否则返回已封装的T类型。

例如:

\begin{cpp}
std::unwrap_reference_t<decltype(std::ref(s))> // std::string&
std::unwrap_reference_t<decltype(std::cref(s))> // const std::string&
std::unwrap_reference_t<decltype(s)> // std::string
std::unwrap_reference_t<decltype(s)&> // std::string&
std::unwrap_reference_t<int[4]> // int[4]

std::unwrap_ref_decay_t<decltype(std::ref(s))> // std::string&
std::unwrap_ref_decay_t<decltype(std::cref(s))> // const std::string&
std::unwrap_ref_decay_t<decltype(s)> // std::string
std::unwrap_ref_decay_t<decltype(s)&> // std::string
std::unwrap_ref_decay_t<int[4]> // int*
\end{cpp}


\mySubsubsection{20.3.3}{common\_reference<>\_t}

\mySamllsection{std::common\_reference\_t<T...>}

生成所有类型T…的通用类型(若有的话),可以给它赋值。因此,给定类型T1、T2和T3, 特性生成的类型可以同时赋值这三种类型。理想情况下,其是引用类型。但弱涉及到类型转换并创建临时对象,则为值类型。

例如:

\begin{cpp}
std::common_reference_t<int&, int> // int
std::common_reference_t<int&, int&> // int&
std::common_reference_t<int&, int&&> // const int&
std::common_reference_t<int&&, int&&> // int&&
std::common_reference_t<int&, double> // double
std::common_reference_t<int&, double&&> // double
std::common_reference_t<char*, std::string, std::string_view> // std::string_view
std::common_reference_t<char, std::string> // ERROR
\end{cpp}

\mySubsubsection{20.3.4}{type\_identity\_t<>}

\mySamllsection{std::type\_identity\_t<T>}

只生成T类型。

这个类型特性有大量的用例:

\begin{itemize}
\item 
可以禁用使用参数来推断模板参数。例如:

\begin{cpp}
template<typename T>
void insert(std::vector<T>& coll, const std::type_identity_t<T>& value)
{
	coll.push_back(value);
}

std::vector<double> coll;
...
insert(coll, 42); // OK: type of 42 not used to deduce type T
\end{cpp}

若只使用const T\&声明形参值,则编译器会引发错误,其会为类型T推断出两种不同的类型。

\item 
可以使用它作为构建块来定义生成类型的类型特征。例如,可以简单地定义一个类型特性来删除常量化:[可参见CppCon 2014 (\url{http: //www.youtube.com/watch?v=Am2is2QCvxY})上Walter E. Brown的演讲现代模板元编程]。

\begin{cpp}
template<typename T>
struct remove_const : std::type_identity<T> {
};

template<typename T>
struct remove_const<const T> : std::type_identity<T> {
};
\end{cpp}
\end{itemize}




