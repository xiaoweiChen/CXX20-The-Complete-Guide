

C++20提供了一种将并发输出同步到流的新机制。

\mySubsubsection{13.4.1}{添加同步输出流的动机}

若多个线程并发地写一个流,输出通常必须同步:

\begin{itemize}
\item 
通常,流的并发输出会导致未定义的行为(这是数据竞争,指具有未定义行为的竞争条件)。

\item 
支持并发输出到标准流(如std::cout),但结果不是很有用,因为来自不同线程的字符可能以任意顺序混合在一起。
\end{itemize}

例如,考虑下面的程序:

\filename{lib/atomicticket.cpp}

\begin{cpp}
#include <iostream>
#include <cmath>
#include <thread>

void squareRoots(int num)
{
	for (int i = 0; i < num ; ++i) {
		std::cout << "squareroot of " << i << " is "
		<< std::sqrt(i) << '\n';
	}
}

int main()
{
	std::jthread t1(squareRoots, 5);
	std::jthread t2(squareRoots, 5);
	std::jthread t3(squareRoots, 5);
}	
\end{cpp}

三个线程并发地写std::cout。这是有效的,因为使用了标准输出流。

然而,输出可能是这样的:

\begin{shell}
squareroot of squareroot of 0 is 0 is 0
0squareroot of squareroot of
01squareroot of is is 101 is

1squareroot of squareroot of
12squareroot of is is 21 is 1.41421

1.41421squareroot of squareroot of
23squareroot of is is 31.41421 is 1.73205
1.73205squareroot of squareroot of
34squareroot of is is 41.73205 is 2

2squareroot of
4 is 2
\end{shell}

\mySubsubsection{13.4.2}{使用同步输出流}

通过使用同步输出流,现在可以将多个线程的并发输出同步到同一个流,只需要使用用相应的输出流初始化的std::osyncstream。例如:

\filename{lib/atomicticket.cpp}

\begin{cpp}
#include <iostream>
#include <cmath>
#include <thread>
#include <syncstream>

void squareRoots(int num)
{
	for (int i = 0; i < num ; ++i) {
		std::osyncstream coutSync{std::cout};
		coutSync << "squareroot of " << i << " is "
		<< std::sqrt(i) << '\n';
	}
}

int main()
{
	std::jthread t1(squareRoots, 5);
	std::jthread t2(squareRoots, 5);
	std::jthread t3(squareRoots, 5);
}
\end{cpp}

这里,同步输出缓冲区将输出与其他输出同步到同步输出缓冲区,以便仅在调用同步输出缓冲区的析构函数时刷新输出。

结果,输出如下所示:

\begin{shell}
squareroot of 0 is 0
squareroot of 0 is 0
squareroot of 1 is 1
squareroot of 0 is 0
squareroot of 1 is 1
squareroot of 2 is 1.41421
squareroot of 1 is 1
squareroot of 2 is 1.41421
squareroot of 3 is 1.73205
squareroot of 2 is 1.41421
squareroot of 3 is 1.73205
squareroot of 4 is 2
squareroot of 3 is 1.73205
squareroot of 4 is 2
squareroot of 4 is 2
\end{shell}

这三个线程现在逐行写入std::cout,但确切顺序仍然是开放的。可以通过实现循环得到相同的结果,如下所示:

\begin{cpp}
for (int i = 0; i < num ; ++i) {
	std::osyncstream{std::cout} << "squareroot of " << i << " is "
								<< std::sqrt(i) << '\n';
}
\end{cpp}

请注意,' \verb|\|n '、std::endl和std::flush都不会写入输出,这里需要析构函数。若在循环外创建同步输出流,则在到达析构函数时将所有线程的输出一起打印。

然而,在调用析构函数之前,有一个新的操纵符用于写入输出:std::flush\_emit。因此,可以创建和初始化同步输出流,并逐行输出,如下所示:

\begin{cpp}
std::osyncstream coutSync{std::cout};
for (int i = 0; i < num ; ++i) {
	coutSync << "squareroot of " << i << " is "
			 << std::sqrt(i) << '\n' << std::flush_emit;
}
\end{cpp}

\mySubsubsection{13.4.3}{为文件使用同步输出流}

还可以为文件使用同步输出流。考虑下面的例子:

\filename{lib/syncfilestream.cpp}

\begin{cpp}
#include <fstream>
#include <cmath>
#include <thread>
#include <syncstream>

void squareRoots(std::ostream& strm, int num)
{
	std::osyncstream syncStrm{strm};
	for (int i = 0; i < num ; ++i) {
		syncStrm << "squareroot of " << i << " is "
				 << std::sqrt(i) << '\n' << std::flush_emit;
	}
}

int main()
{
	std::ofstream fs{"tmp.out"};
	std::jthread t1(squareRoots, std::ref(fs), 5);
	std::jthread t2(squareRoots, std::ref(fs), 5);
	std::jthread t3(squareRoots, std::ref(fs), 5);
}
\end{cpp}

该程序使用三个并发线程逐行写入同一个文件。

注意,每个线程都使用自己的同步输出流,但都必须使用相同的文件流。因此,如果每个线程都打开文件,程序将无法工作。

\mySubsubsection{13.4.4}{使用同步输出流作为输出流}

同步输出流是一个流。类std::osyncstream派生自std::ostream(准确地说:与通常的流类一样,类std::basic\_osyncstream<>派生自类std::basic\_ostream<>)。因此,还可以按如下方式实现上述程序:

\filename{lib/syncfilestream2.cpp}

\begin{cpp}
#include <fstream>
#include <cmath>
#include <thread>
#include <syncstream>

void squareRoots(std::ostream& strm, int num)
{
	for (int i = 0; i < num ; ++i) {
		strm << "squareroot of " << i << " is "
			 << std::sqrt(i) << '\n' << std::flush_emit;
	}
}

int main()
{
	std::ofstream fs{"tmp.out"};
	std::osyncstream syncStrm1{fs};
	std::jthread t1(squareRoots, std::ref(syncStrm1), 5);
	std::osyncstream syncStrm2{fs};
	std::jthread t2(squareRoots, std::ref(syncStrm2), 5);
	std::osyncstream syncStrm3{fs};
	std::jthread t3(squareRoots, std::ref(syncStrm3), 5);
}
\end{cpp}

操纵符std::flush\_emit通常是为输出流定义的,可以在这里使用。对于未同步的输出流,没有影响。

请注意,创建一个同步输出流并将其传递给所有三个线程将无法工作,因为多个线程将写入一个流:

\begin{cpp}
// undefined behavior (concurrent writes to the same stream):
std::osyncstream syncStrm{fs};
std::jthread t1(squareRoots, std::ref(syncStrm), 5);
std::jthread t2(squareRoots, std::ref(syncStrm), 5);
std::jthread t3(squareRoots, std::ref(syncStrm), 5);
\end{cpp}

\mySubsubsection{13.4.5}{实践同步输出流}

C++20起,经常使用同步输出流来“调试”带有print语句的多线程程序。只需要定义以下内容:

\begin{cpp}
#include <iostream> // for std::cout
#include <syncstream> // for std::osyncstream

inline auto syncOut(std::ostream& strm = std::cout) {
	return std::osyncstream{strm};
}
\end{cpp}

有了这个定义,可只使用syncOut(),而非std::cout,来确保并发输出是逐行写入的。例如:

\begin{cpp}
void foo(std::string name) {
	syncOut() << "calling foo(" << name
			  << ") in thread " << std::this_thread::get:id() << '\n';
	...
}
\end{cpp}

我们在本书中使用它来可视化并发协程输出。

为了能够关闭这样的调试输出,我有时会这样做:

\begin{cpp}
#include <iostream>
#include <syncstream> // for std::osyncstream

constexpr bool debug = true; // switch to false to disable output

inline auto coutDebug() {
	if constexpr (debug) {
		return std::osyncstream{std::cout};
	}
	else {
		struct devnullbuf : public std::streambuf {
			int_type overflow (int_type c) { // basic output primitive
				return c; // - without any print statement
			}
		};
		static devnullbuf devnull;
		return std::ostream{&devnull};
	}
}
\end{cpp}















