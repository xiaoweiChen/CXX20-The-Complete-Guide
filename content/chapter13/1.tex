
两个新的类型为多线程同步异步计算/处理提供了新的机制:

\begin{itemize}
\item
锁存器可以进行一次性同步,线程可以等待多个任务完成。

\item
栅栏对多个线程进行重复同步,当其都完成当前/下一个处理时,必须进行响应。
\end{itemize}

\mySubsubsection{13.1.1}{锁存器}

锁存器是用于并发执行的一种新的同步机制,支持单次使用异步倒计时。从初始整数值开始,各种线程可以自动将该值计数到零。当计数器达到零时,等待此倒计时的所有线程继续运行。

考虑下面的例子:

\filename{lib/latch.cpp}

\begin{cpp}
#include <iostream>
#include <array>
#include <thread>
#include <latch>
using namespace std::literals; // for duration literals

void loopOver(char c) {
	// loop over printing the char c:
	for (int j = 0; j < c/2; ++j) {
		std::cout.put(c).flush();
		std::this_thread::sleep_for(100ms);
	}
}

int main()
{
	std::array tags{'.', '?', '8', '+', '-'}; // tags we have to perform a task for

	// initialize latch to react when all tasks are done:
	std::latch allDone{tags.size()}; // initialize countdown with number of tasks

	// start two threads dealing with every second tag:
	std::jthread t1{[tags, &allDone] {
			for (unsigned i = 0; i < tags.size(); i += 2) { // even indexes
				loopOver(tags[i]);
				// signal that the task is done:
				allDone.count_down(); // atomically decrement counter of latch
			}
			...
	}};
	std::jthread t2{[tags, &allDone] {
			for (unsigned i = 1; i < tags.size(); i += 2) { // odd indexes
				loopOver(tags[i]);
				// signal that the task is done:
				allDone.count_down(); // atomically decrement counter of latch
			}
			...
	}};
	...
	// wait until all tasks are done:
	std::cout << "\nwaiting until all tasks are done\n";
	allDone.wait(); // wait until counter of latch is zero
	std::cout << "\nall tasks done\n"; // note: threads might still run
	...
}
\end{cpp}

本例中,启动两个线程(使用std::jthread)来执行两个任务,每个任务处理数组标记的一个字符,所以标签的大小决定了任务的数量。主线程阻塞,直到所有任务完成。:

\begin{itemize}
\item
用标签/任务的数量初始化锁存器:

\begin{cpp}
std::latch allDone{tags.size()};
\end{cpp}

\item
每个任务完成后减少计数器:

\begin{cpp}
allDone.count_down();
\end{cpp}

\item
主线程等待直到所有任务完成(计数器为零):

\begin{cpp}
allDone.wait();
\end{cpp}
\end{itemize}

程序的输出可能如下所示:

\begin{shell}
?
waiting until all tasks are done
.??..??.?..?.??.?..??.?..?.??..?.??.?.?.?.?.8??88??88??8?8?8+8+88++8+8+8+88++
8+8+8+8+88++8+8+88++8+8-+---------------------
all tasks done
\end{shell}

主线程不应该假设任务完成后,所有线程都完成了。线程可能仍然会做其他事情,系统可能仍然会清理它们。要等到所有线程都完成,必须为两个线程调用join()。

还可以使用锁存器在特定点同步多个线程,然后继续,但每个锁存器只能执行一次此操作。这样做的一个应用是确保(尽可能地)多个线程一起启动它们的实际工作,即使启动和初始化线程可能需要一些时间。

看下下面的例子:

\filename{lib/latch.cpp}

\begin{cpp}
#include <iostream>
#include <array>
#include <vector>
#include <thread>
#include <latch>
using namespace std::literals; // for duration literals

int main()
{
	std::size_t numThreads = 10;
	// initialize latch to start the threads when all of them have been initialized:

	std::latch allReady = 10; // initialize countdown with number of threads

	// start numThreads threads:
	std::vector<std::jthread> threads;
	for (int i = 0; i < numThreads; ++i) {
		std::jthread t{[i, &allReady] {
				// initialize each thread (simulate to take some time):
				std::this_thread::sleep_for(100ms * i);
				...
				// synchronize threads so that all start together here:
				allReady.arrive_and_wait();
				// perform whatever the thread does
				// (loop printing its index):
				for (int j = 0; j < i + 5; ++j) {
					std::cout.put(static_cast<char>('0' + i)).flush();
					std::this_thread::sleep_for(50ms);
				}
		}};
		threads.push_back(std::move(t));
	}
	...
}
\end{cpp}

启动numThreads线程(使用std::jthread),这些线程需要一些时间来初始化和启动。为了启动它们的功能,这里使用一个锁存器来进行阻塞,直到所有启动的线程都初始化并启动为止:

\begin{itemize}
\item
用线程数量初始化锁存器:

\begin{cpp}
std::latch allReady{numThreads};
\end{cpp}

\item
每个线程递减计数器,直到所有线程初始化完成:

\begin{cpp}
allReady.arrive_and_wait(); // count_down() and wait()
\end{cpp}

\end{itemize}

程序的输出可能如下所示:

\begin{shell}
86753421098675342019901425376886735241907863524910768352491942538679453876945876957869786789899
\end{shell}

可以看到,10个线程(每个线程都打印其索引)或多或少是一起启动的。

若没有锁存器,输出可能如下所示:

\begin{shell}
00101021021321324132435243524365463547635746854768547968579685796587968769876987987987989898999
\end{shell}

这里,早启动的线程已经在运行,而之后的可能还没有启动。

\mySamllsection{锁存器的详情}

类std::latch在头文件<latch>中声明,“latch类对象的操作”表列出了std::latch的API。

% Please add the following required packages to your document preamble:
% \usepackage{longtable}
% Note: It may be necessary to compile the document several times to get a multi-page table to line up properly
\begin{longtable}[c]{|l|l|}
\hline
\textbf{操作}      & \textbf{效果}                              \\ \hline
\endfirsthead
%
\endhead
%
latch l\{counter\} & 创建一个锁存器,将计数器作为倒计时的起始值 \\ \hline
l.count\_down()    & 自动减少计数器(若不为0)                     \\ \hline
l.count\_down(val)      & 计数器按val自动递减     \\ \hline
l.wait()                & 阻塞直到锁存器的计数器为0   \\ \hline
l.try\_wait()           & 生成锁存器的计数器是否为0 \\ \hline
l.arrive\_and\_wait()   & 调用count\_down()和wait()               \\ \hline
l.arrive\_and wait(val) & 调用count\_down(val)和wait()            \\ \hline
max()              & 产生counter最大可能值的静态函数   \\ \hline
\end{longtable}

\begin{center}
表13.1 类latch对象的操作
\end{center}

注意,不能复制或移动(分配)锁存器。

将容器的大小(std::array除外)作为计数器的初始值是错误的。构造函数接受一个带符号的std::ptrdiff\_t,会得到以下结果:

\begin{cpp}
std::latch l1{10}; // OK
std::latch l2{10u}; // warnings may occur
std::vector<int> coll{ ... };
...
std::latch l3{coll.size()}; // ERROR
std::latch l4 = coll.size(); // ERROR
std::latch l5(coll.size()); // OK (no narrowing checked)
std::latch l6{int(coll.size())}; // OK
std::latch l7{ssize(coll)}; // OK (see std::ssize())
\end{cpp}

\mySubsubsection{13.1.2}{栅栏}

栅栏是用于并发执行的新的同步机制,允许多次同步多个异步任务。设置初始计数后,多个线程可以对其进行计数,并等待计数器达到零。与锁存器相比,当达到零时,将调用一个(可选的)回调,计数器将重新初始化为初始计数。

当多个线程一起重复计算/执行某些事情时,栅栏是有用的。当所有线程都完成了它们的任务时,可选回调可以处理结果或新状态,再异步计算/处理可以继续进行下一轮。

例如,重复使用多个线程来计算多个值的平方根:

\filename{lib/barrier.cpp}

\begin{cpp}
#include <iostream>
#include <format>
#include <vector>
#include <thread>
#include <cmath>
#include <barrier>
int main()
{
	// initialize and print a collection of floating-point values:
	std::vector values{1.0, 2.0, 3.0, 4.0, 5.0, 6.0, 7.0, 8.0};

	// define a lambda function that prints all values
	// - NOTE: has to be noexcept to be used as barrier callback
	auto printValues = [&values] () noexcept{
						for (auto val : values) {
							std::cout << std::format(" {:<7.5}", val);
						}
						std::cout << '\n';
					};
	// print initial values:
	printValues();

	// initialize a barrier that prints the values when all threads have done their computations:
	std::barrier allDone{int(values.size()), // initial value of the counter
						 printValues}; // callback to call whenever the counter is 0

	// initialize a thread for each value to compute its square root in a loop:
	std::vector<std::jthread> threads;
	for (std::size_t idx = 0; idx < values.size(); ++idx) {
		threads.push_back(std::jthread{[idx, &values, &allDone] {
				// repeatedly:
				for (int i = 0; i < 5; ++i) {
					// compute square root:
					values[idx] = std::sqrt(values[idx]);
					// and synchronize with other threads to print values:
					allDone.arrive_and_wait();
				}
		}});
	}
	...
}
\end{cpp}

声明了一个浮点值数组之后,定义一个函数来输出它们(使用std::format()来输出格式化的输出):

\begin{cpp}
// initialize and print a collection of floating-point values:
std::vector values{1.0, 2.0, 3.0, 4.0, 5.0, 6.0, 7.0, 8.0};

// define a lambda function that prints all values
// - NOTE: has to be noexcept to be used as barrier callback
auto printValues = [&values] () noexcept{
	for (auto val : values) {
		std::cout << std::format(" {:<7.5}", val);
	}
	std::cout << '\n';
};
\end{cpp}

回调必须用noexcept声明。

我们的目标是使用多个线程,以便每个线程处理一个值,所以会重复计算这些值的平方根。因此:

\begin{itemize}
\item
初始化栅栏allDone,以便在所有线程完成下一个计算时打印所有值:标签/任务的数量:

\begin{cpp}
std::barrier allDone{int(values.size()), // initial value of the counter
					 printValues}; // callback to call whenever the counter is 0
\end{cpp}

构造函数应该接受带符号整数值。否则,代码可能无法编译。

\item
循环中,每个线程在完成计算后减少计数器,并等待计数器为零(即所有其他线程也已经发出信号,表示已经完成):

\begin{cpp}
..
allDone.arrive_and_wait();
\end{cpp}

\item
当计数器达到零时,将调用回调函数来输出结果。

回调是由最终将计数器减为零的线程调用的,所以在循环中,回调由不同的线程调用。
\end{itemize}

程序的输出可能如下所示:

% Please add the following required packages to your document preamble:
% \usepackage{longtable}
% Note: It may be necessary to compile the document several times to get a multi-page table to line up properly
\begin{longtable}[c]{llllllll}
1 & 2      & 3      & 4      & 5      & 6      & 7      & 8      \\
\endfirsthead
%
\endhead
%
1 & 1.4142 & 1.7321 & 2      & 2.2361 & 2.4495 & 2.6458 & 2.8284 \\
1 & 1.1892 & 1.3161 & 1.4142 & 1.4953 & 1.5651 & 1.6266 & 1.6818 \\
1 & 1.0905 & 1.1472 & 1.1892 & 1.2228 & 1.251  & 1.2754 & 1.2968 \\
1 & 1.0443 & 1.0711 & 1.0905 & 1.1058 & 1.1185 & 1.1293 & 1.1388 \\
1 & 1.0219 & 1.0349 & 1.0443 & 1.0516 & 1.0576 & 1.0627 & 1.0671
\end{longtable}

栅栏的API还提供了从该机制中删除线程的函数,例如:为了避免死锁,在循环运行时发出停止信号。

启动的线程的代码可以这样写:

\begin{cpp}
// initialize a thread for each value to compute its square root in a loop:
std::vector<std::jthread> threads;
for (std::size_t idx = 0; idx < values.size(); ++idx) {
	threads.push_back(std::jthread{[idx, &values, &allDone] (std::stop_token st) {
									// repeatedly:
									while (!st.stop_requested()) {
										// compute square root:
										values[idx] = std::sqrt(values[idx]);
										// and synchronize with other threads to print values:
										allDone.arrive_and_wait();
									}
									// drop thread from barrier so that other threads not wait:
									allDone.arrive_and_drop();
							}});
}
\end{cpp}

每个线程调用的Lambda现在接受一个停止令牌,以便在请求停止时做出反应(显式地或通过调用线程的析构函数)。若主线程通知线程停止计算(例如,通过调用threads.clear()),重要的是每个线程会将自己从栅栏中删除:

\begin{cpp}
allDone.arrive_and_drop();
\end{cpp}

该调用对计数器进行计数,并确保该值的下一个初始值递减。下一轮(其他线程可能仍在运行)中,栅栏将不再等待丢弃的线程。

参见lib/barrierstop.cpp获得完整的示例。

\mySamllsection{栅栏的详情}

类std::barrier在头文件<barrier>中声明。“类barrier对象的操作”表列出了std::barrier的API。

% Please add the following required packages to your document preamble:
% \usepackage{longtable}
% Note: It may be necessary to compile the document several times to get a multi-page table to line up properly
\begin{longtable}[c]{|l|l|}
\hline
\textbf{操作} & \textbf{效果}                                                 \\ \hline
\endfirsthead
%
\endhead
%
barrier b\{num\}   & 为num异步任务创建栅栏                    \\ \hline
barrier b\{num, cb\}  & 为num异步任务和cb作为回调创建栅栏                                                \\ \hline
b.arrive()         & 将任务标记为已完成,并产生到达令牌              \\ \hline
b.arrive(val)      & 将所有任务标记为已完成,并产生到达令牌             \\ \hline
b.wait(arrivalToken)  & 阻塞,直到所有的任务都已经完成,并回调已调用(若有的话)                        \\ \hline
b.arrive\_and\_wait() & 将一个任务标记为完成并阻塞,直到所有任务都完成,并回调已调用(若有的话) \\ \hline
b.arrive\_and\_drop() & 将一个任务标记为已完成,并减少重复执行的任务数量                                \\ \hline
max()              & 生成num的最大可能值的静态函数 \\ \hline
\end{longtable}

\begin{center}
表13.2 类barrier对象的操作
\end{center}

std::barrier<>是一个类模板,其回调类型作为模板参数。通常,类型是通过类模板参数推导出来的:

\begin{cpp}
void callback() noexcept; // forward declaration
...
std::barrier b{6, callback}; // deduces std::barrier<decltype(callback)>
\end{cpp}

C++标准要求栅栏的回调保证不会抛出异常。为了便于移植,必须用noexcept声明函数或Lambda。若没有传递回调,则使用特定于实现的类型,表示没有效果的操作。

\begin{cpp}
l.arrive_and_wait();
\end{cpp}

等价于

\begin{cpp}
l.wait(l.arrive());
\end{cpp}

也就是说,arrive()函数会返回一个类型为std::barrier::arrival\_token的到达令牌,以确保barrier知道要等待哪个线程。否则,将无法正确处理arrive\_and\_drop()。

注意,不能复制或移动(分配)栅栏。

将容器的大小(std::array除外)作为计数器的初始值是错误的。构造函数接受一个带符号的std::ptrdiff\_t,会看到以下行为:

\begin{cpp}
std::barrier b1{10, cb}; // OK
std::barrier b2{10u, cb}; // warnings may occur

std::vector<int> coll{ ... };
...
std::barrier b3{coll.size(), cb}; // ERROR
std::barrier b4(coll.size(), cb); // OK (no narrowing checked)
std::barrier b5{int(coll.size()), cb}; // OK
std::barrier b6{std::ssize(coll), cb}; // OK (see std::ssize())
\end{cpp}

函数std::ssize()是在C++20中加入的。




