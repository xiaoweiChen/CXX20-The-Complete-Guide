
来看一些使用跨度的第一个例子。首先,我们必须讨论如何指定span中的元素数量。

\mySubsubsection{9.1.1}{固定和动态区段}

声明span时,可以在指定的固定数量的元素之间进行选择,也可以不设置元素数量,这样span引用的元素数量就可以改变。

具有指定固定数量元素的span称为具有固定区段的span,可以通过指定元素类型和大小作为模板参数来声明,或者通过带有迭代器和大小的数组(raw array或std::array<>)来初始化:

\begin{cpp}
int a5[5] = {1, 2, 3, 4, 5};
std::array arr5{1, 2, 3, 4, 5};
std::vector vec{1, 2, 3, 4, 5, 6, 7, 8};

std::span sp1 = a5; // span with fixed extent of 5 elements
std::span sp2{arr5}; // span with fixed extent of 5 elements
std::span<int, 5> sp3 = arr5; // span with fixed extent of 5 elements
std::span<int, 3> sp4{vec}; // span with fixed extent of 3 elements
std::span<int, 4> sp5{vec.data(), 4}; // span with fixed extent of 4 elements
std::span sp6 = sp1; // span with fixed extent of 5 elements
\end{cpp}

对于这样的span,成员函数size()总是产生作为类型的一部分指定的大小。这里不能调用默认构造函数(除非区段为0)。

元素数量在其生命周期内不稳定的sapn称为具有动态区段的span。元素的数量取决于span所引用的元素的顺序,并且可能由于分配了一个新的基础范围而改变(没有其他方法可以改变元素的数量)。例如:

\begin{cpp}
int a5[5] = {1, 2, 3, 4, 5};
std::array arr5{1, 2, 3, 4, 5};
std::vector vec{1, 2, 3, 4, 5, 6, 7, 8};

std::span<int> sp1; // span with dynamic extent (initially 0 elements)
std::span sp2{a5, 3}; // span with dynamic extent (initially 3 elements)
std::span<int> sp3{arr5}; // span with dynamic extent (initially 5 elements)
std::span sp4{vec}; // span with dynamic extent (initially 8 elements)
std::span sp5{arr5.data(), 3}; // span with dynamic extent (initially 3 elements)
std::span sp6{a5+1, 3}; // span with dynamic extent (initially 3 elements)
\end{cpp}

注意,需要由开发者保证span指向一个具有足够元素的有效范围。

对于这两种情况,让我们看一下完整的例子。

\mySubsubsection{9.1.2}{使用实例使用带动态区段的span}

下面是第一个使用动态区段span的例子:

\filename{lib/spandyn.cpp}

\begin{cpp}
#include <iostream>
#include <string>
#include <vector>
#include <ranges>
#include <algorithm>
#include <span>

template<typename T, std::size_t Sz>
void printSpan(std::span<T, Sz> sp)
{
	for (const auto& elem : sp) {
		std::cout << '"' << elem << "\" ";
	}
	std::cout << '\n';
}

int main()
{
	std::vector<std::string> vec{"New York", "Tokyo", "Rio", "Berlin", "Sydney"};
	
	// define view to first 3 elements:
	std::span<const std::string> sp{vec.data(), 3};
	std::cout << "first 3: ";
	printSpan(sp);
	
	// sort elements in the referenced vector:
	std::ranges::sort(vec);
	std::cout << "first 3 after sort(): ";
	printSpan(sp);
	
	// insert a new element:
	// - must reassign the internal array of the vector if it reallocated new memory
	auto oldCapa = vec.capacity();
	vec.push_back("Cairo");
	if (oldCapa != vec.capacity()) {
		sp = std::span{vec.data(), 3};
	}
	std::cout << "first 3 after push_back(): ";
	printSpan(sp);

	// let span refer to the vector as a whole:
	sp = vec;
	std::cout << "all: ";
	printSpan(sp);

	// let span refer to the last five elements:
	sp = std::span{vec.end()-5, vec.end()};
	std::cout << "last 5: ";
	printSpan(sp);
	
	// let span refer to the last four elements:
	sp = std::span{vec}.last(4);
	std::cout << "last 4: ";
	printSpan(sp);
}
\end{cpp}

程序输出如下:

\begin{shell}
first 3:              "New York" "Tokyo" "Rio"
first 3 after sort(): "Berlin" "New York" "Rio"
first 3 after push_back(): "Berlin" "New York" "Rio"
all:   "Berlin" "New York" "Rio" "Sydney" "Tokyo" "Cairo"
last 5: "New York" "Rio" "Sydney" "Tokyo" "Cairo"
last 4: "Rio" "Sydney" "Tokyo" "Cairo"
\end{shell}

让我们一步一步地来分析这个例子。

\mySamllsection{声明span}

在main()中,首先用vector的前三个元素初始化了一个由三个常量字符串组成的span:

\begin{cpp}
std::vector<std::string> vec{"New York", "Rio", "Tokyo", "Berlin", "Sydney"};

std::span<const std::string> sp{vec.data(), 3};
\end{cpp}

初始化时,传递序列的开头和元素的数目。本例中,引用vec的前三个元素。

关于这个声明,有很多事情需要注意:

\begin{itemize}
\item
由开发者来确保元素的数量与跨度的范围相匹配,并且元素是有效的。若vector没有足够的元素,则行为未定义:

\begin{cpp}
std::span<const std::string> sp{vec.begin(), 10}; // undefined behavior
\end{cpp}

\item
指定元素为const std::string时,就不能通过span来修改它们了。注意,将span本身声明为const并不提供对引用元素的只读访问(通常对于视图,const不会传播):

\begin{cpp}
std::span<const std::string> sp1{vec.begin(), 3}; // elements cannot be modified
const std::span<std::string> sp2{vec.begin(), 3}; // elements can be modified
\end{cpp}

\item
为span使用不同于引用元素的元素类型,看起来就像可以为转换为基础元素类型的span使用的类型一样。然而,事实并非如此,只能添加const这样的限定符:

\begin{cpp}
std::vector<int> vec{ ... };
std::span<long> sp{vec.data(), 3}; // ERROR
\end{cpp}
\end{itemize}

\mySamllsection{传递和输出span}

接下来,输出span,并将其传递给一个通用的print函数:

\begin{cpp}
printSpan(sp);
\end{cpp}

print函数可以处理任何span(只要为元素定义了输出操作符):

\begin{cpp}
template<typename T, std::size_t Sz>
void printSpan(std::span<T, Sz> sp)
{
	for (const auto& elem : sp) {
		std::cout << '"' << elem << "\" ";
	}
	std::cout << '\n';
}
\end{cpp}

读者们可能会感到惊讶,可以调用函数模板printSpan<>(),即使其有一个非类型模板参数表示span的大小。其之所以有效,是因为std::span<T>是快捷方式——具有伪长度std::dynamic\_extent的span:

\begin{cpp}
std::span<int> sp; // equivalent to std::span<int, std::dynamic_extent>
\end{cpp}

实际上,类模板std::span<>是这样声明的:

\begin{cpp}
namespace std {
	template<typename ElementType, size_t Extent = dynamic_extent>
	class span {
		...
	};
}
\end{cpp}

这允许开发者提供像printSpan<>()这样的通用代码,其既适用于固定范围的跨度,也适用于动态区段的span。当使用具有固定区段的span调用printSpan<>()时,区段可作为模板参数传递:

\begin{cpp}
std::span<int, 5> sp{ ... };

printSpan(sp); // calls printSpan<int, 5>(sp)
\end{cpp}

span是按值传递的。这是传递span的推荐方法,因为复制它们的成本很低,并在内部,span只是一个指针和一个长度信息。

print函数内部,使用基于范围的for循环遍历span的元素。因为span提供了begin()和end()的迭代器支持,所以这是可能的。

但要注意:无论通过值传递还是通过const引用传递span,只要没有将元素声明为const,在函数内部仍可以修改元素。这就是为什么将span的元素声明为const是有意义的。

\mySamllsection{处理引用语义}

接下来,对容器所引用的元素进行排序(这里使用新的std::ranges::sort(),将容器作为一个整体):

\begin{cpp}
std::ranges::sort(vec);
\end{cpp}

由于span具有引用语义,因此这种排序也会影响span的元素,其结果是span现在引用不同的值。

若没有const元素的span,也可以调用sort()传递span。

引用语义在使用span时必须非常小心,在下个示例中就会看到相关语句。这里,我们将一个新元素插入vector中,该vector中保存了跨度所引用的元素。由于span的引用语义,必须非常小心,若vector分配新的内存,会使所有迭代器和指向其元素的指针无效。因此,重新分配也会使引用vector元素的span失效。span指向的是不再存在的元素。

出于这个原因,需要在插入前后都要仔细检查容量(分配内存的最大元素数量)。若容量发生变化,则重新初始化span,使其指向新内存中的前三个元素:

\begin{cpp}
auto oldCapa = vec.capacity();
vec.push_back("Cairo");
if (oldCapa != vec.capacity()) {
	sp = std::span{vec.data(), 3};
}
\end{cpp}

只能执行这种重新初始化,因为span本身不为const。

\mySamllsection{将容器分配给span}

接下来,我们将vector作为一个整体赋值给span,并将其进行输出:

\begin{cpp}
std::span<const std::string> sp{vec.begin(), 3};
...
sp = vec;
\end{cpp}

这里,可以看到对具有动态区段span的赋值可以改变元素的数量。只要容器允许使用成员函数data()访问这些元素,则span可以使用任何类型的容器或范围来保存连续内存中的元素。

但由于模板类型推导的限制,不能将这样的容器传递给期望使用span的函数。必须明确指定要将vector转换为span:

\begin{cpp}
printSpan(vec); // ERROR: template type deduction doesn’t work here
printSpan(std::span{vec}); // OK
\end{cpp}

\mySamllsection{分配不同的子序列}

通常,span的赋值操作符接受另一个元素序列。

下面的示例使用这种方式来引用vector中的最后三个元素:

\begin{cpp}
std::span<const std::string> sp{vec.data(), 3};
...
// assign view to last five elements:
sp = std::span{vec.end()-5, vec.end()};
\end{cpp}

这里,还可以看到,可以使用两个迭代器指定引用的序列,将序列的开始和结束定义为半开范围(包括开始的值,不包括结束的值)。要求满足std::size\_sentinel\_for的概念,以便构造函数可以计算差值。

如下面的语句所示,最后n个元素也可以使用span的成员函数来赋值:

\begin{cpp}
std::vector<std::string> vec{"New York", "Tokyo", "Rio", "Berlin", "Sydney"};
std::span<const std::string> sp{vec.data(), 3};
...
// assign view to last four elements:
sp = std::span{vec}.last(4);
\end{cpp}

span是唯一提供在范围中间或末尾生成元素序列的视图。

只要元素类型合适,就可以传递任何其他类型的元素序列。例如:

\begin{cpp}
std::vector<std::string> vec{"New York", "Tokyo", "Rio", "Berlin", "Sydney"};
std::span<const std::string> sp{vec.begin(), 3};
...
std::array<std::string, 3> arr{"Tick", "Trick", "Track"};
sp = arr; // OK
\end{cpp}

但请注意,span不支持元素类型的隐式类型转换(除了添加const)。例如,无法编译以下代码:

\begin{cpp}
std::span<const std::string> sp{vec.begin(), 3};
...
std::array arr{"Tick", "Trick", "Track"}; // deduces std::array<const char*, 3>
sp = arr; // ERROR: different element types
\end{cpp}

\mySubsubsection{9.1.3}{使用具有非const元素的实例}

初始化span时,可以使用类模板实参推导,从而推导出元素的类型(和范围):

\begin{cpp}
std::span sp{vec.begin(), 3}; // deduces: std::span<std::string>
\end{cpp}

然后,span会声明元素的类型,使其具有基础范围的元素类型,甚至可以修改基础范围的值,前提是基础范围没有将其元素声明为const。

此特性可用于允许span在一个语句中修改范围的元素。例如,可以对元素的一个子集进行排序,如下所示:

\filename{lib/spanview.cpp}

\begin{cpp}
#include <iostream>
#include <string>
#include <vector>
#include <ranges>
#include <algorithm>
#include <span>

void print(std::ranges::input_range auto&& coll)
{
	for (const auto& elem : coll) {
		std::cout << '"' << elem << "\" ";
	}
	std::cout << '\n';
}

int main()
{
	std::vector<std::string> vec{"New York", "Tokyo", "Rio", "Berlin", "Sydney"};
	print(vec);
	
	// sort the three elements in the middle:
	std::ranges::sort(std::span{vec}.subspan(1, 3));
	print(vec);
	
	// print last three elements:
	print(std::span{vec}.last(3));
}
\end{cpp}

这里,创建临时span来对vector vec中的元素子集进行排序,并打印vector的最后三个元素。

该程序有以下输出:

\begin{shell}
"New York" "Tokyo" "Rio" "Berlin" "Sydney"
"New York" "Berlin" "Rio" "Tokyo" "Sydney"
"Rio" "Tokyo" "Sydney"
\end{shell}

span是视图。要处理一个范围的前n个元素,也可以使用范围工厂std::views:: counts(),若对一个连续内存中有元素的范围进行迭代器使用,会创建一个具有动态区段的span:

\begin{cpp}
std::vector<int> vec{1, 2, 3, 4, 5, 6, 7, 8, 9};

auto v = std::views::counted(vec.begin()+1, 3); // span with 2nd to 4th elem of vec
\end{cpp}

\mySubsubsection{9.1.4}{使用实例使用固定区段的span}

作为具有固定区段的span的第一个示例,让我们转换前面的示例,声明具有固定区段的span:

\filename{lib/spanfix.cpp}

\begin{cpp}
#include <iostream>
#include <string>
#include <vector>
#include <ranges>
#include <algorithm>
#include <span>

template<typename T, std::size_t Sz>
void printSpan(std::span<T, Sz> sp)
{
	for (const auto& elem : sp) {
		std::cout << '"' << elem << "\" ";
	}
	std::cout << '\n';
}

int main()
{
	std::vector<std::string> vec{"New York", "Tokyo", "Rio", "Berlin", "Sydney"};
	
	// define view to first 3 elements:
	std::span<const std::string, 3> sp3{vec.data(), 3};
	std::cout << "first 3: ";
	printSpan(sp3);
	
	// sort referenced elements:
	std::ranges::sort(vec);
	std::cout << "first 3 after sort(): ";
	printSpan(sp3);
	
	// insert a new element:
	// - must reassign the internal array of the vector if it reallocated new memory
	auto oldCapa = vec.capacity();
	vec.push_back("Cairo");
	if (oldCapa != vec.capacity()) {
		sp3 = std::span<std::string, 3>{vec.data(), 3};
	}
	std::cout << "first 3: ";
	printSpan(sp3);
	
	// let span refer to the last three elements:
	sp3 = std::span<const std::string, 3>{vec.end()-3, vec.end()};
	std::cout << "last 3: ";
	printSpan(sp3);
	
	// let span refer to the last three elements:
	sp3 = std::span{vec}.last<3>();
	std::cout << "last 3: ";
	printSpan(sp3);
}
\end{cpp}

程序输出如下所示:

\begin{shell}
first 3:              "New York" "Tokyo" "Rio"
first 3 after sort(): "Berlin" "New York" "Rio"
first 3: "Berlin" "New York" "Rio"
last 3: "Sydney" "Tokyo" "Cairo"
last 3: "Sydney" "Tokyo" "Cairo"
\end{shell}

让我们一步一步地回顾程序示例中值得注意的部分。

\mySamllsection{声明span}

这一次,首先初始化一个由三个固定长度的常量字符串组成的span:

\begin{cpp}
std::vector<std::string> vec{"New York", "Rio", "Tokyo", "Berlin", "Sydney"};

std::span<const std::string, 3> sp3{vec.data(), 3};
\end{cpp}

对于固定区段,可指定元素的类型和大小。同样,需要由开发者来保证元素的数量与span的区段相匹配。

若作为第二个参数传递的计数与范围不匹配,则行为未定义。

\begin{cpp}
std::span<const std::string, 3> sp3{vec.begin(), 4}; // undefined behavior
\end{cpp}

\mySamllsection{分配不同的子序列}

对于具有固定区段的span,只能分配具有相同数量元素的新范围。

因此,这一次,只赋值具有三个元素的span:

\begin{cpp}
std::span<const std::string, 3> sp3{vec.data(), 3};
...
sp3 = std::span<const std::string, 3>{vec.end()-3, vec.end()};
\end{cpp}

注意,以下代码将不可编译:

\begin{cpp}
std::span<const std::string, 3> sp3{vec.data(), 3};
...
sp3 = std::span{vec}.last(3); // ERROR
\end{cpp}

这样做的原因是,赋值右侧的表达式创建了一个具有动态区段的span。但通过使用last()和指定元素数量作为模板参数的语法,则得到一个具有相应固定区段的span:

\begin{cpp}
std::span<const std::string, 3> sp3{vec.data(), 3};
...
sp3 = std::span{vec}.last<3>(); // OK
\end{cpp}

仍可以使用类模板实参推导来为数组元素赋值,甚至可以直接赋值:

\begin{cpp}
std::span<const std::string, 3> sp3{vec.data(), 3};
...
std::array<std::string, 3> arr{"Tic", "Tac", "Toe"};
sp3 = std::span{arr}; // OK
sp3 = arr; // OK
\end{cpp}

\mySubsubsection{9.1.5}{固定与动态区段}

固定和动态区段各有好处。

指定固定的大小使编译器能够在运行时,甚至在编译时检测长度是否违规。例如,不能将元素数目错误的std::array<>赋值给区段固定的span:

\begin{cpp}
std::vector vec{1, 2, 3};
std::array arr{1, 2, 3, 4, 5, 6};
std::span<int, 3> sp3{vec};
std::span sp{vec};
sp3 = arr; // compile-time ERROR
sp = arr; // OK
\end{cpp}

具有固定区段的span还需要更少的内存,它们不需要具有实际大小的成员(大小是类型的一部分)。

使用动态span提供了更大的灵活性:

\begin{cpp}
std::span<int> sp; // OK
...
std::vector vec{1, 2, 3, 4, 5};
sp = vec; // OK (span has 5 elements)
sp = {vec.data()+1, 3}; // OK (span has 3 elements)
\end{cpp}








