
设计一个引用元素序列的类型并不容易。需要权衡很多方面,从而做出决定:

\begin{itemize}
\item
性能与安全

\item
const的正确性

\item
可能的隐式和显式类型转换

\item
支持类型的要求

\item
支持的API
\end{itemize}

先说明一下span不是什么:

\begin{itemize}
\item
span不是容器,可能具有容器的几个特点(例如,使用begin()和end()迭代元素的能力),但由于其引用语义已经存在几个问题:

\begin{itemize}
\item
若span是const,那元素也应该是const的吗?

\item
赋值的作用是什么:给引用的元素分配新的序列或者赋一个新的值?

\item
应该提供swap()吗?它能做什么?
\end{itemize}

\item
span不是指针(有大小),提供解引用操作符和->没有意义。
\end{itemize}

std::span类型是对元素序列的一个非常特定的引用。要正确使用这种类型,有必要具体了解一下。

Barry Revzin有一篇非常有用的博客文章,我强烈建议大家去阅读:\url{http://brevzin.github.io/c++/2018/12/03/span-best-span/}。

C++20还提供了其他方法来处理对(子)序列的引用,例如:子序列。也适用于不存储在连续内存中的序列,使用范围工厂std::views::counted(),可以让编译器决定哪种类型最适合由开始和大小定义的范围。

\mySubsubsection{9.3.1}{span生命周期的依赖性}

由于引用语义,只能在存在底层值序列时下遍历span,但迭代器并不与创建他们的span绑定生命周期。

span的迭代器不引用创建它们的span。相反,它们直接引用基础范围,所以span是一个租借范围。因此,即使span不再存在(元素序列必须存在),也可以使用迭代器。但当底层范围不再存在时,迭代器可悬空。

\mySubsubsection{9.3.2}{span的性能}

span的设计以最佳性能为目标。在内部,仅使用指向元素序列的原始指针,但原始指针希望元素按顺序存储在一个内存块中(否则,原始指针可以在元素来回跳转时计算元素的位置)。所以,span要求元素存储在连续内存中。

有了这个要求,span可以为所有类型的视图提供最佳性能。span不需要任何内存分配,也不附带任何间接内容,使用span的唯一开销是构造。编译时使用概念检查引用序列是否在连续内存中有其元素,初始化序列或赋值新序列时,迭代器必须满足std::consecuous\_iterator概念,容器或范围必须满足std::ranges::consecuous\_range和std::ranges::sized\_range两个概念。

因为span在内部只是一个指针和一个大小,所以复制成本非常低。出于这个原因,应该更倾向于按值传递span,而非通过const引用传递(然而,对于处理容器和视图的泛型函数来说,这是一个问题)。

\mySamllsection{类型擦除}

span使用指向内存的原始指针执行元素访问,span类型会擦除元素存储位置的信息。指向vector元素的span操作与指向数组元素的span操作具有相同的类型(前提是它们具有相同的区段):

\begin{cpp}
std::array arr{1, 2, 3, 4, 5};
std::vector vec{1, 2, 3, 4, 5};

std::span<int> vecSpanDyn{vec};
std::span<int> arrSpanDyn{arr};
std::same_as<decltype(arrSpanDyn), decltype(vecSpanDyn)> // true
\end{cpp}

但span的类模板参数推导从数组推导出固定区段,从vector推导出动态区段:

\begin{cpp}
std::array arr{1, 2, 3, 4, 5};
std::vector vec{1, 2, 3, 4, 5};
std::span arrSpan{arr}; // deduces std::span<int, 5>
std::span vecSpan{vec}; // deduces std::span<int>
std::span<int, 5> vecSpan5{vec};

std::same_as<decltype(arrSpan), decltype(vecSpan)> // false
std::same_as<decltype(arrSpan), decltype(vecSpan5)> // true
\end{cpp}

\mySamllsection{span与子范围}

对元素连续存储的要求是与子范围的主要区别,子范围也是C++20引入的。在内部,子范围仍然使用迭代器,因此可以引用所有类型的容器和范围,但这可能会导致更多的开销。

此外,span不需要对其引用的类型支持迭代器,可以传递data()成员函数以访问元素序列的类型。

\mySubsubsection{9.3.3}{span的const正确性}

span是具有引用语义的视图,其行为就像指针:若span是const,并不意味着该span所引用的元素也是const。

并且,可以对const span的元素进行写访问(前提是这些元素不是const):

\begin{cpp}
std::array a1{1, 2, 3, 4, 5, 6,7 ,8, 9, 10};
std::array a2{0, 8, 15};

const std::span<int> sp1{a1}; // span/view is const
std::span<const int> sp2{a1}; // elements are const

sp1[0] = 42; // OK
sp2[0] = 42; // ERROR

sp1 = a2; // ERROR
sp2 = a2; // OK
\end{cpp}

只要std::span<>的元素没有声明为const,即使对于const span,也会有一些操作提供对这些元素的写访问(遵循普通容器的规则):

\begin{itemize}
\item
operator[], first(), last()

\item
data()

\item
begin(), end(), rbegin(), rend()

\item
std::cbegin(), std::cend(), std::crbegin(), std::crend()

\item
std::ranges::cbegin(), std::ranges::cend(), std::ranges::crbegin(), std::ranges::crend()
\end{itemize}

是的,所有设计用来确保元素为const的c*函数,都可使用std::span替代。

例如:

\begin{cpp}
template<typename T>
void modifyElemsOfConstColl (const T& coll)
{
	coll[0] = {}; // OK for spans, ERROR for regular containers

	auto ptr = coll.data();
	*ptr = {}; // OK for spans, ERROR for regular containers

	for (auto pos = std::cbegin(coll); pos != std::cend(coll); ++pos) {
		*pos = {}; // OK for spans, ERROR for regular containers
	}
}

std::array arr{1, 2, 3, 4, 5, 6, 7 ,8, 9, 10};

modifyElemsOfConstColl(arr); // ERROR: elements are const
modifyElemsOfConstColl(std::span{arr}); // OOPS: compiles and modifies elements of a1
\end{cpp}

这里的问题不在于破坏了std::span;问题是,像std::cbegin()和std::ranges::cbegin()这样的函数破坏了具有引用语义的集合(例如视图)。

为了确保函数只接受不能以这种方式修改元素的序列,可以要求const容器的begin()返回一个指向const元素的迭代器:

\begin{cpp}
template<typename T>
void ensureReadOnlyElemAccess (const T& coll)
requires std::is_const_v<std::remove_reference_t<decltype(*coll.begin())>>
{
	...
}
\end{cpp}

std::cbegin()和std::ranges::cbegin()提供写访问,这是标准化C++20之后讨论的问题。提供cbegin()和cend()的全部意义在于确保元素在迭代时不能被修改。最初,span确实为const\_iterator、cbegin()和cend()类型提供了成员,以确保不能修改元素。但在C++20完成前的最后一刻,std::cbegin()仍然迭代可变元素(std::ranges::cbegin()也有同样的问题)。然而,不是修复std::cbegin()和std::ranges::cbegin(),而是删除了span中const迭代器的成员(参见\url{http://wg21.link/lwg3320}),这使得问题更加严重,因为在C++20中,现在没有简单的方法对span进行只读迭代,除非元素为const。std::ranges::cbegin()将在C++23中修复(参见\url{http://wg21.link/p2278}),但std::cbegin()仍然会遭到破坏(叹气)。

\mySubsubsection{9.3.4}{泛型代码中使用span作为参数}

如前所述,可以通过以下声明实现所有span的泛型函数:

\begin{cpp}
template<typename T, std::size_t Sz>
void printSpan(std::span<T, Sz> sp);
\end{cpp}

这甚至适用于具有动态范围的span,可使用特殊值std::dynamic\_extent作为大小。

实现中,可以按以下方式处理固定与动态区段的区别:

\filename{lib/spanprint.hpp}

\begin{cpp}
#ifndef SPANPRINT_HPP
#define SPANPRINT_HPP

#include <iostream>
#include <span>

template<typename T, std::size_t Sz>
void printSpan(std::span<T, Sz> sp)
{
	std::cout << '[' << sp.size() << " elems";
	if constexpr (Sz == std::dynamic_extent) {
		std::cout << " (dynamic)";
	}
	else {
		std::cout << " (fixed)";
	}
	std::cout << ':';
	for (const auto& elem : sp) {
		std::cout << ' ' << elem;
	}
	std::cout << "]\n";
	}
#endif // SPANPRINT_HPP
\end{cpp}

还可以考虑将元素类型声明为const:

\begin{cpp}
printSpan(vec); // ERROR: template type deduction doesn’t work
printSpan(std::span{vec}); // OK
printSpan<int, std::dynamic_extent>(vec); // OK (provided it is a vector of ints)
\end{cpp}

因此,不应将std::span<>用作处理存储在连续内存中元素序列泛型函数的词汇表类型。

出于性能考虑,可以这样做:

\begin{cpp}
template<typename E>
void processSpan(std::span<typename E>) {
	... // span specific implementation
}

template<typename T>
void print(const T& t) {
	if constexpr (std::ranges::contiguous_range<T> t) {
		processSpan<std::ranges::range_value_t<T>>(t);
	}
	else {
		... // generic implementations for all containers/ranges
	}
}
\end{cpp}

\mySamllsection{使用span作为范围和视图}

span是视图,所以满足std::ranges::view[最初的C++20标准确实要求视图必须有一个默认构造函数,而固定的span则不是这样,但这一要求后来删除了\url{http://wg21.link/P2325R3}。],span可以用于范围和视图的所有算法和函数中。在讨论所有视图细节的那一章中,列出了span的特定于视图的属性。

span的一个属性是租借范围,所以迭代器的生命周期不依赖于span,所以可以在生成迭代器的算法中使用临时span作为范围:

\begin{cpp}
std::vector<int> coll{25, 42, 2, 0, 122, 5, 7};
auto pos1 = std::ranges::find(std::span{coll.data(), 3}, 42); // OK
std::cout << *pos1 << '\n';
\end{cpp}

但若span是引用的临时对象,则会出现错误。即使返回已销毁的临时对象的迭代器,以下代码也可以编译通过:

\begin{cpp}
auto pos2 = std::ranges::find(std::span{getData().data(), 3}, 42);
std::cout << *pos2 << '\n'; // runtime ERROR
\end{cpp}











