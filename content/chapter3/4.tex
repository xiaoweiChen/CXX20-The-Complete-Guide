
概念可同时检查语法和语义约束:

\begin{itemize}
\item
\textbf{语法约束}在编译时,可以检查是否满足某些功能需求(“是否支持特定的操作?”或“特定操作是否产生特定类型?”)。

\item
\textbf{语义约束}满足了某些只能在运行时检查的需求(“操作是否具有相同的效果?”或“对特定值执行相同的操作是否总是产生相同的结果?”)。
\end{itemize}

有时,概念允许开发者通过接口来指定是否满足语义约束,从而将语义约束转换为语法约束。

\mySubsubsection{3.4.1}{语义约束的例子}

来看一些语义约束的例子。

\mySamllsection{std::ranges::sized\_range}

语义约束的第一个例子是概念std::ranges::sized\_range,其保证可以在常量时间内计算一个范围内的元素数量(通过成员函数size()或计算开始和结束之间的差值)。

若范围类型提供size()(作为成员函数或独立函数),则默认情况下满足此概念。要从这个概念中退出(例如,迭代所有元素以产生结果),可以将std::disable\_size\_range<Rg>设置为true:

\begin{cpp}
class MyCont {
	...
	std::size_t size() const; // assume this is expensive, so that this is not a sized range
};
// opt out from concept std::ranges::sized_range:
constexpr bool std::ranges::disable_sized_range<MyCont> = true;
\end{cpp}

\mySamllsection{std::ranges::range与std::ranges::view}

语义约束的一个示例是概念std::ranges::view。除了一些语法约束外,还保证移动构造函数/赋值、复制构造函数/赋值(如可用)和析构函数具有恒定的复杂性(所花费的时间不取决于元素的数量)。

实现者可以通过公开地从std::ranges::view\_base或std::ranges::view\_interface<>派生,或者通过将模板特化std::ranges::enable\_view<Rg>设置为true来提供相应的保证。

\mySamllsection{std::invocable和std::regular\_invocable}

语义约束的简单示例是std::invocable和std::regular\_invocable概念之间的区别,后者保证不修改传递的操作和传递的参数的状态。

但不能用编译器检查这两个概念之间的区别,所以std::regular\_invocable这个概念记录了指定API的意图。简单起见,这里只使用了std::invocable。

\mySamllsection{std::weakly\_incrementable和std::incrementable}

incrementable和weak\_incrementable这两个概念除了在语法上有一定的区别外,在语义上也有区别:

\begin{itemize}
\item
incrementable要求相同值的每次递增都产生相同的结果。

\item
weakly\_incrementable只要求类型支持自增操作符,增加相同的值,可能会产生不同的结果。
\end{itemize}

因此:

\begin{itemize}
\item
当满足incrementable时,可以从一个起始值在一个范围内迭代多次。

\item
当仅满足weakly\_incrementable条件时,只能在一个范围内迭代一次。具有相同起始值的第二次迭代可能产生不同的结果。
\end{itemize}

这种差异对迭代器很重要:输入流迭代器(从流中读取值的迭代器)只能迭代一次,因为下一次迭代产生不同的值,所以输入流迭代器满足weakly\_incrementable概念,但不满足可递增概念,但这些概念不能用于检查这种差异:

\begin{cpp}
std::weakly_incrementable<std::istream_iterator<int>> // yields true
std::incrementable<std::istream_iterator<int>> // OOPS: also yields true
\end{cpp}

原因是这种差异是一种语义约束,不能在编译时检查,所以这些概念可以用来记录约束条件:

\begin{cpp}
template<std::weakly_incrementable T>
void algo1(T beg, T end); // single-pass algorithm

template<std::incrementable T>
void algo2(T beg, T end); // multi-pass algorithm
\end{cpp}

注意,这里对算法使用了不同的名称。由于无法检查约束的语义差异,因此开发者可以决定不传递输入流迭代器:

\begin{cpp}
algo1(std::istream_iterator<int>{std::cin}, // OK
	  std::istream_iterator<int>{});

algo2(std::istream_iterator<int>{std::cin}, // OOPS: violates constraint
	  std::istream_iterator<int>{});
\end{cpp}

然而,不能根据这个差异来区分两种实现:

\begin{cpp}
template<std::weakly_incrementable T>
void algo(T beg, T end); // single-pass implementation

template<std::incrementable T>
void algo(T beg, T end); // multi-pass implementation
\end{cpp}

若在这里传递一个输入流迭代器,编译器将不正确地使用多通道实现:

\begin{cpp}
algo(std::istream_iterator<int>{std::cin}, // OOPS: calls the wrong overload
	 std::istream_iterator<int>{});
\end{cpp}

这里有一个解决方案,因为对于这种语义差异,C++98已经引入了迭代器特性,迭代器概念使用了这些特性。若使用这些概念(或相应的范围概念),一切都会很好:

\begin{cpp}
template<std::input_iterator T>
void algo(T beg, T end); // single-pass implementation
	
template<std::forward_iterator T>
void algo(T beg, T end); // multi-pass implementation

algo(std::istream_iterator<int>{std::cin}, // OK: calls the right overload
	std::istream_iterator<int>{});
\end{cpp}

推荐使用更具体的迭代器和范围概念,其也支持新的和修改过的迭代器类别。


















