
现在,看看如何使用概念的一些指导方针。请注意,我们仍在学习如何最好地使用概念。此外,随着时间的推移,对概念的改进支持可能会改变一些指导方针。

\mySubsubsection{3.5.1}{概念应该分组}

为类型的每个属性或功能引入一个概念肯定是过于细粒度了,所以我们有太多编译器必须处理的概念,并且都必须将其指定为约束。

因此,应该提供常见和典型方面的概念,以区分不同类别的需求或类型。然而,也有一些极端情况。

C++标准库提供了一个很好的设计示例,可以遵循这种方法。提供的大多数概念都用于将类型分类,如范围、迭代器、函数等作为一个整体。然而,为了支持包容并确保概念的一致性,提供了几个基本概念(例如std::movable)。

结果是一个相当复杂的包含图。描述C++标准概念的那一章,将对概念进行分组。

\mySubsubsection{3.5.2}{小心定义概念}

概念包含,这意味着一个概念可以是另一个概念的子集,因此在重载解析中更受约束的概念是首选。

然而,需求和约束可以用不同的方式定义。对于编译器来说,找出一组需求是否是另一组需求的子集可能并不容易。

例如,若两个模板参数的概念是可交换的(因此两个参数的顺序不重要),则需要仔细设计概念。有关详细信息和示例,请参阅如何定义概念std::same\_as的讨论。

\mySubsubsection{3.5.3}{概念与类型特征和布尔表达式}

概念不仅仅是在编译时计算布尔值结果的表达式。与类型特征和其他编译时表达式相比,读者们通常应该更喜欢使用它们。

不过,概念有几个好处:

\begin{itemize}
\item
互包含。

\item
可以直接用作模板参数或auto前面的类型约束。

\item
可以与前面介绍的编译时if(if constexpr)一起使用。
\end{itemize}

\mySamllsection{从包含中获益}


概念的主要好处是互包含,而类型特征不可互包含。

考虑下面的例子,用定义为类型特性的两个需求重载函数foo():

\begin{cpp}
template<typename T, typename U>
requires std::is_same_v<T, U> // using traits
void foo(T, U)
{
	std::cout << "foo() for parameters of same type" << '\n';
}

template<typename T, typename U>
requires std::is_same_v<T, U> && std::is_integral_v<T>
void foo(T, U)
{
	std::cout << "foo() for integral parameters of same type" << '\n';
}

foo(1, 2); // ERROR: ambiguity: both requirements are true
\end{cpp}

问题是,若两个需求都求值为true,则两个重载都适合,并且没有规则表明其中一个优先于另一个。因此,编译器将停止编译,并出现歧义错误。

若使用相应的概念,编译器会发现第二个需求是一种特化,若两个需求都满足,则更倾向于使用它:

\begin{cpp}
template<typename T, typename U>
requires std::same_as<T, U> // using concepts
void foo(T, U)
{
	std::cout << "foo() for parameters of same type" << '\n';
}

template<typename T, typename U>
requires std::same_as<T, U> && std::integral<T>
void foo(T, U)
{
	std::cout << "foo() for integral parameters of same type" << '\n';
}

foo(1, 2); // OK: second foo() preferred
\end{cpp}

\mySamllsection{从if constexpr概念中获益}

C++17引入了编译时if,允许根据特定的编译时条件切换代码。

例如(如前所述):

\begin{cpp}
template<typename Coll, typename T>
void add(Coll& coll, const T& val) // for floating-point value types
{
	if constexpr(std::is_floating_point_v<T>) {
		... // special code for floating-point values
	}
	coll.push_back(val);
}
\end{cpp}

当泛型代码必须为不同类型的参数提供不同的实现,但签名是相同的时候,使用这种方法比提供重载或特化的模板更具可读性。

但是,不能使用if constexpr来提供不同的API,以允许其他人稍后添加其他重载或特化,或者在某些情况下完全禁用此模板。但请记住,可以根据需求约束成员函数来启用或禁用API的某些部分。








