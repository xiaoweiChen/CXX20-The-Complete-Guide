

C++20引入了一个扩展,允许对泛型Lambda使用模板形参。可以在捕获子句和调用参数(若有的话)之间指定这些模板参数:

\begin{cpp}
auto foo = []<typename T>(const T& param) { // OK since C++20
				T tmp{}; // declare object with type of the template parameter
				...
			};
\end{cpp}

Lambda的模板形参有一个优点,即在声明泛型参数时为类型或类型的一部分提供名称。例如:

\begin{cpp}
[]<typename T>(T* ptr) { // OK since C++20
	... // can use T as type of the value that ptr points to
};
\end{cpp}

或:

\begin{cpp}
[]<typename T, int N>(T (&arr)[N]) {
	... // can use T as element type and N as size of the passed array
};
\end{cpp}

若想知道为什么在这些情况下不使用函数模板,请记住Lambda提供了一些函数无法提供的便利:

\begin{itemize}
\item 
可以在函数内部定义。

\item 
可以捕获运行时的值,来指定在运行时的功能行为。

\item 
可以将它们作为参数传递,而无需指定参数类型。
\end{itemize}


\mySubsubsection{17.1.1}{使用泛型Lambda的模板参数}

显式模板参数可用于特化(或部分限制)泛型Lambda的参数类型。考虑下面的例子:

\begin{cpp}
[]<typename T>(const std::vector<T>& vec) { // can only pass vectors
	...
};
\end{cpp}

这个Lambda只接受vector作为参数。当使用auto时,将实参限制为vector并不容易,因为C++(还)不支持std::vector<auto>类型,但还可以使用类型约束来约束参数的类型(例如:要求可随机访问,甚至特定类型)。

显式模板参数也有助于避免对decltype的需要。例如,要在Lambda中完美地转发泛型参数包,可以这样:

\begin{cpp}
[]<typename... Types>(Types&&... args) {
	foo(std::forward<Types>(args)...);
};
\end{cpp}

而非:

\begin{cpp}
[] (auto&&... args) {
	foo(std::forward<decltype(args)>(args)...);
};
\end{cpp}

类似的例子是在访问std::variant<>时,会具有特定类型的行为(C++17中引入):

\begin{cpp}
std::variant<int, std::string> var;
...
// call generic lambda with type-specific behavior:
std::visit([](const auto& val) {
				if constexpr(std::is_same_v<decltype(val), const std::string&>) {
					... // string-specific processing
				}
				std::cout << val << '\n';
			},
			var);
\end{cpp}

必须使用decltype()来获取参数的类型,并将该类型作为const\&进行比较(或删除const和引用)。C++20起,可以这样写代码:

\begin{cpp}
std::visit([]<typename T>(const T& val) { // since C++20
				if constexpr(std::is_same_v<T, std::string>) {
					... // string-specific processing
				}
				std::cout << "value: " << val << '\n';
			},
			var);
\end{cpp}

还可以为coneval的Lambda声明模板参数,强制在编译时执行。

\mySubsubsection{17.1.2}{Lambda模板参数的显式规范}

Lambda提供了一种方便的方式来定义函数对象(函子)。对于泛型lambda,函数调用函数操作符是一个模板。使用为参数指定名称,而不是使用auto的语法,可以在生成的函数调用操作符中为模板参数指定名称。

例如,若定义了下面的Lambda:

\begin{cpp}
auto primeNumbers = [] <int Num> () {
						std::array<int, Num> primes{};
						... // compute and assign first Num prime numbers
						return primes;
					};
\end{cpp}

编译器定义了相应的闭包类型:

\begin{cpp}
class NameChosenByCompiler {
public:
	...
	template<int Num>
	auto operator() () const {
		std::array<int, Num> primes{};
		... // compute and assign first Num prime numbers
		return primes;
	}
};
\end{cpp}

并创建该类的一个对象(若没有捕获值则使用默认构造函数):

\begin{cpp}
auto primeNumbers = NameChosenByCompiler{};
\end{cpp}

要显式指定模板形参,当将Lambda用作函数时,必须将其传递给函数操作符:

\begin{cpp}
// initialize array with the first 20 prime numbers:
auto primes20 = primeNumbers.operator()<20>();
\end{cpp}

除非使用间接调用,否则在指函数操作符的模板形参时,无法避免指定函数操作符。

可以尝试将模板参数设置为编译时值,使其可推导,但最终的语法并没有好到哪里去:

\begin{cpp}
auto primeNumbers = [] <int Num> (std::integral_constant<int, Num>) {
						std::array<int, Num> primes{};
						... // compute and assign first Num prime numbers
						return primes;
					};

// initialize array with the first 20 prime numbers:
auto primes20 = primeNumbers(std::integral_constant<int,20>{});
\end{cpp}

或者,可以考虑使用变量模板,这是C++14中引入的一种技术。这样,就可以将primeNumbers变量设置为泛型,而不是将Lambda设置为泛型:

\begin{cpp}
template<int Num>
auto primeNumbers = [] () {
						std::array<int, Num> primes{};
						... // compute and assign first Num prime numbers
						return primes;
					};
...
// initialize array with the first 20 prime numbers:
auto primes20 = primeNumbers<20>();
\end{cpp}

但这种情况下,不能在函数作用域中定义Lambda。泛型Lambda可以在作用域内局部定义泛型功能。








