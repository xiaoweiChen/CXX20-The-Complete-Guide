
C++20为聚合提供了一些改进,这些改进将在本节中介绍:

\begin{itemize}
\item 
(部分)支持指定初始化式(特定成员的初始值)

\item 
可以用圆括号初始化聚合

\item 
固定聚合的定义和std::is\_default\_constructible<>
\end{itemize}

此外,本书的其他部分描述了聚合在泛型代码中的使用:

\begin{itemize}
\item 
对聚合使用类模板参数推导(CTAD)

\item 
聚合可以用作非类型模板参数(NTTP)。
\end{itemize}

\mySubsubsection{21.5.1}{指定初始化器}

对于聚合,C++20提供了一种方法来指定应该用传递的初始值初始化哪个成员,但只能使用它来跳过参数。

例如,假设有以下聚合类型:

\begin{cpp}
struct Value {
	double amount = 0;
	int precision = 2;
	std::string unit = "Dollar";
};
\end{cpp}

那么,现在支持以下方式来初始化该类型的值:

\begin{cpp}
Value v1{100}; // OK (not designated initializers)
Value v2{.amount = 100, .unit = "Euro"}; // OK (second member has default value)
Value v3{.precision = 8, .unit = "$"}; // OK (first member has default value)
\end{cpp}

完整的示例请参见lang/designated.cpp。

注意以下限制:

\begin{itemize}
\item 
必须使用=或\{\}传递初始值。

\item 
可以跳过会员,但必须遵循其顺序。按名称初始化的成员的顺序必须与声明中的顺序匹配。

\item 
必须对所有参数使用指定初始化式,或者不使用指定初始化式。不允许混合初始化。

\item 
不支持对数组使用指定初始化。

\item 
可以使用指定初始化式嵌套初始化,但不能直接使用.mem.mem。

\item 
初始化带有圆括号的聚合时,不能使用指定初始化器。

\item 
指定初始化式也可以用在联合体中。
\end{itemize}

例如:

\begin{cpp}
Value v4{100, .unit = "Euro"}; // ERROR: all or none designated
Value v5{.unit = "$", .amount = 20}; // ERROR: invalid order
Value v6(.amount = 29.9, .unit = "Euro"); // ERROR: only supported for curly braces
\end{cpp}

与编程语言C相比,指定初始化器遵循成员顺序的限制,可以用于所有参数,也可以不用于参数,不支持直接嵌套,不支持数组。尊重成员顺序的原因是确保初始化反映了调用构造函数的顺序(与调用析构函数的顺序相反)。

作为使用=、\{\}和联合体嵌套初始化的示例:

\begin{cpp}
union Sub {
	double x = 0;
	int y = 0;
};

struct Data {
	std::string name;
	Sub val;
};

Data d1{.val{.y=42}}; // OK
Data d2{.val = {.y{42}}}; // OK
\end{cpp}

不能直接嵌套指定初始化式:

\begin{cpp}
Data d2{.val.y = 42}; // ERROR
\end{cpp}


\mySubsubsection{21.5.2}{用圆括号聚合初始化}

假设已经声明了以下集合:

\begin{cpp}
struct Aggr {
	std::string msg;
	int val;
};
\end{cpp}

C++20之前,只能使用花括号来初始化带有值的聚合:

\begin{cpp}
Aggr a0; // OK, but no initialization
Aggr a1{}; // OK, value initialized with "" and 0
Aggr a2{"hi"}; // OK, initialized with "hi" and 0
Aggr a3{"hi", 42}; // OK, initialized with "hi" and 42
Aggr a4 = {}; // OK, initialized with "" and 0
Aggr a5 = {"hi"}; // OK, initialized with "hi" and 0
Aggr a6 = {"hi", 42}; // OK, initialized with "hi" and 42
\end{cpp}

自C++20起,还可以使用圆括号作为外部字符来直接初始化,不带=:

\begin{cpp}
Aggr a7("hi"); // OK since C++20: initialized with "hi" and 0
Aggr a8("hi", 42); // OK since C++20: initialized with "hi" and 42
Aggr a9({"hi", 42}); // OK since C++20: initialized with "hi" and 42
\end{cpp}

使用=或内括号仍然不起作用:

\begin{cpp}
Aggr a10 = "hi"; // ERROR
Aggr a11 = ("hi", 42); // ERROR
Aggr a12(("hi", 42)); // ERROR
\end{cpp}

使用内括号甚至可以编译,其用作使用逗号操作符的表达式周围的圆括号。

注意,现在甚至可以使用圆括号来初始化一个边界未知的数组:

\begin{cpp}
int a1[]{1, 2, 3}; // OK since C++11
int a2[](1, 2, 3); // OK since C++20
int a3[] = {1, 2, 3}; // OK
int a4[] = (1, 2, 3); // still ERROR
\end{cpp}

然而,不支持"大括号省略"(没有嵌套的大括号可以省略):

\begin{cpp}
struct Arr {
	int elem[10];
};

Arr arr1{1, 2, 3}; // OK
Arr arr2(1, 2, 3); // ERROR
Arr arr3{{1, 2, 3}}; // OK
Arr arr4({1, 2, 3}); // OK (even before C++20)
\end{cpp}

因此,要初始化std::arrays,仍然必须使用大括号:

\begin{cpp}
std::array<int,3> a1{1, 2, 3}; // OK: shortcut for std::array{{1, 2, 3}}
std::array<int,3> a2(1, 2, 3); // still ERROR
\end{cpp}

\mySamllsection{用圆括号初始化聚合的原因}

支持带圆括号的聚合初始化的原因是,可以用圆括号调用new操作符:

\begin{cpp}
struct Aggr {
	std::string msg;
	int val;
};

auto p1 = new Aggr{"Rome", 200}; // OK since C++11
auto p2 = new Aggr("Rome", 200); // OK since C++20 (ERROR before C++20)
\end{cpp}

这有助于支持在类型中使用聚合,这些类型在内部调用new时使用括号将值存储在现有内存中,就像容器和智能指针的情况一样。事实上,自C++20起,以下是可能的:

\begin{itemize}
\item 
现在可以对聚合使用std::make\_unique<>()和std::make\_shared<>():

\begin{cpp}
auto up = std::make_unique<Aggr>("Rome", 200); // OK since C++20
auto sp = std::make_shared<Aggr>("Rome", 200); // OK since C++20
\end{cpp}

C++20之前,无法对聚合使用这些辅助函数。

\item 
现在可以将新值放置到聚合容器中:

\begin{cpp}
std::vector<Aggr> cont;
cont.emplace_back("Rome", 200); // OK since C++20
\end{cpp}
\end{itemize}

请注意,仍然有一些类型不能用括号初始化,但可以用花括号初始化:作用域枚举(枚举类类型)。类型std::byte (C++17引入)就是一个例子:

\begin{cpp}
std::byte b1{0}; // OK
std::byte b2(0); // still ERROR
auto upb2 = std::make_unique<std::byte>(0); // still ERROR
auto upb3 = std::make_unique<std::byte>(std::byte{0}); // OK
\end{cpp}

对于std::array,仍然需要大括号(如上所述):

\begin{cpp}
std::vector<std::array<int, 3>> ca;
ca.emplace_back(1, 2, 3); // ERROR
ca.emplace_back({1, 2, 3}); // ERROR
ca.push_back({1, 2, 3}); // still OK
\end{cpp}

\mySamllsection{详细地用圆括号聚合初始化}

引入圆括号初始化的建议列出了以下设计准则:

\begin{itemize}
\item 
Type(val)的现有含义都不应该改变。

\item 
括号初始化和括号初始化应尽可能相似,但也应尽可能不同,以符合现有的括号列表和括号列表的心智模型。
\end{itemize}

实际上,带花括号的聚合初始化和带圆括号的聚合初始化有以下区别:

\begin{itemize}
\item 
带圆括号的初始化不会检测窄化转换。

\item 
用圆括号初始化允许所有隐式转换(不仅是从派生类到基类)。

\item
当使用括号时,引用成员不会延长传递的临时对象的生命周期.

\item
使用圆括号不支持大括号省略(使用它们就像向形参传递实参一样)。

\item
带空括号的初始化甚至适用于显式成员。

\item
使用圆括号不支持指定初始化式。
\end{itemize}

缺少检测窄化的例子如下:

\begin{cpp}
struct Aggr {
	std::string msg;
	int val;
};

Aggr a1{"hi", 1.9}; // ERROR: narrowing
Aggr a2("hi", 1.9); // OK, but initializes with 1

std::vector<Aggr> cont;
cont.emplace_back("Rome", 1.9); // initializes with 1
\end{cpp}

注意,emplace函数永远不会检测窄化。

处理隐式转换时的区别示例如下:

\begin{cpp}
example of the difference when dealing with implicit conversions is this:
struct Other {
	...
	operator Aggr(); // defines implicit conversion to Aggr
};

Other o;
Aggr a7{o}; // ERROR: no implicit conversion supported
Aggr a8(o); // OK, implicit conversion possible
\end{cpp}

注意,聚合本身不能定义转换,因为聚合不能有用户定义的构造函数。

缺少大括号省略和大括号初始化的复杂规则导致以下行为:

\begin{cpp}
Aggr a01{"x", 65}; // init string with "x" and int with 65
Aggr a02("x", 65); // OK since C++20 (same effect)

Aggr a11{{"x", 65}}; // runtime ERROR: "x" doesn’t have 65 characters
Aggr a12({"x", 65}); // OK even before C++20: init string with "x" and int with 65

Aggr a21{{{"x", 65}}}; // ERROR: cannot initialize string with initializer list of "x" and 65
Aggr a22({{"x", 65}}); // runtime ERROR: "x" doesn’t have 65 characters

Aggr a31{'x', 65}; // ERROR: cannot initialize string with ’x’
Aggr a32('x', 65); // ERROR: cannot initialize string with ’x’

Aggr a41{{'x', 65}}; // init string with ’x’ and char(65)
Aggr a42({'x', 65}); // OK since C++20 (same effect)

Aggr a51{{{'x', 65}}}; // init string with ’x’ and char(65)
Aggr a52({{'x', 65}}); // OK even before C++20: init string with ’x’ and 65

Aggr a61{{{{'x', 65}}}}; // ERROR
Aggr a62({{{'x', 65}}}); // OK even before C++20: init string with ’x’ and 65
\end{cpp}

当执行复制初始化(用=初始化)时,显式很重要,使用空括号可能会产生不同的效果,如下所示:

\begin{cpp}
struct C {
	explicit C() = default;
};

struct A { // aggregate
	int i;
	C c;
};

auto a1 = A{42, C{}}; // OK: explicit initialization
auto a2 = A(42, C()); // OK since C++20: explicit initialization

auto a3 = A{42}; // ERROR: cannot call explicit constructor
auto a4 = A(42); // ERROR: cannot call explicit constructor

auto a5 = A{}; // ERROR: cannot call explicit constructor
auto a6 = A(); // OK: can call explicit constructor
\end{cpp}

然而,这在C++20中并不新鲜,在C++20之前就已经支持a6的初始化了。

最后,若对具有右值引用成员的聚合使用圆括号进行初始化,则初始值的生存期不会延长。因此,不应该传递临时对象(右值):

\begin{cpp}
struct A {
	int a;
	int&& r;
};

int f();
int n = 10;

A a1{1, f()}; // OK, lifetime is extended
std::cout << a1.r << '\n'; // OK
A a2(1, f()); // OOPS: dangling reference
std::cout << a2.r << '\n'; // runtime ERROR
A a3(1, std::move(n)); // OK as long as a3 is used only while n exists
std::cout << a3.r << '\n'; // OK
\end{cpp}

由于这些复杂的规则和陷阱,只有在必要时才应该使用带圆括号的聚合初始化,比如在使用std::make\_unique<>()、std::make\_shared<>()或emplace函数时。

\mySubsubsection{21.5.3}{聚合体的定义}

再一次,聚合的定义在C++20中发生了变化。这一次,修改逆转了C++11中出现的扩展,结果证明这是一个错误。从C++11到C++17,不允许使用用户提供的构造函数。基于这个原因,下面是聚合的合法定义:

\begin{cpp}
struct A { // aggregate from C++11 until C++17
	...
	A() = delete; // user-declared, but not user-provided constructor
};
\end{cpp}

有了这个定义,即使类型有一个删除的默认构造函数,聚合初始化也是有效的:

\begin{cpp}
A a1; // ERROR
A a2{}; // OK from C++11 until C++17
\end{cpp}

这不仅适用于默认构造函数。例如:

\begin{cpp}
struct D { // aggregate from C++11 until C++17
	int i = 0;
	D(int) = delete; // user-declared, but not user-provided constructor
};

D d1(3); // ERROR
D d2{3}; // OK from C++11 until C++17
\end{cpp}

这种特殊的行为是在一个非常特殊的情况下引入的,但完全违反直觉。[该特性是作为“hack”引入的,以在初始化原子类型时支持C兼容性。]然而,就连几名委员会成员也不知道这种行为,他们感到非常惊讶,甚至发现这种支持根本不需要。C++20修正了这个问题,重新要求聚合没有用户声明的构造函数(就像C++11之前的情况一样):

\begin{cpp}
struct A { // NO aggregate since C++20
	...
	A() = delete; // user-declared, but not user-provided constructor
};

A a1; // ERROR
A a2{}; // ERROR since C++20
\end{cpp}

因此,如上所述,对于类型A和D,类型特性std::is\_default\_constructible\_v<>将不再为true。

但请注意,一些程序员在创建聚合类型的对象时确实使用此特性来强制(可能为空)花括号(这确实可确保始终对成员进行初始化)。对于它们来说,获得相同行为的解决方法是从基类型派生,如下所示:

\begin{cpp}
struct MustInit {
	MustInit(MustInit&&) = default;
};

struct A : MustInit {
	...
};

A a1; // ERROR
A a2{}; // OK
\end{cpp}

总而言之,自C++20起,聚合的定义如下:

\begin{itemize}
\item 
一个数组

\item
或者一个类型(类、结构或联合):

\begin{itemize}
\item 
没有用户声明的构造函数

\item
using声明没有继承构造函数

\item
没有私有或受保护的非静态数据成员

\item
没有虚函数

\item
没有虚拟、私有或受保护的基类
\end{itemize}

\end{itemize}

为了初始化聚合,需要应用以下附加限制:

\begin{itemize}
\item 
没有私有或受保护的基类成员

\item 
没有私有或受保护的构造函数
\end{itemize}











