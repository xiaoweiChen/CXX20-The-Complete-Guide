
自C++11起,可以指定属性(启用或禁用警告的正式注释)。在C++20中,又引入了新的属性,并扩展了现有的属性。

\mySubsubsection{21.6.1}{属性[[likely]]和[unlikely]]}

C++20引入了新属性[[likely]]和[[unlikely]],以帮助编译器执行分支优化。

当代码中有多条路径时,可以使用这些属性向编译器提示哪条路径最有可能出现,哪条路径最不可能出现。

例如:

\begin{cpp}
int f(int n)
{
	if (n <= 0) [[unlikely]] { // n <= 0 is considered to be arbitrarily unlikely
		return n;
	}
	else {
		return n * n;
	}
}
\end{cpp}

例如,这可能会迫使编译器生成直接处理else分支的汇编代码,同时跳转到后面的then情况的汇编命令。

可能有,但不保证有相同的效果:

\begin{cpp}
int f(int n)
{
	if (n <= 0) {
		return n;
	}
	else [[likely]] { // n > 0 is considered to be arbitrarily likely
		return n * n;
	}
}
\end{cpp}

下面是另一个例子:

\begin{cpp}
int g(int n)
{
	switch (n) {
		case 1:
		...
		break;
		[[likely]] case 2: // n == 2 is considered to be arbitrarily most likely
		...
		break;
	}
	...
}
\end{cpp}

这些属性的效果特定于编译器,不能保证这些属性的影响。

在使用这些属性时应该小心,并仔细检查它们的效果。通常,编译器更了解如何优化代码,若过度使用这些属性可能会适得其反。

\mySubsubsection{21.6.2}{属性[[no\_unique\_address]]}

类通常具有影响行为但不提供状态的成员,例如:无序容器的哈希函数、std::unique\_ptr的delete函数,或者容器或字符串的标准分配器。它们提供的只是成员函数(和静态成员),而不是非静态数据成员。

然而,成员通常需要内存,即使他们不存储任何东西。例如,下面的代码:

\begin{cpp}
struct Empty {}; // empty class: size is usually 1

struct I { // size is same as sizeof(int)
	int i;
};

struct EandI { // size is sum of size of members with alignment
	Empty e;
	int i;
};

std::cout << "sizeof(Empty): " << sizeof(Empty) << '\n';
std::cout << "sizeof(I): " << sizeof(I) << '\n'
std::cout << "sizeof(EandI): " << sizeof(EandI) << '\n';
\end{cpp}

根据int类型的大小,输出可能如下所示:

\begin{shell}
sizeof(E):     1
sizeof(I):     4
sizeof(EandI): 8
\end{shell}

这是不必要的空间浪费。在C++20之前,可以使用空基类优化(EBCO)来避免不必要的开销。通过从没有数据成员的类派生,编译器可以节省相应的空间:

\begin{cpp}
struct EbasedI : Empty { // using EBCO
	int i;
};

std::cout << "sizeof(EbasedI): " << sizeof(EbasedI) << '\n';
\end{cpp}

在具有上述情况的平台上,输出如下:

\begin{shell}
sizeof(EbasedI): 4
\end{shell}

然而,这种解决方法有点笨拙,可能并不总是有效。例如,若空基类是final,则不能使用EBCO。

自C++20起,有一种不同的方法可以获得相同的效果。只需要用属性[[no\_unique\_address]]声明不提供状态的成员:

\begin{cpp}
struct EattrI { // same effect as EBCO
	[[no_unique_address]] Empty e;
	int i;
};

struct IattrE { // same effect as EBCO
	int i;
	[[no_unique_address]] Empty e;
};

std::cout << "sizeof(EattrI): " << sizeof(EattrI) << '\n';
std::cout << "sizeof(IattrE): " << sizeof(IattrE) << '\n';
\end{cpp}

在上面支持此属性的平台上,输出会变为:

\begin{shell}
sizeof(EattrI): 4
sizeof(IattrE): 4
\end{shell}

请注意,编译器不需要遵循此属性。

标记为[[no\_unique\_address]]的成员在初始化时仍会视为成员:

\begin{cpp}
EattrI ei = {42}; // ERROR: can’t initialize member e with 42
EattrI ei = {{},42}; // OK
\end{cpp}

这种优化还意味着成员e的地址与同一对象的成员i的地址相同,则两个不同对象的成员e有不同的地址。

若数据类型只有具有此属性的数据成员,则类型trait std::is\_empty\_v<>是否产生true是由实现定义的:

\begin{cpp}
struct OnlyEmpty {
	[[no_unique_address]] Empty e;
};

std::is_empty_v<OnlyEmpty> // might yield true or false
\end{cpp}

最后,请注意Visual C++目前忽略了这个属性。原因是Visual C++最初允许这个属性不遵守它,现在遵守它变成了一个ABI中断。不过,Visual C++可能会在未来的ABI版本中支持它。在此之前,可以使用[[msvc::no\_unique\_address]]代替:

\begin{cpp}
struct EattrI { // also works for Visual C++
	[[no_unique_address]] [[msvc::no_unique_address]] Empty e;
	Type i;
};
\end{cpp}

\mySubsubsection{21.6.3}{带参数的属性[[nodiscard]]}

C++17引入了属性[[nodiscard]],若没有使用函数的返回值,可以使用该属性鼓励编译器发出警告。

当返回值未被使用时,[[nodiscard]]通常用来表示错误行为。不当行为可能是:

\begin{itemize}
\item 
内存泄漏,例如:不使用返回的已分配内存。

\item 
意外或非直观的行为,例如:在不使用返回值时获得不同/意外的行为

\item 
不必要的开销,例如:若返回值未被使用,则使用no-op的行为
\end{itemize}

但是,没有办法指定一条消息来解释为什么开发者应该使用返回值。C++20为此引入了一个可选参数。

例如:

\begin{cpp}
class MyType {
	public:
	...
	[[nodiscard("Possible memory leak")]] // OK, since C++20
	char* release();
	
	void clear();
	
	[[nodiscard("Did you mean clear()?")]] // OK, since C++20
	bool empty() const;
};
\end{cpp}

当empty()的返回值未使用时,第二条声明要求编译器打印警告消息“Did you mean clear()?”。事实上,C++20为所有标准容器的成员函数empty()引入了这个特性。根据反馈,我们知道编译器确实发现了一些错误,开发者认为他们要求将集合变为空。


