
C++20是现代C++的下一个版本,其已经(部分地)在最新版本的GCC、Clang和Visual C++中有所支持。了解到C++20和了解到C++11一样重要,C++20包含了大量新特性和库,将再次改变C++编程的方式。这既适用于应用程序开发者,也适用于提供基础库的开发者。

\subsection*{实践方式}
\addcontentsline{toc}{subsection}{实践方式}

本书从两个方面进行了实践:

\begin{itemize}
\item
我正在写一本深入实践的书,涵盖了不同开发者和C++开发工作组,以及介绍了复杂的新特性。不过,我也会一些问题。

\item
我自己在Leanpub上出版这本书,并按需印刷。这本书是一步一步写的,当有了重要的改进,我就会出版新版本。
\end{itemize}

好的一面是:

\begin{itemize}
\item
You get the view of the language features from an experienced application programmer—somebody who feels the pain a feature might cause and asks the relevant questions to be able to explain the motivation for a feature, its design, and all consequences for using it in practice.

\item
You can benefit from my experience with C++20 while I am still learning and writing.

\item
This book and all readers can benefit from your early feedback.
\end{itemize}

This means that you are also part of the experiment. So help me out: give feedback about flaws, errors, features that are not explained well, or gaps, so that we all can benefit from these improvements.

\subsection*{致谢}
\addcontentsline{toc}{subsection}{致谢}

This book would not have been possible without the help and support of a huge number of people.

First of all, I would like to thank you, the C++ community, for making this book possible. The incredible design of all the features of C++20, the helpful feedback, and your curiosity are the basis for the evolution of a successful language. In particular, thanks for all the issues you told me about and explained and for the feedback you gave.

I would especially like to thank everyone who reviewed drafts of this book or corresponding slides and provided valuable feedback and clarification. These reviews increased the quality of the book significantly, again proving that good things are usually the result of collaboration between many people. Therefore, so far (this list is still growing) huge thanks to Carlos Buchart, Javier Estrada, Howard Hinnant, Yung-Hsiang Huang, Daniel Krugler, Dietmar K ¨ uhl, Jens Maurer, Paul Ranson, Thomas Symalla, Steve Vinoski, Ville ¨ Voutilainen. Andreas Weis, Hui Xie, Leor Zolman, and Victor Zverovich.

In addition, I would like to thank everyone on the C++ standards committee. In addition to all the work involved in adding new language and library features, these experts spent many, many hours explaining and discussing their work with me, and they did so with patience and enthusiasm. Special thanks here go to Howard Hinnant, Tom Honermann, Tomasz Kaminski, Peter Sommerlad, Tim Song, Barry Revzin, Ville Voutilainen, and Jonathan Wakely.

Special thanks go to the LaTeX community for a great text system and to Frank Mittelbach for solving my \LaTeX{} issues (it was almost always my fault).

And finally, many thanks go to my proofreader, Tracey Duffy, who has again done a tremendous job of converting my “German English” into native English.


















