下面几节解释自C++20后比较操作符的详细情况。

\mySubsubsection{1.2.1}{使用<=>操作符}

<=>操作符是一个新的二元操作符。所有基本数据类型都有定义,对于其中定义了关系运算符的数据类型,也可以由用户自行重载定义。

<=>操作符优先于所有其他比较操作符,所以需要在输出语句中使用括号,但不需要将其结果与其他值进行比较:

\begin{cpp}
std::cout << (0 < x <=> y) << '\n'; // calls 0 < (x <=> y)
\end{cpp}

请注意,必须包含一个特定的头文件来处理<=>操作符的结果:

\begin{cpp}
#include <compare>
\end{cpp}

这适用于声明(默认情况下)、实现或使用。例如:

\begin{cpp}
#include <compare> // for calling <=>

auto x = 3 <=> 4; // does not compile without header <compare>
\end{cpp}

大多数标准类型(字符串、容器、<utility>)的头文件都包含这个头文件,但要对不需要此头的值或类型使用操作符,必须包含<compare>。

注意,<=>操作符用于实现类型。在<=>操作符的实现之外,开发者不应该直接调用<=>。虽然这样做没问题,但永远不要用a<=>b < 0来代替a<b。

\mySubsubsection{1.2.2}{比较类别类型}

<=>操作符不返回布尔值,但其作用类似于三向比较,产生负值表示信号较少,正值表示信号较大,0表示信号相等或等效。这种行为类似于C函数strcmp()的返回值,也有一个重要的区别:返回值不是整数值,C++标准库提供了三种可能的返回类型,其反映了相应的比较类别。

\mySamllsection{比较类别}

当比较两个值并按顺序排列时,有可能发生不同的“类别”行为:

\begin{itemize}
\item
对于\textbf{强排序}(也称为全排序),给定类型的值都小于或等于或大于该类型的其他值(包括其本身)。

这一类的典型例子是整数值或常见的字符串类型,字符串s1小于等于或大于字符串s2。

若此类别的一个值既不小于也不大于另一个值,则两个值相等。若有多个对象,可以按升序或降序进行排序(相等值的顺序任意)。

\item
对于\textbf{弱排序},给定类型的任何值都小于、等于或大于该类型的任何其他值(包括其本身),但相等的值不一定是相等的(具有相同的值)。

此类别的典型示例是用于不区分大小写的字符串的类型,字符串"hello"小于"hello1"且大于"hell"。尽管,"hello"与"HELLO"这两个字符串不相等但等价。

若这个类别的值既不小于也不大于另一个值,那么这两个值至少是相等的(甚至可能是相等的)。若有多个对象,则可以按升序或降序进行排序(相等值的顺序可以任意)。

\item
对于\textbf{偏排序},给定类型的任何值都可以小于、等于或大于该类型的任何其他值(包括其本身),可能根本不能指定两个值之间的特定顺序。

这种类型的典型示例是浮点类型,因为可能需要处理特殊值NaN(“非数字”),任何值与NaN的比较结果都为false。因此,比较可能导致两个值是无序的,并且比较操作符可能返回四个值中的一个。

若有多个对象,可能无法按升序或降序对其进行排序(除非确保不存在无法排序的值)。
\end{itemize}

\mySamllsection{使用标准库比较类别}

\begin{itemize}
\item
std::strong\_ordering的值:
\begin{itemize}
\item
std::strong\_ordering::less

\item
std::strong\_ordering::equal (也可是std::strong\_ordering::equivalent)

\item
std::strong\_ordering::greater
\end{itemize}

\item
std::weak\_ordering的值:
\begin{itemize}
\item
std::weak\_ordering::less

\item
std::weak\_ordering::equivalent

\item
std::weak\_ordering::greater
\end{itemize}

\item
std::partial\_ordering的值:
\begin{itemize}
\item
std::partial\_ordering::less

\item
std::partial\_ordering::equivalent

\item
std::partial\_ordering::greater

\item
std::partial\_ordering::unordered
\end{itemize}
\end{itemize}

注意,所有类型都有less、greater和equivalent值。但strong\_ordering也有equal,这与这里的equal相同,而partial\_ordering的值为unordered,既不表示小于,等于和大于。

较强比较类型具有向较弱比较类型的隐式类型转换,可以将strong\_ordering值用作weak\_ordering值,或partial\_ordering值(相等则为equivalent)。

\mySubsubsection{1.2.3}{使用<=>操作符比较类别}

<=>操作符应该返回比较类别类型的值,表示比较结果,以及该结果是否能够创建强/全、弱或偏排序的信息。

例如,为MyType类型定义的<=>如下所示:

\begin{cpp}
std::strong_ordering operator<=> (MyType x, MyOtherType y)
{
	if (xIsEqualToY) return std::strong_ordering::equal;
	if (xIsLessThanY) return std::strong_ordering::less;
	return std::strong_ordering::greater;
}
\end{cpp}

或者,作为一个更具体的例子,为MyType类型定义<=>操作符:

\begin{cpp}
class MyType {
	...
	std::strong_ordering operator<=> (const MyType& rhs) const {
		return value == rhs.value ? std::strong_ordering::equal :
			   value < rhs.value ? std::strong_ordering::less :
							       std::strong_ordering::greater;
	}
};
\end{cpp}

通过将操作符映射到底层类型的结果,通常更容易定义操作符。所以,上面的成员<=>操作符最好只输出其成员值的值和类别:

\begin{cpp}
class MyType {
	...
	auto operator<=> (const MyType& rhs) const {
		return value <=> rhs.value;
	}
};
\end{cpp}

这不仅返回正确的值,还确保根据成员值的类型,返回值具有正确的比较类别类型。

\mySubsubsection{1.2.4}{直接调用<=>操作符}

也可以直接使用已定义的<=>操作符:

\begin{cpp}
MyType x, y;
...
x <=> y // yields a value of the resulting comparison category type
\end{cpp}

如前所述,实现<=>操作符时,应该只直接调用<=>操作符,了解返回的比较类别会非常有意义。

<=>操作符是为定义了关系操作符的所有基本类型预定义的。例如:

\begin{cpp}
int x = 17, y = 42;
x <=> y // yields std::strong_ordering::less
x <=> 17.0 // yields std::partial_ordering::equivalent
&x <=> &x // yields std::strong_ordering::equal
&x <=> nullptr // ERROR: relational comparison with nullptr not supported
\end{cpp}

此外,所有提供关系操作符的C++标准库类型现在也提供了<=>操作符。例如:

\begin{cpp}
std::string{"hi"} <=> "hi" // yields std::strong_ordering::equal;
std::pair{42, 0.0} <=> std::pair{42, 7.7} // yields std::partial_ordering::less
\end{cpp}

对于自己的类型,只需将<=>操作符定义为成员或独立函数即可。

因为返回类型取决于比较类别,所以可以根据特定的返回值进行检查:

\begin{cpp}
if (x <=> y == std::partial_ordering::equivalent) // always OK
\end{cpp}

由于隐式类型转换为较弱排序类型,若<=>操作符产生strong\_ordering或weak\_ordering类型的值,这也是可以编译的。

反过来不行。若比较产生weak\_ordering或partial\_ordering类型的值,则不能将其与strong\_ordering类型的值进行比较。

\begin{cpp}
if (x <=> y == std::strong_ordering::equal) // might not compile
\end{cpp}

然而,与0的比较总是可能的,而且通常更容易理解:

\begin{cpp}
if (x <=> y == 0) // always OK
\end{cpp}

此外,由于关系型操作符调用的新重写,<=>操作符可以间接调用:

\begin{cpp}
if (!(x < y || y < x)) // might call operator<=> to check for equality
\end{cpp}

或:

\begin{cpp}
if (x <= y && y <= x) // might call operator<=> to check for equality
\end{cpp}

注意!=操作符,永远不会重写为调用<=>操作符的方式,但可能调用由于默认操作符<=>成员隐式生成的操作符==。

\mySubsubsection{1.2.5}{处理多种排序条件}

要基于多个属性计算运算符<=>的结果,通常可以实现一连串的子比较,直到结果不相等或到达要最终属性的比较:

\begin{cpp}
class Person {
	...
	auto operator<=> (const Person& rhs) const {
		auto cmp1 = lastname <=> rhs.lastname; // primary member for ordering
		if (cmp1 != 0) return cmp1; // return result if not equal
		auto cmp2 = firstname <=> rhs.firstname; // secondary member for ordering
		if (cmp2 != 0) return cmp2; // return result if not equal
		return value <=> rhs.value; // final member for ordering
	}
};
\end{cpp}

但若属性具有不同的比较类别,则返回类型不会编译。例如,若成员名是string类型,而成员值是double类型,则返回类型会冲突:

\begin{cpp}
class Person {
	std::string name;
	double value;
	...
	auto operator<=> (const Person& rhs) const { // ERROR: different return types deduced
		auto cmp1 = name <=> rhs.name;
		if (cmp1 != 0) return cmp1; // return strong_ordering for std::string
		return value <=> rhs.value; // return partial_ordering for double
	}
};
\end{cpp}

可以使用到最弱比较类型的转换。若已知最弱的比较类型,可以声明为返回类型:

\begin{cpp}
class Person {
	std::string name;
	double value;
	...
	std::partial_ordering operator<=> (const Person& rhs) const { // OK
		auto cmp1 = name <=> rhs.name;
		if (cmp1 != 0) return cmp1; // strong_ordering converted to return type
		return value <=> rhs.value; // partial_ordering used as the return type
	}
};
\end{cpp}

若不知道比较类型(例如,类型是模板形参),可以使用新的类型特征std::common\_comparison\_category<>来计算最强的比较类型:

\begin{cpp}
class Person {
	std::string name;
	double value;
	...
	auto operator<=> (const Person& rhs) const // OK
	-> std::common_comparison_category_t<decltype(name <=> rhs.name),
										 decltype(value <=> rhs.value)> {
		auto cmp1 = name <=> rhs.name;
		if (cmp1 != 0) return cmp1; // used as or converted to common comparison type
		return value <=> rhs.value; // used as or converted to common comparison type
	}
};
\end{cpp}

通过使用尾部返回类型语法(auto在前面,返回类型在->后面),可以使用参数来计算比较类型。这种情况下,可以只使用name,而非rhs.name,这种方法通常是有效的(例如,也适用于独立函数)。

若希望提供比内部使用的类别更强的类别,则必须将内部比较的所有可能值映射到返回类型的值。若不能映射某些值,这可能包括一些错误处理。例如:

\begin{cpp}
class Person {
	std::string name;
	double value;
	...
	std::strong_ordering operator<=> (const Person& rhs) const {
		auto cmp1 = name <=> rhs.name;
		if (cmp1 != 0) return cmp1; // return strong_ordering for std::string
		auto cmp2 = value <=> rhs.value; // might be partial_ordering for double
		// map partial_ordering to strong_ordering:
		assert(cmp2 != std::partial_ordering::unordered); // RUNTIME ERROR if unordered
		return cmp2 == 0 ? std::strong_ordering::equal
		                 : cmp2 > 0 ? std::strong_ordering::greater
		                            : std::strong_ordering::less;
	}
};
\end{cpp}

C++标准库为此提供了一些辅助函数对象。例如,映射浮点值,可以使用std::strong\_order()来比较两个值:

\begin{cpp}
class Person {
	std::string name;
	double value;
	...
	std::strong_ordering operator<=> (const Person& rhs) const {
		auto cmp1 = name <=> rhs.name;
		if (cmp1 != 0) return cmp1; // return strong_ordering for std::string
		// map floating-point comparison result to strong ordering:
		return std::strong_order(value, rhs.value);
	}
};
\end{cpp}

如若可能,std::strong\_order()会根据传入的参数产生一个std::strong\_ordering值,如下所示:

\begin{itemize}
\item
对传递的类型使用strong\_order(val1, val2)(若已定义)

\item
否则,若传递的值是浮点类型,则使用ISO/IEC/IEEE 60559中指定的totalOrder()的值(例如,-0小于+0,-NaN小于任何非NaN值和+NaN)

\item
若是为传递的类型定义的,可使用新的函数对象std::compare\_three\_way\{\}(val1, val2),
\end{itemize}

这为浮点类型提供强排序的最简单方法,即使在运行时,若一个或两个操作数可能具有值NaN,这种方法也可以工作。

std::compare\_three\_way是用于调用<=>操作符的新函数对象类型,就像std::less是用于小于操作符的函数对象类型一样。

对于其他具有较弱排序和==和<操作符定义的类型,可以使用函数对象std::compare\_strong\_order\_fallback():

\begin{cpp}
class Person {
	std::string name;
	SomeType value;
	...
	std::strong_ordering operator<=> (const Person& rhs) const {
		auto cmp1 = name <=> rhs.name;
		if (cmp1 != 0) return cmp1; // return strong_ordering for std::string
		// map weak/partial comparison result to strong ordering:
		return std::compare_strong_order_fallback(value, rhs.value);
	}
};
\end{cpp}

用于映射比较类别类型的表,函数对象列出了用于映射比较类型的所有的辅助函数。

要为泛型类型定义<=>操作符,还应该考虑使用函数对象std::compare\_three\_way或算法std::lexicographical\_compare\_three\_way()。
