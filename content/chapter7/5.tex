
本节列出了C++20中ranges库提供的所有新的辅助类型函数和工具。

至于泛型帮助函数,在C++20之前就已经存在了一些这样的工具,其在命名空间std中具有相同的名称或略有不同的名称(并且仍然提供向后兼容性),但范围工具通常为指定的功能提供更好的支持。他们可能会修复旧版本的缺陷,或者使用一些概念来限制旧版本的使用。

\mySubsubsection{7.5.1}{范围的泛型类型}

表“使用范围时产生所涉及类型的泛型函数”列出了范围的所有泛型类型定义,其被定义为别名模板。

\begin{longtable}[c]{|l|l|}
\hline
\textbf{函数类型} &
\textbf{意义} \\ \hline
\endfirsthead
%
\endhead
%
\begin{tabular}[c]{@{}l@{}}std::ranges::iterator\_t\textless{}Rg\textgreater\\ std::ranges::sentinel\_t\textless{}Rg\textgreater\\ std::ranges::range\_value\_t\textless{}Rg\textgreater\\ std::ranges::range\_reference\_t\textless{}Rg\textgreater\\ std::ranges::range\_difference\_t\textless{}Rg\textgreater\\ std::ranges::range\_size\_t\textless{}Rg\textgreater\\ std::ranges::range\_rvalue\_reference\_t\textless{}Rg\textgreater\\ std::ranges::borrowed\_iterator\_t\textless{}Rg\textgreater\\ std::ranges::borrowed\_subrange\_t\textless{}Rg\textgreater{}\end{tabular} &
\begin{tabular}[c]{@{}l@{}}在Rg上迭代的迭代器类型(begin()的结果)\\ Rg的end迭代器类型(end()的结果)\\ 范围内元素的类型\\ 元素类型引用的类型\\ 两个迭代器差的类型\\ size()函数返回的类型\\ 元素类型的右值引用的类型\\ 对于借来的范围为std::ranges::iterator\_t\textless{}Rg\textgreater ,否则为std::ranges::dangling\\ 借来范围的迭代器类型的子范围类型,否则为std::ranges::dangling\end{tabular} \\ \hline
\end{longtable}

\begin{center}
表7.7 使用范围时产生所涉及类型的泛型函数
\end{center}

正如在范围介绍中提到的,这些类型函数的主要优点是它们一般适用于所有类型的范围和视图。这甚至适用于原始数组。

注意,std::ranges::range\_value\_t<Rg>只是以下操作的快捷方式:

\begin{cpp}
std::iter_value_t<std::ranges::iterator_t<Rg>>
\end{cpp}

应该优先于Rg::value\_type。

\mySubsubsection{7.5.2}{迭代器的泛型类型}

ranges库还为迭代器引入了新的类型特征,这些特征不是在命名空间std::范围中定义的,而是在命名空间std中定义的。

表“泛型函数在使用迭代器时产生的类型”列出了迭代器的所有泛型类型特征,其定义为别名模板。


\begin{longtable}[c]{|l|l|}
\hline
\textbf{函数类型} &
\textbf{意义} \\ \hline
\endfirsthead
%
\endhead
%
\begin{tabular}[c]{@{}l@{}}std::iter\_value\_t\textless{}It\textgreater\\ std::iter\_reference\_t\textless{}It\textgreater\\ std::iter\_rvalue\_reference\_t\textless{}It\textgreater\\ std::iter\_common\_reference\_t\textless{}It\textgreater\\ std::iter\_difference\textless{}It\textgreater{}\end{tabular} &
\begin{tabular}[c]{@{}l@{}}迭代器引用的值的类型\\ 值类型引用的类型\\ 值类型的右值引用的类型\\ 引用类型的公共类型和对值类型的引用\\ 两个迭代器差的类型\end{tabular} \\ \hline
\end{longtable}

\begin{center}
表7.8 泛型函数在使用迭代器时产生的类型 iterators
\end{center}

注意,由于对新的迭代器类别有更好的支持,更推荐这些工具,而非传统的迭代器特征(std::iterator\_traits<>)。

这些函数将在关于迭代器特性的一节中详细描述。

\mySubsubsection{7.5.3}{新函数类型}

表“新函数类型”列出了可作为辅助函数使用的新类型,在<functional>中定义。

\begin{longtable}[c]{|l|l|}
\hline
\textbf{函数类型} &
\textbf{意义} \\ \hline
\endfirsthead
%
\endhead
%
\begin{tabular}[c]{@{}l@{}}std::identity\\ std::compare\_three\_way\end{tabular} &
\begin{tabular}[c]{@{}l@{}}返回自身的函数对象\\ 调用操作符\textless{}=\textgreater{}的函数对象\end{tabular} \\ \hline
\end{longtable}

\begin{center}
表7.9 新函数类型
\end{center}

若算法支持传递投影,函数对象std::identity通常用于在可以传递投影的地方不传递投影:

\begin{cpp}
auto pos = std::ranges::find(coll, 25, // find value 25
							[](auto x) {return x*x;}); // for squared elements
\end{cpp}

使用std::identity允许程序员传递“无投影”:

\begin{cpp}
auto pos = std::ranges::find(coll, 25, // find value 25
							std::identity{}); // for elements as they are
\end{cpp}

该函数对象用作默认模板形参,用于声明可以跳过投影的算法:

\begin{cpp}
template<std::ranges::input_range R,
		typename T,
		typename Proj = std::identity>
constexpr std::ranges::borrowed_iterator_t<R>
find(R&& r, const T& value, Proj proj = {});
\end{cpp}

函数对象类型std::compare\_three\_way是一个新的函数对象类型,用于指定调用/使用新的操作符<=>(就像std::less或std::ranges::less表示调用操作符<)。在<=>操作符的章节中进行了描述。

\mySubsubsection{7.5.4}{处理迭代器的其他新类型}

“迭代器的其他新类型”表列出了用于处理迭代器的所有新类型,在<iterator>中定义。


\begin{longtable}[c]{|l|l|}
\hline
\textbf{函数类型} &
\textbf{意义} \\ \hline
\endfirsthead
%
\endhead
%
\begin{tabular}[c]{@{}l@{}}std::incrementable\_traits\textless{}It\textgreater\\ std::projected\textless{}It1, It2\textgreater{}\end{tabular} &
\begin{tabular}[c]{@{}l@{}}产生两个迭代器的difference\_type的辅助类型\\ 为投影制定约束的类型\end{tabular} \\ \hline
\end{longtable}

\begin{center}
表7.10 迭代器的其他新类型
\end{center}

还要注意新的迭代器和哨兵类型。


















