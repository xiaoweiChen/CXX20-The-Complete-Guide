迭代器有不同的功能,这些功能很重要,因为有些算法需要特殊的迭代器。例如,排序算法需要能够执行随机访问的迭代器,否则,性能就会很差。由于这个原因,迭代器有不同的类别,C++20中扩展了一个新的类别:连续迭代器。这些类别的功能列在表Iterator类别中。

\begin{longtable}[c]{|l|l|l|}
\hline
\textbf{迭代器类别} &
\textbf{功能} &
\textbf{供应者} \\ \hline
\endfirsthead
%
\endhead
%
\begin{tabular}[c]{@{}l@{}}Output\\ Input\\ Forward\\ Bidirectional\\ Random-access\\ Contiguous\end{tabular} &
\begin{tabular}[c]{@{}l@{}}向前写入\\ 向前读一次\\ 向前读取\\ 向前和向后读取\\ 随机读取\\ 读取存储在连续内存中的元素\end{tabular} &
\begin{tabular}[c]{@{}l@{}}osteram迭代器, 插入器\\ istream迭代器\\ std::forward\_list\textless{}\textgreater{},无序容器\\ list\textless{}\textgreater{}, set\textless{}\textgreater{}, multset\textless{}\textgreater{}, map\textless{}\textgreater{}, multimap\textless{}\textgreater\\ deque\textless{}\textgreater\\ array\textless{}\textgreater{}, vector\textless{}\textgreater{}, string, C风格数组\end{tabular} \\ \hline
\end{longtable}

\begin{center}
表7.1 迭代器类别
\end{center}

注意,在C++20中,类别的细节发生了变化。最重要的变化是:

\begin{itemize}
\item
新的迭代器类别是连续的。对于这个类别,定义了一个新的迭代器标记类型:std::contiguous\_iterator\_tag。

\item
产生临时对象(右值)的迭代器可以拥有比输入迭代器更强的类别,生成引用的要求不再适用于前向迭代器。

\item
输入迭代器不再保证是可复制的,应该只可移动。

\item
输入迭代器的后置自增操作符不再需要产生任何结果。对于输入迭代器pos,不应再使用pos++,可使用++pos代替。
\end{itemize}

这些更改向后不兼容:

\begin{itemize}
\item
对于std::vector或数组等容器的迭代器,不再检查迭代器是否为随机访问迭代器。所以,开发者必须检查迭代器是否支持随机访问。

\item
前向迭代器、双向迭代器或随机访问迭代器的值是引用的假设不再总是适用。
\end{itemize}

出于这个原因,C++20引入了一个新的可选迭代器属性iterator\_concept,并定义可以设置该属性来表示一个不同于传统类别的C++20类别。例如:

\begin{cpp}
std::vector vec{1, 2, 3, 4};
auto pos1 = vec.begin();
decltype(pos1)::iterator_category // type std::random_access_iterator_tag
decltype(pos1)::iterator_concept // type std::contiguous_iterator_tag

auto v = std::views::iota(1);
auto pos2 = v.begin();
decltype(pos2)::iterator_category // type std::input_iterator_tag
decltype(pos2)::iterator_concept // type std::random_access_iterator_tag
\end{cpp}

注意,std::iterator\_traits没有提供成员iterator\_concept,对于迭代器,成员iterator\_category可能并不总有定义。

迭代器概念和范围概念考虑了新的C++20分类。对于必须处理迭代器类别的代码,C++20起会进行如下处理:

\begin{itemize}
\item
使用迭代器概念和范围概念来检查类别,而不是std::iterator\_traits<I>::iterator\_category。

\item
若实现了自己的迭代器类型,可参考\url{http://wg21.link/p2259}中的提供的iterator\_category和/或iterator\_concept。
\end{itemize}

还要注意,有效的C++20输入迭代器可能根本不是C++17迭代器(例如,不提供复制)。对于这些迭代器,传统的迭代器特性不起作用。由于这个原因,从C++20开始:

\begin{itemize}
\item
使用std::iter\_value\_t

代替iterator\_traits<>::value\_type.

\item
使用std::iter\_reference\_t

代替iterator\_traits<>::reference.

\item
使用std::iter\_difference\_t

代替iterator\_traits<>::difference\_type.
\end{itemize}





















