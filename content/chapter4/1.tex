To specify requirements for generic parameters you need constraints, which are used at compile time to decide whether to instantiate and compile a template or not.

You can constrain function templates, class templates, variable templates, and alias templates.

An ordinary constraint is usually specified by using a requires clause. For example:

\begin{cpp}
template<typename T>
void foo(const T& arg)
requires MyConcept<T>
...
\end{cpp}

In front of a template parameter or auto you can also use a concept as a type constraint directly:

\begin{cpp}
template<MyConcept T>
void foo(const T& arg)
...
\end{cpp}

Alternatively:

\begin{cpp}
void foo(const MyConcept auto& arg)
...
\end{cpp}












