requires子句使用关键字requires和编译时布尔表达式来限制模板的可用性。布尔表达式可以是:

\begin{itemize}
\item
特别的布尔编译时表达式

\item
一个概念

\item
一个requires表达式
\end{itemize}

可以使用布尔表达式的地方都可以使用有约束(特别是以if constexpr作为条件的)。

\mySubsubsection{4.2.1}{在requires字句中使用\&\&和||}

要在requires子句中组合多个约束,可以使用运算符\&\&。例如:

\begin{cpp}
template<typename T>
requires (sizeof(T) > 4) // ad-hoc Boolean expression
			&& requires { typename T::value_type; } // requires expression
			&& std::input_iterator<T> // concept
void foo(T x) {
	...
}
\end{cpp}

约束的顺序无关紧要。

还可以使用运算符||表示“可选”约束。例如:

\begin{cpp}
template<typename T>
requires std::integral<T> || std::floating_point<T>
T power(T b, T p);
\end{cpp}

很少需要指定可选约束,也不应该随意指定,因为在requires子句中过度使用运算符||可能会增加编译资源的负担(使编译明显变慢)。

单个约束还可以涉及多个模板参数,约束可以在多个类型(或值)之间施加影响。例如:

\begin{cpp}
template<typename T, typename U>
requires std::convertible_to<T, U>
auto f(T x, U y) {
	...
}
\end{cpp}

操作符\&\&和||是唯一可以用来组合多个约束而不必使用括号的操作符。对于其他所有内容,使用括号(正式地将一个特殊的布尔表达式传递给requires子句)。






























