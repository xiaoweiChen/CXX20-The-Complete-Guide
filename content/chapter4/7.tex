
两个概念可以有一个包含关系,可以指定一个概念,使其限制一个或多个其他概念。这样做的好处是,当两个约束都得到满足时,重载解析更倾向于使用约束较多的泛型代码,而不是使用约束较少的泛型代码。

例如,假设引入以下两个概念:

\begin{cpp}
template<typename T>
concept GeoObject = requires(T obj) {
	{ obj.width() } -> std::integral;
	{ obj.height() } -> std::integral;
	obj.draw();
};

template<typename T>
concept ColoredGeoObject =
	GeoObject<T> && // subsumes concept GeoObject
	requires(T obj) { // additional constraints
		obj.setColor(Color{});
		{ obj.getColor() } -> std::convertible_to<Color>;
	};
\end{cpp}

因为它显式地规定了类型T也必须满足概念GeoObject的约束,所以概念ColoredGeoObject显式地包含了概念GeoObject。

当为两个概念重载模板并且两个概念都满足时,不会得到歧义错误,重载解析更倾向于包含其他概念的概念:

\begin{cpp}
template<GeoObject T>
void process(T) // called for objects that do not provide setColor() and getColor()
{
	...
}

template<ColoredGeoObject T>
void process(T) // called for objects that provide setColor() and getColor()
{
	...
}
\end{cpp}

约束包含仅在使用概念时起作用。当一个概念/约束比另一个更特殊时,不存在自动包含。

约束和概念不会仅仅基于需求进行包含:

\begin{cpp}
// declared in a header file:
template<typename T>
concept GeoObject = requires(T obj) {
						obj.draw();
					};

// declared in another header file:
template<typename T>
concept Cowboy = requires(T obj) {
					obj.draw();
					obj = obj;
				};
\end{cpp}

假设为GeoObject和Cowboy重载函数模板:

\begin{cpp}
template<GeoObject T>
void print(T) {
	...
}

template<Cowboy T>
void print(T) {
	...
}
\end{cpp}

对于Circle或Rectangle(都有draw()成员函数),我们不希望这样,因为Cowboy的概念更特殊,对print()的调用更倾向于对Cowboy的print()调用,所以希望看到在本例中有两个可能的print()函数发生冲突。

检查假设的工作只针对概念进行评估。若没有使用概念,则具有不同约束的重载不明确:

\begin{cpp}
template<typename T>
requires std::is_convertible_v<T, int>
void print(T) {
	...
}

template<typename T>
requires (std::is_convertible_v<T, int> && sizeof(int) >= 4)
void print(T) {
	...
}

print(42); // ERROR: ambiguous (if both constraints are true)
\end{cpp}

当使用概念时,下面的代码可以工作:

\begin{cpp}
template<typename T>
requires std::convertible_to<T, int>
void print(T) {
	...
}

template<typename T>
requires (std::convertible_to<T, int> && sizeof(int) >= 4)
void print(T) {
	...
}

print(42); // OK
\end{cpp}

产生这种行为的一个原因是,处理概念之间的依赖关系需要更多编译的时间。

C++标准库提供的概念经过精心设计,以便在有意义时包含其他概念。事实上,标准概念构建了一个相当复杂的包含图。例如:

\begin{itemize}
\item
std::random\_access\_range包含了std::bidirectional\_range,两者都包含了std::forward\_range,三者都包含了std::input\_range,所以都包含了std::range。

但是,std::sized\_range只包含std::range,而不包含其他的概念。

\item
std::regular包含了std::semiregular,而两者都包含了std::copyable和std::default\_initializable(其中包含了其他几个概念,如std::movable、std::copy\_constructible和std::destructible)。

\item
std::sortable包含std::permutable,两者都包含std::indirectly\_swappable,这两个参数都是相同的类型。
\end{itemize}

\mySubsubsection{4.7.1}{间接包含}

约束甚至可以间接包含,所以重载解析仍然可以选择一个重载或特化,而非另一个(即使其约束不是根据彼此定义的)。

例如,假设已经定义了以下两个概念:

\begin{cpp}
template<typename T>
concept RgSwap = std::ranges::input_range<T> && std::swappable<T>;

template<typename T>
concept ContCopy = std::ranges::contiguous_range<T> && std::copyable<T>;
\end{cpp}

当为这两个概念重载两个函数,并传递一个同时适合这两个概念的对象时,这并不具有二义性:

\begin{cpp}
template<RgSwap T>
void foo1(T) {
	std::cout << "foo1(RgSwap)\n";
}

template<ContCopy T>
void foo1(T) {
	std::cout << "foo1(ContCopy)\n";
}

foo1(std::vector<int>{}); // OK: both fit, ContCopy is more constrained
\end{cpp}

原因是ContCopy包含RgSwap:

\begin{itemize}
\item
contiguous\_range概念是根据input\_range概念定义。

(包含random\_access\_range,包含bidirectional\_range,包含forward\_range,包含input\_range。)

\item
copyable概念是根据可交换的概念来定义。

(包含movable,也就包含着可互换。)
\end{itemize}

使用以下声明,当两个概念都适合时,就会产生歧义:

\begin{cpp}
template<typename T>
concept RgSwap = std::ranges::sized_range<T> && std::swappable<T>;

template<typename T>
concept ContCopy = std::ranges::contiguous_range<T> && std::copyable<T>;
\end{cpp}

原因是contiguous\_range和sized\_range这两个概念都不包含对方。

此外,对于下列声明,没有一个概念包含另一个概念:

\begin{cpp}
template<typename T>
concept RgCopy = std::ranges::input_range<T> && std::copyable<T>;

template<typename T>
concept ContMove = std::ranges::contiguous_range<T> && std::movable<T>;
\end{cpp}

一方面,ContMove更受约束,因为contiguous\_range包含input\_range。另一方面,因为copyable包含着movable,所以RgCopy更受约束。

为了避免混淆,不要对相互包含的概念做太多的假设。若有疑问,请指定所需的所有具体概念。

\mySubsubsection{4.7.2}{定义交换概念}

要正确地实现假设,必须非常谨慎。一个例子是std::same\_as概念的实现,检查两个模板形参是否具有相同的类型。

为了理解为什么定义概念并不简单,假设定义了概念SameAs:

\begin{cpp}
template<typename T, typename U>
concept SameAs = std::is_same_v<T, U>; // define concept SameAs
\end{cpp}

这个定义对于这样的情况足够了:

\begin{cpp}
template<typename T, typename U>
concept SameAs = std::is_same_v<T, U>; // define concept SameAs

template<typename T, typename U>
requires SameAs<T, U> // use concept SameAs
void foo(T, U)
{
	std::cout << "foo() for parameters of same type" << '\n';
}

template<typename T, typename U>
requires SameAs<T, U> && std::integral<T> // use concept SameAs again
void foo(T, U)
{
	std::cout << "foo() for integral parameters of same type" << '\n';
}

foo(1, 2); // OK: second foo() preferred
\end{cpp}

这里,foo()的第二个定义包含了第一个定义,当传递了两个相同整型的参数,并且两个foo()的约束都满足时,会调用第二个foo()。

但当在第二个foo()的约束中调用SameAs<>时,若稍微修改参数会发生什么:

\begin{cpp}
template<typename T, typename U>
requires SameAs<T, U> // use concept SameAs
void foo(T, U)
{
	std::cout << "foo() for parameters of same type" << '\n';
}

template<typename T, typename U>
requires SameAs<U, T> && std::integral<T> // use concept SameAs with other order
void foo(T, U)
{
	std::cout << "foo() for integral parameters of same type" << '\n';
}

foo(1, 2); // ERROR: ambiguity: both constraints are satisfied without one subsuming the other
\end{cpp}

问题是编译器无法检测到SameAs<>是可交换的。对于编译器来说,模板参数的顺序很重要,所以第一个需求不一定是第二个需求的子集。

为了解决这个问题,必须以一种不影响参数顺序的方式来设计SameAs概念,这里需要一个辅助概念:

\begin{cpp}
template<typename T, typename U>
concept SameAsHelper = std::is_same_v<T, U>;

template<typename T, typename U>
concept SameAs = SameAsHelper<T, U> && SameAsHelper<U, T>; // make commutative
\end{cpp}

对于IsSame<>,参数的顺序不再重要:

\begin{cpp}
template<typename T, typename U>
requires SameAs<T, U> // use SameAs
void foo(T, U)
{
	std::cout << "foo() for parameters of same type" << '\n';
}

template<typename T, typename U>
requires SameAs<U, T> && std::integral<T> // use SameAs with other order
void foo(T, U)
{
	std::cout << "foo() for integral parameters of same type" << '\n';
}

foo(1, 2); // OK: second foo() preferred
\end{cpp}

编译器可以发现第一个构建块SameAs<U,T>是SameAs定义的子概念的一部分,因此其他构建块SameAs<T,U>和std::integral<T>是扩展,所以第二个foo()更优先使用。

C++20标准库提供的概念(请参见std::same\_as)中就包含了类似这样的细节,所以应该在适合需求时使用,而不是定义自己的概念。

\begin{cpp}
template<typename T, typename U>
requires std::same_as<T, U> // standard same_as<> is commutative
void foo(T, U)
{
	std::cout << "foo() for parameters of same type" << '\n';
}

template<typename T, typename U>
requires std::same_as<U, T> && std::integral<T> // so different order does not matter
void foo(T, U)
{
	std::cout << "foo() for integral parameters of same type" << '\n';
}

foo(1, 2); // OK: second foo() preferred
\end{cpp}

对于自定义的概念,请考虑尽可能更多的使用方法(和滥用),越多越好。就像任何好软件一样,概念也需要好的设计和测试用例。












