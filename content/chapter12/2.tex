
C++20不仅为线程提供了停止令牌。它是一种通用机制,可以异步请求停止,并使用各种方式对该请求作出响应。

其机制如下所示:

\begin{itemize}
\item 
C++20标准库可创建一个共享的停止状态。默认情况下,停止状态不会发出信号。

\item 
类型为std::stop\_source的停止源可以在其关联的共享停止状态下请求停止。

\item 
std::stop\_token类型的停止令牌可用于,在其关联的共享停止状态下响应停止请求。可以主动轮询是否请求了停止,或者注册一个类型为std::stop\_callback的回调,该回调将在请求停止时调用。

\item 
当停止请求发出,就不能撤销(后续的停止请求无效)。

\item 
可以复制和移动停止源和停止令牌,以允许代码在多个位置发出停止信号或对停止做出反应。复制源代码或令牌的成本相对较低,因此按值传递可避免生命周期问题。

然而,复制并不像传递整型值或原始指针那么便宜,更像是传递一个共享指针。若经常将其传递给子函数,那么通过引用传递可能会更好。

\item 
该机制是线程安全的,可以在并发情况下使用。停止请求、检查请求的停止,以及对注册或取消注册回调的调用是正确同步的,并且当最后一个用户(停止源、停止令牌或停止回调)被销毁时,关联的共享停止状态将自动销毁。
\end{itemize}

下面的例子展示了如何建立一个停止源和一个停止令牌:

\begin{cpp}
#include <stop_token>
...

// create stop_source and stop_token:
std::stop_source ssrc; // creates a shared stop state
std::stop_token stok{ssrc.get_token()}; // creates a token for the stop state
\end{cpp}

第一步是简单地创建stop\_source对象,该对象提供了请求停止的API。构造函数还创建关联的共享停止状态。然后,可以向停止源请求stop\_token对象,该对象提供对停止请求作出反应的API(通过轮询或注册回调)。

然后,可以将令牌(和/或源)传递给位置/线程,以在可能请求停止的位置和可能对停止做出反应的位置之间建立异步通信。

没有其他方法可以创建具有关联的共享停止状态的停止令牌,停止令牌的默认构造函数没有关联的停止状态。

\mySubsubsection{12.2.1}{停止源和停止令牌的详情}

详细的了解一下停止源、停止令牌和停止回调的API,所有类型都在头文件<stop\_token>中声明。

\mySamllsection{停止源的详情}

“stop\_source类对象的操作”表列出了std::stop\_source的API。

构造函数通常在堆上为停止状态分配内存,所有停止令牌和停止回调都使用该内存。没有办法用分配器指定不同的位置。

请注意,停止源、停止令牌和停止回调之间没有生命周期约束。当使用该状态的最后一个停止源、停止令牌或停止回调销毁时,用于停止状态的内存将自动释放。

为了能够创建没有关联停止状态的停止源(这可能很有用,因为停止状态需要资源),可以用一个特殊的构造函数创建一个停止源,并在稍后分配一个停止源:

\begin{cpp}
std::stop_source ssrc{std::nostopstate}; // no associated shared stop state
...
ssrc = std::stop_source{}; // assign new shared stop state
\end{cpp}


\mySamllsection{停止令牌的详情}

“stop\_token类对象的操作表”列出了std::stop\_token的API。

请注意,stop\_possible()会产生是否仍然可以停止的错误信号。其会在两种情况下产生false:

\begin{itemize}
\item 
若没有关联的停止状态

\item 
若存在停止状态,但不再有停止源,并且从未请求过停止
\end{itemize}

这可以用来避免定义反应停止(永远不会发生)。

% Please add the following required packages to your document preamble:
% \usepackage{longtable}
% Note: It may be necessary to compile the document several times to get a multi-page table to line up properly
\begin{longtable}[c]{|l|l|}
\hline
\textbf{操作}            & \textbf{效果}                                                                     \\ \hline
\endfirsthead
%
\endhead
%
stop\_source s                & 默认构造函数;创建具有关联停止状态的停止源               \\ \hline
stop\_source s\{nostopstate\} & 创建没有关联停止状态的停止源                                 \\ \hline
stop\_source s\{s2\}          & 复制构造函数;创建一个停止源,该停止源共享s2的关联停止状态  \\ \hline
stop\_source s\{move(s2)\} & \begin{tabular}[c]{@{}l@{}}移动构造函数;创建一个获取s2的关联停止状态的停止源 \\ (s2不再具有关联的停止状态) \end{tabular} \\ \hline
s.$\sim$stop\_source()        & 析构函数;若这是相关的共享停止状态的最后一个用户,则销毁 \\ \hline
s = s2 &
\begin{tabular}[c]{@{}l@{}}复制赋值;复制赋值s2的状态,这样s现在也共享了s2的停止状态\\(s之前的停止状态都释放了) \end{tabular} \\ \hline
s = move(s2) &
\begin{tabular}[c]{@{}l@{}}移动赋值:移动赋值s2的状态,使s现在共享s2的停止状态\\(s2不再有停止状态,s之前的停止状态被释放) \end{tabular} \\ \hline
s.get\_token() &
\begin{tabular}[c]{@{}l@{}}为关联的停止状态生成一个stop\_token\\(若没有可共享的停止状态,则返回一个没有关联的停止状态的停止令牌) \end{tabular} \\ \hline
s.request\_stop() &
若其中任何一个尚未完成(返回是否请求了停止),则相关的停止状态下请求停止 \\ \hline
s.stop\_possible()            & 生成s是否具有关联的停止状态                                       \\ \hline
s.stop\_requested()           & 生成s是否具有请求停止的关联停止状态        \\ \hline
s1 == s2                      & 生成s1和s2是否共享相同的停止状态(或者两者都不共享)             \\ \hline
s1 != s2                      & 生成s2和s2是否共享相同的停止状态                           \\ \hline
s1.swap(s2)                   & 交换s1和s2的状态                                                       \\ \hline
swap(s1, s2)                  & 交换s1和s2的状态                                                       \\ \hline
\end{longtable}

\begin{center}
表12.2 类stop\_source对象的操作
\end{center}

\mySubsubsection{12.2.2}{使用停止回调}

停止回调是RAII类型std::stop\_callback的对象。构造函数注册一个可调用对象(函数、函数对象或Lambda),以便在为指定的停止令牌请求停止时调用:

\begin{cpp}
void task(std::stop_token st)
{
	// register temporary callback:
	std::stop_callback cb{st, []{
			std::cout << "stop requested\n";
			...
	}};
	...
} // unregisters callback
\end{cpp}

% Please add the following required packages to your document preamble:
% \usepackage{longtable}
% Note: It may be necessary to compile the document several times to get a multi-page table to line up properly
\begin{longtable}[c]{|l|l|}
\hline
\textbf{操作}     & \textbf{效果}                                                                          \\ \hline
\endfirsthead
%
\endhead
%
stop\_token t          & 默认构造函数;创建没有关联停止状态的停止令牌                  \\ \hline
stop\_token t\{t2\}    & 复制构造函数;创建一个停止令牌,共享t2的相关停止状态       \\ \hline
stop\_token t\{move(t2)\} &
\begin{tabular}[c]{@{}l@{}}移动构造函数;创建一个停止令牌,获取t2的关联停止状态\\(t2不再具有关联停止状态) \end{tabular} \\ \hline
t.$\sim$stop\_token()  & 析构函数;若这是相关的共享停止状态的最后一个用户,则销毁      \\ \hline
t = t2 &
\begin{tabular}[c]{@{}l@{}}复制赋值;复制分配t2的状态,使t现在也共享t2的停止状态\\(释放任何以前的t停止状态) \end{tabular}\\ \hline
t = move(t2) &
\begin{tabular}[c]{@{}l@{}}移动赋值;移动赋值t2的状态,使t现在共享t2的停止状态\\(t2不再具有停止状态,并且释放t之前的停止状态) \end{tabular}\\ \hline
t.stop\_possible()     & 生成的t是否具有关联的停止状态,以及是否已经或仍可以请求停止 \\ \hline
t.stop\_requested()    & 生成t是否具有请求停止的关联停止状态             \\ \hline
t1 == t2               & 输出t1和t2是否共享相同的停止状态(或者两者都不共享)                  \\ \hline
t1 != t2               & 生成t1和t2是否共享相同的停止状态                                \\ \hline
t1.swap(t2)            & 交换t1和t2的状态                                                            \\ \hline
swap(t1. t2)           & 交换t1和t2的状态                                                            \\ \hline
stop\_token cb\{t, f\} & 将cb注册为t调用f的停止回调                                             \\ \hline
\end{longtable}

\begin{center}
表12.3 类stop\_token对象的操作
\end{center}

假设已经创建了共享停止状态,并创建了一个异步情况,其中一个线程可能请求停止,另一个线程可能运行task()。可以用下面的代码来模拟这种情况:

\begin{cpp}
// create stop_source with associated stop state:
std::stop_source ssrc;

// register/start task() and pass the corresponding stop token to it:
registerOrStartInBackgound(task, ssrc.get_token());
...
\end{cpp}

函数registerOrStartInBackgound()可以立即开始task()。或启动task()后通过调用std::async()初始化一个std::thread,称之为协程,或注册一个事件处理程序。

现在,每当我们请求停止时:

\begin{cpp}
ssrc.request_stop();
\end{cpp}

可能发生以下情况之一:

\begin{itemize}
\item 
若task()在初始化回调时已经启动,并且仍在运行,并且还没有调用回调的析构函数,则在调用request\_stop()的线程中立即调用已注册的可调用对象。request\_stop()阻塞,直到所有注册的可调用对象调用,且调用的顺序没有定义。

\item
若task()在初始化回调时已经启动,并且仍在运行,并且还没有调用回调的析构函数,则在调用request\_stop()的线程中立即调用已注册的可调用对象。请求\_stop()阻塞,直到所有注册的可调用对象被调用,且调用的顺序没有定义。

\item
若task()已经完成(或者至少已经调用了回调函数的析构函数),则永远不会调用可调用对象。回调生命周期的结束表明不再需要调用可调用对象。
\end{itemize}

这些场景都是精心同步的。若正在初始化一个stop\_callback,以便注册可调用对象,就会发生上述情况之一。若在可调用对象由于破坏了停止回调而未注册时请求停止,则同样适用。若可调用对象已经用另一个线程启动,则析构函数阻塞,直到可调用对象完成。

对于编程逻辑,则从初始化回调到其销毁结束的那一刻起,可能会调用已注册的可调用对象。直到构造函数结束,回调函数在初始化的线程中运行;之后,在请求停止的线程中运行。请求停止的代码可能会立即调用已注册的可调用对象,也可能会稍后调用(若回调稍后初始化),也可能永远不会调用(若调用回调为时已晚)。

例如,下面的代码:

\filename{lib/stop.cpp}

\begin{cpp}
#include <iostream>
#include <stop_token>
#include <future> // for std::async()
#include <thread> // for sleep_for()
#include <syncstream> // for std::osyncstream
#include <chrono>
using namespace std::literals; // for duration literals

auto syncOut(std::ostream& strm = std::cout) {
	return std::osyncstream{strm};
}

void task(std::stop_token st, int num)
{
	auto id = std::this_thread::get_id();
	syncOut() << "call task(" << num << ")\n";
	
	// register a first callback:
	std::stop_callback cb1{st, [num, id]{
		syncOut() << "- STOP1 requested in task(" << num
			<< (id == std::this_thread::get_id() ? ")\n"
												: ") in main thread\n");
	}};
	std::this_thread::sleep_for(9ms);
	
	// register a second callback:
	std::stop_callback cb2{st, [num, id]{
		syncOut() << "- STOP2 requested in task(" << num
			<< (id == std::this_thread::get_id() ? ")\n"
												: ") in main thread\n");
	}};
	std::this_thread::sleep_for(2ms);
}
	
int main()
{
	// create stop_source and stop_token:
	std::stop_source ssrc;
	std::stop_token stok{ssrc.get_token()};
	
	// register callback:
	std::stop_callback cb{stok, []{
			syncOut() << "- STOP requested in main()\n" << std::flush;
	}};

	// in the background call task() a bunch of times:
	auto fut = std::async([stok] {
							for (int num = 1; num < 10; ++num) {
								task(stok, num);
							}
						});
	
	// after a while, request stop:
	std::this_thread::sleep_for(120ms);
	ssrc.request_stop();
}
\end{cpp}

注意,使用同步输出流来确保不同线程的print语句逐行同步。

例如,输出可能如下所示:

\begin{shell}
call task(1)
call task(2)
...
call task(7)
call task(8)
- STOP2 requested in task(8) in main thread
- STOP1 requested in task(8) in main thread
- STOP requested in main()
call task(9)
- STOP1 requested in task(9)
- STOP2 requested in task(9)
\end{shell}


或者也可能是这样的:

\begin{shell}
call task(1)
call task(2)
call task(3)
call task(4)
- STOP2 requested in task(4) in main thread
call task(5)
- STOP requested in main()
- STOP1 requested in task(5)
- STOP2 requested in task(5)
call task(6)
- STOP1 requested in task(6)
- STOP2 requested in task(6)
call task(7)
- STOP1 requested in task(7)
- STOP2 requested in task(7)
...
\end{shell}

或者它看起来像这样:

\begin{shell}
call task(1)
call task(2)
call task(3)
call task(4)
- STOP requested in main()
call task(5)
- STOP1 requested in task(5)
- STOP2 requested in task(5)
call task(6)
- STOP1 requested in task(6)
- STOP2 requested in task(6)
...
\end{shell}

它甚至可能只是:

\begin{shell}
call task(1)
call task(2)
...
call task(8)
call task(9)
- STOP requested in main()
\end{shell}

若不使用syncOut(),输出甚至可能有交错字符,因为主线程和运行task()的线程的输出可能完全混合在一起。

\mySamllsection{停止回调的详情}

停止回调的类型,stop\_callback,是一个具有非常有限API的类模板。实际上,只提供了一个构造函数来为停止令牌注册可调用对象,并提供了一个析构函数来取消可调用对象的注册。删除复制和移动,不提供其他成员函数。

模板形参是可调用对象的类型,通常在初始化构造函数时推导出来:

\begin{cpp}
auto func = [] { ... };

std::stop_callback cb{myToken, func}; // deduces stop_callback<decltype(func)>
\end{cpp}

除了构造函数和析构函数之外,唯一的公共成员是callback\_type,存储的可调用对象的类型。

构造函数接受左值(有名字的对象)和右值(临时对象或用std::move()标记的对象):

\begin{cpp}
auto func = [] { ... };

std::stop_callback cb1{myToken, func}; // copies func
std::stop_callback cb2{myToken, std::move(func)}; // moves func
std::stop_callback cb3{myToken, [] { ... }}; // moves the lambda
\end{cpp}

\mySubsubsection{12.2.3}{停止令牌的限制和保证}

对于可能发生在异步上下文中的几个场景,处理停止请求的特性是相当健壮的,但并不能避免所有的陷阱。

C++20标准库保证了以下内容:

\begin{itemize}
\item 
所有的request\_stop()、stop\_requested()和stop\_possible()调用都是同步的。

\item 
保证自动执行回调注册。若另一个线程并发调用request\_stop(),那么当前线程将看到请求停止并立即调用当前线程上的回调,或者另一个线程将看到回调注册并在从request\_stop()返回之前调用回调。

\item 
stop\_callback的可调用对象保证在stop\_callback的析构函数返回后不会调用。

\item 
若回调函数的析构函数刚刚在另一个线程调用,则它等待可调用对象完成(它不等待其他可调用对象完成)。
\end{itemize}

然而,请注意以下限制:

\begin{itemize}
\item 
回调不应该抛出异常,若对其可调用对象的调用通过异常退出,则调用std::terminate()。

\item 
不要在回调的可调用对象中销毁回调,析构函数不会等待回调完成的。
\end{itemize}
