
C++20为聚合提供了一些改进。对于泛型代码,现在有:

\begin{itemize}
\item
对聚合使用类模板参数推导(CTAD)

\item
聚合可以用作非类型模板参数(NTTP)。
\end{itemize}

本节将介绍前者。

聚合还有其他新特性:

\begin{itemize}
\item
(部分)支持指定初始化式(特定成员的初始值)

\item
可以用圆括号初始化聚合

\item
聚合的固定定义和std::is\_default\_constructible<>
\end{itemize}

\mySubsubsection{22.2.1}{聚合的类模板参数推导(CTAD)}

从C++17开始,构造函数可用于推断类模板的模板形参。例如:

\begin{cpp}
template<typename T>
class Type {
	T value;
public:
	Type(T val)
	: value{val} {
	}
	...
};

Type<int> t1{42};
Type t2{42}; // deduces Type<int> since C++17
\end{cpp}

即使对于简单的聚合,也不支持根据对象初始化方式进行类似的推导:

\begin{cpp}
template<typename T>
struct Aggr {
	T value;
};

Aggr<int> a1{42}; // OK
Aggr a2{42}; // ERROR before C++20
\end{cpp}

必须提供一个推导指南:

\begin{cpp}
template<typename T>
struct Aggr {
	T value;
};

template<typename T>
Aggr(T) -> Aggr<T>;

Aggr<int> a1{42}; // OK
Aggr a2{42}; // OK since C++17
\end{cpp}

自C++20起,不再需要推导指南,则以下内容就足够了:

\begin{cpp}
template<typename T>
struct Aggr {
	T value;
};

Aggr<int> a1{42}; // OK
Aggr a2{42}; // OK since C++20 even without deduction guide
\end{cpp}

当使用括号初始化聚合时,此功能也有效:

\begin{cpp}
Aggr a3(42); // OK since C++20 even without deduction guide
\end{cpp}

推导规则可以很微妙:

\begin{cpp}
template<typename T>
struct S {
	T x;
	T y;
};

template<typename T>
struct C {
	S<T> s;
	T x;
	T y;
};

C c1 = {{1, 2}, 3, 4}; // OK, C<int> deduced
C c2 = {{1, 2}, 3}; // OK, C<int> deduced (y is 0)
C c3 = {{1, 2}, 3.3, 4.4}; // OK, C<double> deduced

C c4 = {{1, 2}, 3, 4.4}; // ERROR: int and double deduced for T
C c5 = {{1, 2}}; // ERROR: T only indirectly deduced
C c6 = {1, 2, 3}; // ERROR: don’t know how many values S<T> needs
C<int> c7 = {1, 2, 3}; // OK
\end{cpp}

类模板参数推导甚至适用于具有可变数量元素的聚合:

\begin{cpp}
// aggregate with variadic number of base types:
template<typename... T>
struct C : T... {
};

struct Base1 {
};
struct Base2 {
};

// aggregate initialized with two elements of types Base1 and Base2:
C c1{Base1{}, Base2{}};

// aggregate initialized with three lambda elements:
C c2{[] { f1(); },
	[] { f2(); },
	[] { f3(); }
};
\end{cpp}
















