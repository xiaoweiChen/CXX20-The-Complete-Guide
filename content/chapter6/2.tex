

当将范围作为单个参数传递给算法时,会遇到生命周期问题。本节描述ranges库如何处理此问题。

\mySubsubsection{6.2.1}{租借迭代器}

许多算法将迭代器返回到它们所操作的范围,当将范围作为单个参数传递时,可能会遇到一个新问题,当范围需要两个参数(开始迭代器和结束迭代器)时,这是不可能的:若传递一个临时范围(例如函数返回的范围)并返回一个迭代器给它,当范围销毁时,返回的迭代器可能会在语句结束时失效。使用返回的迭代器(或它的副本)会导致未定义的行为。

例如,考虑将一个临时范围传递给find()算法,该算法在一个范围中搜索一个值:

\begin{cpp}
std::vector<int> getData(); // forward declaration

auto pos = find(getData(), 42); // returns iterator to temporary vector
// temporary vector returned by getData() is destroyed here
std::cout << *pos; // OOPS: using a dangling iterator
\end{cpp}

getData()的返回值在使用它的语句结束时被销毁,所以pos指的是一个不再存在的集合元素。使用pos会导致未定义的行为(充其量,会得到一个段错误,这样就可以看到存在问题了)。

为了解决这个问题,ranges标准库引入了租借迭代器的概念。租借迭代器确保其生命周期不依赖于可能已销毁的临时对象,若存在,则使用它会导致编译时错误。因此,租借迭代器会发出信号,表明它是否可以安全地超过所传递的范围,若该范围不是临时的,或者迭代器的状态不依赖于所传递范围的状态,就会出现这种情况。若有一个租借迭代器来引用一个范围,那么这个迭代器可以安全使用,即使在范围销毁时也不会悬空。[因此,在ranges标准库的草案版本中,这样的迭代器被称为“安全迭代器”。]

通过使用类型std::ranges::borrowed\_iterator\_t<>,算法可以将返回的迭代器声明为借来的。所以该算法总是返回一个可以在语句后安全使用的迭代器。若可能为悬空,则使用一个特殊的返回值来发出信号,并将可能的运行时错误转换为编译时错误。

例如,单个范围的std::ranges::find()声明如下:

\begin{cpp}
template<std::ranges::input_range Rg,
			typename T,
			typename Proj = identity>
...
constexpr std::ranges::borrowed_iterator_t<Rg>
	find(Rg&& r, const T& value, Proj proj = {});
\end{cpp}

通过将返回类型指定为Rg的std::ranges::borrow \_iterator\_t<>,该标准启用了编译时检查:若传递给算法的范围R是临时对象(右值),则返回类型变为悬空迭代器。这时,返回值是std::ranges::dangling类型的对象,对此类对象的使用(复制和赋值除外)都会导致编译时错误。

因此,下面的代码会导致编译时错误:

\begin{cpp}
std::vector<int> getData(); // forward declaration

auto pos = std::ranges::find(getData(), 42); // returns iterator to temporary vector
// temporary vector returned by getData() was destroyed
std::cout << *pos; // compile-time ERROR
\end{cpp}

为了能够为临时对象调用find(),必须将其作为左值传递,所以其必须有一个名字。这样,算法确保调用集合后仍然存在。从而,还可以检查是否找到了一个值(这通常是合适的)。

为返回集合指定名称的最佳方法是将其绑定到引用,集合就永远不会被复制。请注意,根据规则,对临时对象的引用总是会延长其生命周期:

\begin{cpp}
std::vector<int> getData(); // forward declaration

reference data = getData(); // give return value a name to use it as an lvalue
// lifetime of returned temporary vector ends now with destruction of data
...
\end{cpp}

可以在这里使用两种引用:

\begin{itemize}
\item
可以声明一个const左值引用:

\begin{cpp}
std::vector<int> getData(); // forward declaration

const auto& data = getData(); // give return value a name to use it as an lvalue
auto pos = std::ranges::find(data, 42); // yields no dangling iterator
if (pos != data.end()) {
	std::cout << *pos; // OK
}
\end{cpp}

这个引用使返回值为const,这可能不是期望的(注意,当某些视图是const时,不能迭代它们;但由于视图的引用语义,在返回它们时必须非常小心)。

\item
更泛型的代码中,应该使用通用引用(也称为转发引用)或decltype(auto),以便保持返回值的“非”常量:

\begin{cpp}
... getData(); // forward declaration

auto&& data = getData(); // give return value a name to use it as an lvalue
auto pos = std::ranges::find(data, 42); // yields no dangling iterator
if (pos != data.end()) {
	std::cout << *pos; // OK
}
\end{cpp}
\end{itemize}

这个特性的副作用是,即使结果代码是有效的,也不能将临时对象传递给算法:

\begin{cpp}
process(std::ranges::find(getData(), 42)); // compile-time ERROR
\end{cpp}

尽管迭代器在函数调用期间是有效的(临时向量将在调用后销毁),find()返回一个std::ranges::dangling对象。

同样,处理此问题的最佳方法是声明getData()返回值的引用:

\begin{itemize}
\item
使用const左值引用:

\begin{cpp}
const auto& data = getData(); // give return value a name to use it as an lvalue
process(std::ranges::find(data, 42)); // passes a valid iterator to process()
\end{cpp}

\item
使用通用/转发引用:

\begin{cpp}
auto&& data = getData(); // give return value a name to use it as an lvalue
process(std::ranges::find(data, 42)); // passes a valid iterator to process()
\end{cpp}
\end{itemize}

记住,通常情况下,需要一个返回值的名称来检查返回值是否指向一个元素,而不是一个范围的end():

\begin{cpp}
auto&& data = getData(); // give return value a name to use it as an lvalue
auto pos = std::ranges::find(data, 42); // yields a valid iterator
if (pos != data.end()) { // OK
	std::cout << *pos; // OK
}
\end{cpp}

\mySubsubsection{6.2.2}{租借范围}

范围类型可以声明它们是租借的范围。当范围本身不再存在时,迭代器仍可以使用。

C++20为此提供了std::ranges::borrow \_range的概念。若range类型的迭代器从不依赖于其范围的生存期,或者传递的范围对象为左值,则满足此概念。在具体情况下,该概念检查为该范围创建的迭代器在该范围不再存在后是否可以使用。

所有引用作为右值(临时范围对象)传递的范围的标准容器和视图都不是借来的范围,因为迭代器迭代存储在其中的值。在这些情况下,有两种方法让视图成为租借范围:

\begin{itemize}
\item
迭代器存储所有用于本地迭代的信息。例如:

\begin{itemize}
\item
std::ranges::iota\_view,生成一个递增的值序列。这里,迭代器在本地存储当前值,并且不引用任何其他对象。

\item
std::ranges::empty\_view,任何迭代器总是在末尾,因此其根本不能迭代元素值。
\end{itemize}

\item
迭代器直接引用底层范围,而不使用调用begin()和end()的视图。例如:

\begin{itemize}
\item
std::ranges::subrange

\item
std::ranges::ref\_view

\item
std::span

\item
std::string\_view
\end{itemize}
\end{itemize}

注意,当借来的迭代器引用基础范围(上面的后一类),并且基础范围不再存在时,仍然可以悬空。

因此,可以在编译时捕获一些但不是所有可能的运行时错误,可以用各种方法来演示如何在不同范围内查找值为8的元素(是的,我们通常应该检查是否返回了end迭代器):

\begin{itemize}
\item
所有左值(有名字的对象)都是借来的范围,只要迭代器存在于范围的同一作用域或子作用域中,返回的迭代器就不能是悬空的。

\begin{cpp}
std::vector coll{0, 8, 15};

auto pos0 = std::ranges::find(coll, 8); // borrowed range
std::cout << *pos0; // OK (undefined behavior if no 8)

auto pos1 = std::ranges::find(std::vector{8}, 8); // yields dangling
std::cout << *pos1; // compile-time ERROR
\end{cpp}

\item
对于临时视图,视情况而定。例如:

\begin{cpp}
auto pos2 = std::ranges::find(std::views::single(8), 8); // yields dangling
std::cout << *pos2; // compile-time ERROR

auto pos3 = std::ranges::find(std::views::iota(8), 8); // borrowed range
std::cout << *pos3; // OK (undefined behavior if no 8 found)

auto pos4 = std::ranges::find(std::views::empty<int>, 8); // borrowed range
std::cout << *pos4; // undefined behavior as no 8 found
\end{cpp}

例如,单视图迭代器引用视图中元素的值;因此,单个视图不是租借范围。

另一方面,iota视图迭代器保存它们所引用的元素的副本,所以iota视图会声明为租借范围。

\item
对于引用另一个范围(作为一个整体或它的子序列)的视图,情况更加复杂。若可以,会尝试发现类似的问题。例如,适配器std::views::take()也检查右值:

\begin{cpp}
auto pos5 = std::ranges::find(std::views::take(std::vector{0, 8, 15}, 2), 8);
// compile-time ERROR
\end{cpp}

这里,调用take()会出现一个编译时错误。

然而,若使用count(),其只接受一个迭代器,开发者有责任确保迭代器的有效性:

\begin{cpp}
auto pos6 = std::ranges::find(std::views::counted(std::vector{0, 8, 15}.begin(),
							2), 8);
std::cout << *pos6; // runtime ERROR even if 8 found
\end{cpp}

这里用count()创建的视图,根据定义是租借范围,因为其将内部引用传递给它们的迭代器。换句话说:计数视图的迭代器不需要它所属的视图。然而,迭代器仍然可以引用一个不再存在的范围(因为它的视图引用了一个不再存在的对象)。示例的最后一行用pos6演示了这种情况。即使find()正在查找的值可以在临时范围内找到,仍然会出现未定义行为。

\end{itemize}

若实现了一个容器或视图,可以通过特化变量模板std::ranges::enable\_borrowing\_range<>来表示它是一个租借范围。








