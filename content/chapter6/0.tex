自第一个C++标准以来,处理容器和其他序列元素的方法一直是使用迭代器,来确定第一个元素的位置(开始)和最后一个元素后面的位置(结束)。对范围进行操作的算法通常使用两个参数来处理容器的所有元素,容器提供了begin()和end()等函数来提供这些参数。

C++20提供了一种处理范围的新方法,支持将范围和子范围定义和使用为单个对象,例如:将其作为一个整体作为单个参数传递,而不是处理两个迭代器。

这种改变听起来很简单,会产生很多后果。对于调用者和实现者来说,处理算法的方式都发生了巨大的变化,所以C
++20提供了一些处理范围的新特性和工具:


\begin{itemize}
\item
将范围作为单个参数的新重载或标准算法的变体

\item
处理范围对象的几个工具:
\begin{itemize}
\item
用于创建范围对象的辅助函数

\item
处理范围对象的辅助函数

\item
用于处理范围对象的辅助类型

\item
范围的概念
\end{itemize}

\item
轻量级范围,称为视图,用于引用具有可选值转换的范围(子集)

\item
管道是一种灵活的组合范围和视图处理的方式
\end{itemize}

本章介绍了范围和视图的基本方面和特点。





