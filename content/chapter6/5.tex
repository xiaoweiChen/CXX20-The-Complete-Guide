
在使用视图(以及一般的范围库)时,有一些关于常量性的事情令人惊讶,甚至是不正常的。

\begin{itemize}
\item
对于某些视图,当视图为const时,不可能遍历元素。

\item
视图消除了对元素常量性的传播。

\item
像cbegin()和cend()这样的函数的目标是确保元素在迭代时为const,要么没有提供,要么已破坏。
\end{itemize}

这些问题或多或少都有充分的理由,其中一些与视图的性质和所做的一些基本设计决策有关。我认为这是一个严重的设计错误,但C++标准委员会中的其他人有不同的观点。

目前正在进行修复,至少修复了cbegin()和cend()的连续性,但C++标准委员会决定不在C++20中进行应用。这些将在C++23中实现(并改变行为),详见\url{http://wg21.link/p2278r4}。

\mySubsubsection{6.5.1}{视图及其范围之间的生命周期依赖关系}

实现一个可以遍历每种类型的容器和视图的元素,并具有良好性能的泛型函数是非常复杂的。像下面这样声明一个打印所有元素的函数并不总有效:

\begin{cpp}
template<typename T>
void print(const T& coll); // OOPS: might not work for some views
\end{cpp}

考虑下面的具体例子:

\filename{ranges/printconst.cpp}

\begin{cpp}
#include <iostream>
#include <vector>
#include <list>
#include <ranges>

void print(const auto& rg)
{
	for (const auto& elem : rg) {
		std::cout << elem << ' ';
	}
	std::cout << '\n';
}

int main()
{
	std::vector vec{1, 2, 3, 4, 5, 6, 7, 8, 9};
	std::list lst{1, 2, 3, 4, 5, 6, 7, 8, 9};
	
	print(vec | std::views::take(3)); // OK
	print(vec | std::views::drop(3)); // OK
	
	print(lst | std::views::take(3)); // OK
	print(lst | std::views::drop(3)); // ERROR
	for (const auto& elem : lst | std::views::drop(3)) { // OK
		std::cout << elem << ' ';
	}
	std::cout << '\n';
	
	auto isEven = [] (const auto& val) {
		return val % 2 == 0;
	};
	print(vec | std::views::filter(isEven)); // ERROR
}
\end{cpp}

声明一个非常简单的泛型函数print(),输出作为const引用传递的范围的所有元素:

\begin{cpp}
void print(const auto& rg)
{
	...
}
\end{cpp}

然后,为几个视图调用print(),并遇到了令人惊讶的行为:


\begin{itemize}
\item
将vector传递给take视图和drop视图的效果很好:

\begin{cpp}
print(vec | std::views::take(3)); // OK
print(vec | std::views::drop(3)); // OK

print(lst | std::views::take(3)); // OK
\end{cpp}

\item
将drop视图传递给list无法编译:

\begin{cpp}
print(vec | std::views::take(3)); // OK
print(vec | std::views::drop(3)); // OK

print(lst | std::views::take(3)); // OK
print(lst | std::views::drop(3)); // ERROR
\end{cpp}

\item
然而,直接迭代drop视图却运行良好:

\begin{cpp}
for (const auto& elem : lst | std::views::drop(3)) { // OK
	std::cout << elem << ' ';
}
\end{cpp}
\end{itemize}

所以,vector不能很好地工作。例如,若传递一个过滤器视图们会得到所有范围的错误:

\begin{cpp}
print(vec | std::views::filter(isEven)); // ERROR
\end{cpp}

\mySamllsection{const\&并不适用于所有视图}

这种非常奇怪和意外的行为的原因是,当视图为const时,视图并不总是支持遍历元素。这是这样一个事实的结果:迭代这些视图的元素有时需要能够修改视图的状态(例如,由于缓存)。对于某些视图(如过滤器视图)永远不会工作,对于某些视图(例如drop视图),只是有时会起作用。

若用const声明了以下标准视图中的元素,则不能迭代:

\begin{itemize}
\item
永远不能迭代的const视图:

\begin{itemize}
\item
过滤视图

\item
丢弃段视图

\item
拆分视图

\item
IStream视图
\end{itemize}

\item
只能偶尔迭代的const视图:

\begin{itemize}
\item
丢弃视图,若引用的范围没有随机访问或没有size()

\item
反向视图,若引用的范围对于开始迭代器和前哨(结束迭代器)具有不同的类型

\item
连接视图,若引用的范围生成值而不是引用,则使用该视图

\item
若引用的范围本身不是const可迭代的,则引用其他范围的所有视图
\end{itemize}
\end{itemize}

对于这些视图,begin()和end()作为const成员函数只能有条件地提供,或者根本不提供。对于丢弃视图,只有当传递的范围满足随机访问范围和大小范围的要求时,才会提供const对象的begin():[可能会将模板参数V限制为视图这一事实感到惊讶,但它可能是一个从传递的容器中隐式创建的视图引用,而不是视图]。


\begin{cpp}
namespace std::ranges {
	template<view V>
	class drop_view : public view_interface<drop_view<V>> {
		public:
		...
		constexpr auto begin() const requires random_access_range<const V>
		&& sized_range<const V>;
		...
	};
}
\end{cpp}

这里,使用新特性来约束成员函数。

若将形参声明为const引用,则无法提供可以处理所有范围和视图元素的泛型函数。

\begin{cpp}
void print(const auto& coll); // not callable for all views

template<typename T>
void foo(const T& coll); // not callable for all views
\end{cpp}

是否对范围参数的类型有约束并不重要:

\begin{cpp}
void print(const std::ranges::input_range auto& coll); // not callable for all views

template<std::ranges::random_access_range T>
void foo(const T& coll); // not callable for all views
\end{cpp}


\mySamllsection{非const\&\&对所有视图都有效}

为了在泛型代码中支持这些视图,应该将范围参数声明为通用引用(也称为转发引用)。这些引用可以引用所有表达式,同时保留引用对象不是const的事实[C++标准中,通用引用称为转发引用,但在std::forward<>()调用之前并不真正的转发,并且完美转发并不总是使用这些引用的原因(如本例所示)]。例如:

\begin{cpp}
void print(std::ranges::input_range auto&& coll); // can in principle pass all views

template<std::ranges::random_access_range T>
void foo(T&& coll);
\end{cpp}

因此,下面的程序可以正常工作:

\filename{ranges/printranges.cpp}

\begin{cpp}
#include <iostream>
#include <vector>
#include <list>
#include <ranges>

void print(auto&& rg)
{
	for (const auto& elem : rg) {
		std::cout << elem << ' ';
	}
	std::cout << '\n';
}

int main()
{
	std::vector vec{1, 2, 3, 4, 5, 6, 7, 8, 9};
	std::list lst{1, 2, 3, 4, 5, 6, 7, 8, 9};
	
	print(vec | std::views::take(3)); // OK
	print(vec | std::views::drop(3)); // OK
	
	print(lst | std::views::take(3)); // OK
	print(lst | std::views::drop(3)); // OK
	for (const auto& elem : lst | std::views::drop(3)) { // OK
		std::cout << elem << ' ';
	}
	std::cout << '\n';
	
	auto isEven = [] (const auto& val) {
		return val % 2 == 0;
	};
	print(vec | std::views::filter(isEven)); // OK
}
\end{cpp}

出于同样的原因,前面介绍的通用maxValue()函数并不总是有效:

\begin{cpp}
template<std::ranges::input_range Range>
std::ranges::range_value_t<Range> maxValue(const Range& rg)
{
	... // ERROR when iterating over the elements of a filter view
}

// max value of a filtered range:
auto odd = [] (auto val) {
					return val % 2 != 0;
				};
std::cout << maxValue(arr | std::views::filter(odd)) << '\n'; // ERROR
\end{cpp}

如前所述,最好实现泛型maxValue()函数如下:

\filename{ranges/maxvalue2.hpp}

\begin{cpp}
#include <ranges>

template<std::ranges::input_range Range>
std::ranges::range_value_t<Range> maxValue(Range&& rg)
{
	if (std::ranges::empty(rg)) {
		return std::ranges::range_value_t<Range>{};
	}
	auto pos = std::ranges::begin(rg);
	auto max = *pos;
	while (++pos != std::ranges::end(rg)) {
		if (*pos > max) {
			max = *pos;
		}
	}
	return max;
}
\end{cpp}

这里,将参数rg声明为通用/转发引用:

\begin{cpp}
template<std::ranges::input_range Range>
std::ranges::range_value_t<Range> maxValue(Range&& rg)
\end{cpp}

这样,就可以确保传递的视图不会变成const,所以现在也可以传递一个过滤视图:

\filename{ranges/maxvalue2.cpp}

\begin{cpp}
#include "maxvalue2.hpp"
#include <iostream>
#include <algorithm>

int main()
{
	int arr[] = {0, 8, 15, 42, 7};
	// max value of a filtered range:
	auto odd = [] (auto val) { // predicate for odd values
				return val % 2 != 0;
			};
	std::cout << maxValue(arr | std::views::filter(odd)) << '\n'; // OK
}
\end{cpp}

程序的输出是:

\begin{shell}
15
\end{shell}

现在想知道如何声明一个泛型函数,可以调用所有const和非const范围和视图,并保证元素不被修改?好吧,再一次,这里有一个重要的教训:

\textbf{C++20起,不再有一种方法可以声明一个泛型函数来接受所有标准集合类型(const和非const容器和视图),并保证不修改元素。}

所能做的就是获取范围/视图,并确保不在函数体中修改元素,但这样做可能会很复杂:

\begin{itemize}
\item
将视图声明为const并不一定使元素成为const。

\item
视图没有const\_iterator成员。

\item
视图目前还不提供cbegin()和cend()成员。

\item
std::ranges::cbegin()和Std::cbegin()对于视图是无效的。

\item
将视图元素声明为const可能没有效果。
\end{itemize}

\mySamllsection{对并发迭代使用const\&}

关于使用通用/转发引用的建议,有一个重要的限制:当同时遍历视图时不应该使用。

考虑下面的例子:

\begin{cpp}
std::list<int> lst{1, 2, 3, 4, 5, 6, 7, 8};

auto v = lst | std::views::drop(2);

// while another thread prints the elements of the view:
std::jthread printThread{[&] {
		for (const auto& elem : v) {
			std::cout << elem << '\n';
		}
}};

// this thread computes the sum of the element values:
auto sum = std::accumulate(v.begin(), v.end(), // fatal runtime ERROR
0L);
\end{cpp}

使用std::jthread,启动了一个线程,该线程遍历视图v的元素并进行输出。同时,遍历v来计算元素值的和。对于标准容器,只进行读操作的并行迭代是安全的(对于容器来说,调用begin()可以保证作为读访问计数),但此保证不适用于标准视图。因为可能会并发地为调用视图v的begin(),所以这段代码会导致未定义行为(可能的数据竞争)。

执行只读并发迭代的函数应该使用const视图,将可能的运行时错误转换为编译时错误:

\begin{cpp}
const auto v = lst | std::views::drop(2);

std::jthread printThread{[&] {
		for (const auto& elem : v) { // compile-time ERROR
			std::cout << elem << '\n';
		}
}};

auto sum = std::accumulate(v.begin(), v.end(), // compile-time ERROR
0L);
\end{cpp}

\mySamllsection{视图的重载}

读者们可能会声称可以简单地重载容器和视图的泛型函数。对于视图,可能只需要添加一个重载来约束视图的函数,并按值接受参数就足够了:

\begin{cpp}
void print(const auto& rg); // for containers
void print(std::ranges::view auto rg) // for views
\end{cpp}

但当一次按值传递,一次按引用传递时,重载解析规则可能会变得棘手,可能会导致歧义:

\begin{cpp}
std::vector vec{1, 2, 3, 4, 5, 6, 7, 8, 9};
print(vec | std::views::take(3)); // ERROR: ambiguous
\end{cpp}

对于包含用于一般情况的概念的视图,使用概念也没有帮助:

\begin{cpp}
void print(const std::ranges::range auto& rg); // for containers
void print(std::ranges::view auto rg) // for views

std::vector vec{1, 2, 3, 4, 5, 6, 7, 8, 9};
print(vec | std::views::take(3)); // ERROR: ambiguous
\end{cpp}

相反,必须声明const\&重载不适用于视图:

\begin{cpp}
template<std::ranges::input_range T> // for containers
requires (!std::ranges::view<T>) // and not for views
void print(const T& rg);

void print(std::ranges::view auto rg) // for views

std::vector vec{1, 2, 3, 4, 5, 6, 7, 8, 9};
print(vec | std::views::take(3)); // OK
\end{cpp}

复制视图可能会创建一个与源视图具有不同状态和行为的视图,所以按值传递所有视图都会有问题。

\mySubsubsection{6.5.2}{具有写访问权限的视图}

容器具有很深的稳定性,具有值语义并拥有自己的元素,所以可将任何常量性传播给其元素。当容器为const类型时,其元素也是const类型,所以以下代码无法编译:

\begin{cpp}
template<typename T>
void modifyConstRange(const T& range)
{
	range.front() += 1; // modify an element of a const range
}

std::array<int, 10> coll{}; // array of 10 ints with value 0
...
modifyConstRange(coll); // compile-time ERROR
\end{cpp}

这个编译时错误很有帮助,因为有助于检测在运行时可能破坏的代码。例如,若不小心使用赋值操作符而不是比较操作符,代码将无法编译:

\begin{cpp}
template<typename T>
void modifyConstRange(const T& range)
{
	if (range[0] = 0) { // OOPS: should be ==
		...
	}
}

std::array<int, 10> coll{}; // array of 10 ints with value 0
...
modifyConstRange(coll); // compile-time ERROR (thank goodness)
\end{cpp}

知道元素不能修改也有助于优化(避免备份和检查更改),或者确保在使用多个线程时,元素访问不会因数据竞争而导致未定义行为。

有了视图来说,情况就复杂多了。作为左值传递的范围的视图具有引用语义,引用存储在其他地方的元素。按照设计,这些视图不会将常量性传播到其元素上。

\begin{cpp}
std::array<int, 10> coll{}; // array of 10 ints with value 0
...
std::ranges::take_view v{coll, 5}; // view to first five elements of coll
modifyConstRange(v); // OK, modifies the first element of coll
\end{cpp}

这适用于几乎所有关于左值的视图,不管是如何创建的:

\begin{cpp}
modifyConstRange(coll | std::views::drop(5)); // OK, elements of coll are not const
modifyConstRange(coll | std::views::take(5)); // OK, elements of coll are not const
\end{cpp}

因此,可以使用范围适配器std::views::all()将容器传递给接受范围作为const引用的泛型函数:

\begin{cpp}
modifyConstRange(coll); // compile-time ERROR: elements are const
modifyConstRange(std::views::all(coll)); // OK, elements of coll are not const
\end{cpp}

对右值的视图(如果它们是有效的)通常仍然传播常量性:

\begin{cpp}
readConstRange(getColl() | std::views::drop(5)); // compile-time ERROR (good)
readConstRange(getColl() | std::views::take(5)); // compile-time ERROR (good)
\end{cpp}

另外,不能使用视图修改const容器的元素:

\begin{cpp}
const std::array<int, 10> coll{}; // array of 10 ints with value 0
...
auto v = std::views::take(coll, 5); // view to first five elements of coll
modifyConstRange(v); // compile-time ERROR (good)
\end{cpp}

因此,通过在视图引用容器之前将容器设置为const,可以确保容器的元素不会接受该视图的函数修改:

\begin{cpp}
std::array<int, 10> coll{}; // array of 10 ints with value 0
...
std::ranges::take_view v{std::as_const(coll), 5}; // view to five elems of const coll
modifyConstRange(v); // compile-time ERROR
modifyConstRange(std::as_const(coll) | std::views::take(5)); // compile-time ERROR
\end{cpp}

函数std::as\_const()从C++17开始在头文件<utility>中提供,但在创建视图后调用std::as\_const()没有效果(除了少数视图,不再能够遍历元素)。

C++23将引入一个带有范围适配器std::views::as\_const()的辅助视图std::ranges::as\_const\_view,就可以简单地将视图的元素设置为const,如下所示(参见\url{http://wg21.link/p2278r4}):

\begin{cpp}
std::views::as_const(v); // make elements of view v const
\end{cpp}

不能忘记这里的命名空间视图:

\begin{cpp}
std::as_const(v); // OOPS: no effect on views
\end{cpp}

命名空间std和命名空间std::ranges中的函数as\_const()都使某些东西成为const,但前者使对象成为const,而后者使元素成为const[C++标准委员会的成员有时决定支持看起来相同的特性,即使细节不同,也可能成为麻烦的来源。]读者们可能想知道,为什么使用引用语义的视图没有被设计成传播常量的方式。一个论点是在内部使用指针,因此应该表现得像指针。

然而,奇怪的是,右值的视图不是这样工作的。另一种说法是,const几乎没什么作用,其可以很容易地通过将视图复制到非const视图来删除:

\begin{cpp}
void foo(const auto& rg) // if rg would propagate constness for views on lvalues
{
	auto rg2 = rg; // rg2 would no longer propagate constness
	...
}
\end{cpp}

初始化的行为类似于const\_cast<>。但若这是容器的泛型代码,开发者已经习惯不复制传递的集合,这样做的代价很高。因此,现在有了另一个不复制传递的范围的好理由,并将范围(容器和视图)声明为const的效果将更加一致。然而,现在设计决策已经完成,作为开发者必须处理。

\mySubsubsection{6.5.3}{更改范围的视图}

正如所了解的那样,对作为视图的范围使用const是一个问题,原因有两个:

\begin{itemize}
\item
const可以禁用元素的迭代

\item
const可能不会传播到视图的元素
\end{itemize}

显而易见的问题是如何确保代码在使用视图时不能修改范围的元素。

\mySamllsection{使元素为const}

确保视图中的元素不能修改的唯一方法是,在元素访问时强制使用const:

\begin{itemize}
\item
基于范围的for循环中使用const:

\begin{cpp}
for (const auto& elem : rg) {
	...
}
\end{cpp}

\item
或通过传递转换为const的元素:

\begin{cpp}
for (auto pos = rg.begin(); pos != rg.end(); ++pos) {
	elemfunc(std::as_const(*pos));
}
\end{cpp}
\end{itemize}

然而,这里有一个坏消息:视图库的设计者计划禁用某些视图声明元素const的效果,比如即将到来的zip视图:

\begin{cpp}
for (const auto& elem : myZipView) {
	elem.member = value; // WTF: C++23 plans to let this compile
}
\end{cpp}

也许还能阻止这一切,但使用C++标准视图时,将集合或其元素设置为const可能不起作用。

\mySamllsection{const\_iterator和cbegin()未提供或无效}

不幸的是,通常将容器的所有元素都设置为const的方法不适用于视图:

\begin{itemize}
\item
通常,视图不提供const\_iterator来执行以下操作:

\begin{cpp}
for (decltype(rg)::const_iterator pos = rg.begin(); // not for views
pos != range.end();
++pos) {
	elemfunc(*pos);
}
\end{cpp}

因此,也不能简单地将整个范围转换为:

\begin{cpp}
std::ranges::subrange<decltype(rg)::const_iterator> crg{rg}; // not for views
\end{cpp}

\item
通常,视图没有成员函数cbegin()和cend()来执行以下操作:

\begin{cpp}
callTraditionalAlgo(rg.cbegin(), rg.cend()); // not for views
\end{cpp}
\end{itemize}

有的读者可能会建议使用独立的辅助函数std::cbegin()和std::cend(),从C++11开始就提供了。引入它们是为了确保元素在迭代时是const的,因为std::cbegin()和std::cend()对于视图来说是无效的,所以里的情况会更糟。其规范没有考虑到具有浅常量的类型(在内部,调用const成员函数begin(),并且不向值类型添加constness)。对于不传播常量的视图,这些函数被破坏了[这时应该修复std::cbegin()和std::cend(),但目前为止,C++标准委员会已经拒绝了所有修正它们的提案。]

\begin{cpp}
for (auto pos = std::cbegin(range); pos != std::cend(range); ++pos) {
	elemfunc(*pos); // does not provide constness for values of views
}
\end{cpp}

还要注意,在使用这些辅助函数时,ADL存在一些问题。C++20在命名空间std::ranges中引入了range库的相应帮助函数,但C++20中的视图中破坏了std::ranges::cbegin()和std::ranges::cend()(这将在C++23中修复):

\begin{cpp}
for (auto pos = std::ranges::cbegin(rg); pos != std::ranges::cend(rg); ++pos) {
	elemfunc(*pos); // OOPS: Does not provide constness for the values in C++20
}
\end{cpp}

因此,C++20中,当输入使视图的所有元素都为const时,还会遇到另一个严重的问题:

\textbf{在泛型代码中使用cbegin(),cend(),cdata()时要谨慎,这些函数在某些视图中不可用或已破坏。}

不管这是我们有这个问题,还是C++标准委员会的,及其范围库的子小组不愿意在C++20中修复这个问题。可以在std::ranges::view\_interface<>中提供const\_iterator和cbegin()成员,其作为所有视图的基类。有趣的是,到目前为止只有视图提供了const\_iterator支持:std::string\_view。具有讽刺意味的是,这或多或少是唯一不需要的视图,因为在字符串视图中,字符总是const。

然而,还是有一些希望的:\url{http://wg21.link/p2278}提供了一种方法来修复C++23中这个损坏的常量(而std::cbegin()和std::cend()不适用)使用这种方法,在泛型函数内部,可以将元素设为const,如下所示:

\begin{cpp}
void print(auto&& rgPassed)
{
	auto rg = std::views::as_const(rgPassed); // ensure all elements are const
	... // use rg instead of rgPassed now
}
\end{cpp}

另一种方法,可以执行以下操作:

\begin{cpp}
void print(R&& rg)
{
	if constexpr (std::ranges::const_range<R>) {
		// the passed range and its elements are constant
		...
	}
	else {
		// call this function again after making view and elements const:
		print(std::views::as_const(std::forward<R>(rg)));
	}
}
\end{cpp}

不幸的是,C++23仍然没有提供一种通用的方法来声明容器和视图的引用形参,以保证内部的元素是const。不能声明在其签名中保证不修改元素的函数print()。

