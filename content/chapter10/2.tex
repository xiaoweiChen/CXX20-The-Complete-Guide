
仍然使用sprintf()的一个原因是,其性能明显优于使用输出字符串流或std::to\_string()。格式化库的设计就是为了更好地完成这一任务,其格式化速度至少应该和sprintf()一样快,甚至更快。

目前的(草案)实现表明,相同甚至更好的性能是可能的。粗略的测量表明

\begin{itemize}
\item 
std::format()应该和sprintf()一样快,甚至更好。

\item 
std::format\_to()和std::format\_to\_n()应该更好
\end{itemize}

编译器通常可以在编译时检查格式字符串,这确实有很大帮助。有助于避免格式化错误,并获得更好的性能。

性能最终取决于特定平台上格式化库的实现质量。例如,在编写本书时,对于Visual C++,/utf-8选项显著提高了格式化性能,开发者应该自己确定性能。程序format/formatperf10.cpp可能会给您一些关于相关平台情况的提示。

\mySubsubsection{10.2.1}{使用std::vformat()和vformat\_to()}

为了支持这一目标,C++20标准化之后,对格式化库进行了一个重要的修复(参见\url{http://wg21.link/p2216r3})。通过此修复,函数std::format()、std::format\_to()和std::format\_to\_n()要求格式字符串是编译时值,所以必须传递一个字符串字面值或constexpr字符串。例如:

\begin{cpp}
const char* fmt1 = "{}\n"; // runtime format string
std::cout << std::format(fmt1, 42); // compile-time ERROR: runtime format

constexpr const char* fmt2 = "{}\n"; // compile-time format string
std::cout << std::format(fmt2, 42); // OK
\end{cpp}

因此,无效的格式规范成为编译时错误:

\begin{cpp}
std::cout << std::format("{:7.2f}\n", 42); // compile-time ERROR: invalid format

constexpr const char* fmt2 = "{:7.2f}\n"; // compile-time format string
std::cout << std::format(fmt2, 42); // compile-time ERROR: invalid format
\end{cpp}

当然,应用程序有时必须在运行时计算格式细节(例如根据传递的值计算最佳宽度),必须使用std::vformat()或std::vformat\_to(),并使用std::make\_format\_args()将所有参数传递给这些函数:

\begin{cpp}
const char* fmt3 = "{} {}\n"; // runtime format
std::cout << std::vformat(fmt3, std::make_format_args(42, 1.7)); // OK
\end{cpp}

若使用了运行时格式字符串,并且所传递的参数的格式无效,则调用会抛出std::format\_error类型的运行时错误:

\begin{cpp}
const char* fmt4 = "{:7.2f}\n";
std::cout << std::vformat(fmt4, std::make_format_args(42)); // runtime ERROR:
// throws std::format_error excep
\end{cpp}


