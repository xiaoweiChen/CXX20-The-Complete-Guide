
本节来详细介绍格式化的语法。

\mySubsubsection{10.3.1}{格式字符串的通用格式}

指定参数格式的一般方法是传递一个格式字符串,该字符串可以具有由\{…指定的替换字段。\}和字符。所有其他字符按原样打印。要打印字符\{和\},请使用\{\{和\}\}。

例如:

\begin{cpp}
std::format("With format {{}}: {}", 42); // yields "With format {}: 42"
\end{cpp}

替换字段可以有一个索引来指定参数,并在冒号后面有一个格式说明符:

\begin{itemize}
\item 
\{\}

使用带有默认格式的下一个参数

\item 
\{n\}

使用默认格式的第n个参数(第一个参数索引为0)

\item 
\{:fmt\}

使用根据fmt格式化的下一个参数

\item 
\{n:fmt\}

使用根据fmt格式化的第n个参数
\end{itemize}

可以不使用索引指定的任何参数,也可以使用全部参数:

\begin{cpp}
std::format("{}: {}", key, value); // OK
std::format("{1}: {0}", value, key); // OK
std::format("{}: {} or {0}", value, key); // ERROR
\end{cpp}

将忽略其他参数。

格式说明符的语法取决于所传递参数的类型。

\begin{itemize}
\item 
对于算术类型、字符串和原始指针,格式化库本身已经定义了标准格式。

\item 
此外,C++20还指定了时间类型(持续时间、时间点和日历类型)的标准格式。
\end{itemize}

\mySubsubsection{10.3.2}{标准格式说明符}

标准格式说明符有以下格式(每个说明符都是可选的):

\begin{shell}
fill align sign # 0 width .prec L type
\end{shell}

\begin{itemize}
\item 
fill是将值填充到宽度(默认:空格)的字符。它只能在同时指定align时指定。

\item 
align为

\begin{itemize}
\item 
< 左对齐

\item 
> 右对齐

\item 
\^{} 居中
\end{itemize}

默认对齐方式取决于类型。

\item 
sign为

\begin{itemize}
\item 
- 只有负值的负号(默认)

\item 
+ 正号或负号

\item 
space 负号或空格
\end{itemize}

\item 
\# 切换到某些符号的替代形式:

\begin{itemize}
\item 
为整数值的二进制、八进制和十六进制表示法写入0b、0和0x这样的前缀。

\item 
强制浮点表示法总是写一个点。
\end{itemize}

\item 
宽度前的0用零填充算术值。

\item 
width 指定最小字段宽度。

\item 
点后的prec指定精度:

\begin{itemize}
\item 
对于浮点类型,指定在点后面或总共打印多少位数字(取决于表示法)。

\item 
对于字符串类型,指定从字符串处理的最大字符数。
\end{itemize}

\item 
L 激活依赖格式(这可能会影响算术类型和bool)的格式

\item 
type 指定格式化的通用表示法,将字符打印为整数值(反之亦然)或选择浮点值的通用表示法。
\end{itemize}

\mySubsubsection{10.3.3}{宽度、精度和填充字符}

对于所有打印的值,冒号后面的正整数值(不带前导点)指定了整个值输出的最小字段宽度(包括符号等)。它可以与对齐规范一起使用:

例如:

\begin{cpp}
std::format("{:7}", 42);     // yields "           42"
std::format("{:7}", "hi");   // yields "hi           "
std::format("{:^7}", "hi");  // yields "      hi     "
std::format("{:>7}", "hi");  // yields "           hi"
\end{cpp}

还可以指定填充零和填充字符。填充0只能用于算术类型(char和bool除外),若果指定了对齐方式,则忽略填充0:

\begin{cpp}
std::format("{:07}", 42);    // yields "0000042"
std::format("{:^07}", 42);   // yields "  42   "
std::format("{:>07}", -1);   // yields "     -1"
\end{cpp}

填充0不同于一般的填充字符,可以在冒号之后立即指定(在对齐的前面):

\begin{cpp}
std::format("{:^07}", 42);   // yields "  42   "
std::format("{:0^7}", 42);   // yields "0042000"
std::format("{:07}", "hi");  // invalid (padding 0 not allowed for strings)
std::format("{:0<7}", "hi"); // yields "hi00000"
\end{cpp}

精度用于浮点类型和字符串:

\begin{itemize}
\item 
对于浮点类型,可以指定一个精度(默认值为6):

\begin{cpp}
std::format("{}", 0.12345678);         // yields "0.12345678"
std::format("{:.5}", 0.12345678);      // yields "0.12346"
std::format("{:10.5}", 0.12345678);    // yields "   0.12346"
std::format("{:^10.5}", 0.12345678);   // yields " 0.12346  "
\end{cpp}

请注意,根据浮点表示法的不同,精度可能适用于整个值,也可能适用于点之后的数字。

\item 
对于字符串,可以用来指定最大字符数:

\begin{cpp}
std::format("{}", "counterproductive");            // yields "counterproductive"
std::format("{:20}", "counterproductive");         // yields "counterproductive    "
std::format("{:.7}", "counterproductive");         // yields "counter"
std::format("{:20.7}", "counterproductive");       // yields "counter              "
std::format("{:^20.7}", "counterproductive");      // yields "        counter      "
\end{cpp}
\end{itemize}

注意,宽度和精度本身可以作为参数:

\begin{cpp}
int width = 10;
int precision = 2;
for (double val : {1.0, 12.345678, -777.7}) {
	std::cout << std::format("{:+{}.{}f}\n", val, width, precision);
}
\end{cpp}

输出如下:

\begin{shell}
    +1.00
   +12.35
  -777.70
\end{shell}

这里,在运行时指定最小字段宽度为10,点后面有两个数字(使用固定的表示法)。

\mySubsubsection{10.3.4}{格式/类型说明符}

通过指定格式或类型说明符,可以强制为整型、浮点型和原始指针使用不同的符号。

\mySamllsection{整型的说明符}

整型的表格式化选项列出了整型(包括bool和char)可能的格式化类型选项。[根据\url{http://wg21.link/lwg3648},对bool的说明符c的支持可能是一个错误,将被删除。]

% Please add the following required packages to your document preamble:
% \usepackage{longtable}
% Note: It may be necessary to compile the document several times to get a multi-page table to line up properly
\begin{longtable}[c]{|l|l|l|l|l|}
\hline
\textbf{格式} & \textbf{42} & \textbf{'@'} & \textbf{true}       & \textbf{意义}                 \\ \hline
\endfirsthead
%
\endhead
%
none           & 42          & @            & true                & 默认 format                   \\ \hline
d              & 42          & 64           & 1                   & 十进制 notation                 \\ \hline
b/B            & 101010      & 1000000      & 1                   & 二进制 notation                  \\ \hline
\#b            & 0b101010    & 0b1000000    & 0b1                 & 带前缀的二进制表示法      \\ \hline
\#B            & 0B101010    & 0B1000000    & 0B1                 & 带前缀的二进制表示法      \\ \hline
o              & 52          & 100          & 1                   & 八进制表示法                   \\ \hline
x              & 2a          & 40           & 1                   & 十六进制表示法             \\ \hline
X              & 2A          & 40           & 1                   & 十六进制表示法             \\ \hline
\#x            & 0x2a        & 0x40         & 0x1                 & 带前缀的十六进制表示法 \\ \hline
\#X            & 0X2A        & 0X40         & 0X1                 & 带前缀的十六进制表示法 \\ \hline
c              & *           & @            & '\textbackslash{}1' & 作为值的字符      \\ \hline
s              & 非法     & 非法      & true                & 布尔值作为字符串                   \\ \hline
\end{longtable}

\begin{center}
表10.1 整型的格式化选项
\end{center}

例如:

\begin{cpp}
std::cout << std::format("{:#b} {:#b} {:#b}\n", 42, '@', true);
\end{cpp}

输出:

\begin{shell}
0b101010 0b1000000 0b1
\end{shell}

注意事项如下:

\begin{itemize}
\item 
默认的符号是:

\begin{itemize}
\item 
d (十进制)的整数类型

\item 
c (作为字符)用于字符类型

\item 
s (作为字符串)用于bool类型
\end{itemize}

\item 
若在表示法之后指定L,则使用与语言环境相关的布尔值字符序列,和与语言环境相关的千位分隔符和小数点字符来表示算术值。
\end{itemize}

\mySamllsection{浮点类型的说明符}

表浮点类型的格式化选项,列出了浮点类型可能的格式化类型选项。

例如:

\begin{cpp}
std::cout << std::format("{0} {0:#} {0:#g} {0:e}\n", -1.0);
\end{cpp}

输出:

\begin{shell}
-1 -1. -1.00000 -1.000000e+00
\end{shell}

注意,传递整数-1会导致格式错误。

% Please add the following required packages to your document preamble:
% \usepackage{longtable}
% Note: It may be necessary to compile the document several times to get a multi-page table to line up properly
\begin{longtable}[c]{|l|l|l|l|l|}
\hline
\textbf{格式} & \textbf{-1.0} & \textbf{0.0009765625} & \textbf{1785856.0} & \textbf{意义}                                         \\ \hline
\endfirsthead
%
\endhead
%
none           & -1            & 0.0009765625          & 1.785856e+06       & 默认格式                                           \\ \hline
\#             & -1.           & 0.0009765628          & 1.785856e+06       & 强制十进制小数点                                     \\ \hline
f/F            & -1.000000     & 0.000977              & 1785856.000000     & 固定小数点位置(默认点后精度:6)           \\ \hline
g              & -1            & 0.000976562           & 1.78586e+06        & 固定或指数表示法(默认全精度:6) \\ \hline
G              & -1            & 0.000976562           & 1.78586E+06        & 固定或指数表示法(默认全精度:6)  \\ \hline
\#g            & -1.00000      & 0.000976562           & 1.78586e+06        & 固定或指数符号(强制点和零)      \\ \hline
\#G            & -1.00000      & 0.000976562           & 1.78586E+06        & 固定或指数符号(强制点和零)      \\ \hline
e              & -1.00000e00   & 9.765625e-04          & 1.7858560e+06      & 指数表示法(默认点后精度:6)     \\ \hline
E              & -1.00000E00   & 9.765625E-04          & 1.7858560E+06      & 指数表示法(默认点后精度:6)     \\ \hline
a              & -1p+0         & 1p-10                 & 1.b4p+20           & 十六进制。浮点表示法                         \\ \hline
A              & -1P+0         & 1P-10                 & 1.B4P+20           & 十六进制。浮点表示法                         \\ \hline
\#a            & -1.p+0        & 1.p-10                & 1.b4p+20           & 十六进制。浮点表示法                         \\ \hline
\#A            & -1.P+0        & 1.P-10                & 1.B4P+20           & 十六进制。浮点表示法                         \\ \hline
\end{longtable}

\begin{center}
表10.2 浮点类型的格式化选项
\end{center}

\mySamllsection{字符串的说明符}

对于字符串类型,默认的格式说明符是s。因为它是默认的,所以不必提供此说明符。请注意,对于字符串,可以指定一定的精度,这将解释为使用的最大字符数:

\begin{cpp}
std::format("{}", "counter"); // yields "counter"
std::format("{:s}", "counter"); // yields "counter"
std::format("{:.5}", "counter"); // yields "count"
std::format("{:.5}", "hi"); // yields "hi"
\end{cpp}

注意,只支持字符类型char和wchar\_t的标准字符串类型。不支持类型为u8string和char8\_t、u16string和char16\_t或u32string和char32\_t的字符串和序列。实际上,C++标准库为以下类型提供了格式化器:

\begin{itemize}
\item 
char* 和 const char*

\item 
const char[n] (字符串字面值)

\item 
std::string 和 std::basic\_string<char, traits, allocator>

\item 
std::string\_view 和 std::basic\_string\_view<char, traits>

\item 
wchar\_t* 和 const wchar\_t*

\item 
const wchar\_t[n] (宽字符串字面值)

\item 
std::wstring 和 std::basic\_string<wchar\_t, traits, allocator>

\item 
std::wstring\_view 和 std::basic\_string\_view<wchar\_t, traits>
\end{itemize}

注意,其格式字符串和参数必须具有相同的字符类型:

\begin{cpp}
auto ws1 = std::format("{}", L"K\u00F6ln"); // compile-time ERROR
std::wstring ws2 = std::format(L"{}", L"K\u00F6ln"); // OK
\end{cpp}

\mySamllsection{指针的说明符}

对于指针类型,默认的格式说明符是p,通常以十六进制表示法写地址,前缀为0x。在没有uintptr\_t类型的平台上,格式由实现定义:

\begin{cpp}
void* ptr = ... ;
std::format("{}", ptr) // usually yields a value such as 0x7ff688ee64
std::format("{:p}", ptr) // usually yields a value such as 0x7ff688ee64
\end{cpp}

请注意,只支持以下指针类型:

\begin{itemize}
\item 
void* 和 const void*

\item 
std::nullptr\_t
\end{itemize}

可以传递nullptr或原始指针,所以将其强制转换为类型(const) void*:

\begin{cpp}
int i = 42;
std::format("{}", &i) // compile-time error
std::format("{}", static_cast<void*>(&i)) // OK (e.g., 0x7ff688ee64)
std::format("{:p}", static_cast<void*>(&i)) // OK (e.g., 0x7ff688ee64)
std::format("{}", static_cast<const void*>("hi")) // OK (e.g., 0x7ff688ee64)
std::format("{}", nullptr) // OK (usually 0x0)
std::format("{:p}", nullptr) // OK (usually 0x0)
\end{cpp}
